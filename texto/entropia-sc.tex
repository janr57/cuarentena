% entropia-sc.tex
%
% Copyright (C) 2022-2025 José A. Navarro Ramón <janr.devel@gmail.com>
% 1) Código LuaLatex:
%    Licencia GPL-3.
% 2) Producto en pdf, postscript, etc.:
%    Licencia Creative Commons Recognition Share alike. (CC-BY-SA)

\chapter{Entropía en sistemas cuánticos}
En este capítulo vamos a estudiar el concepto de entropía centrándonos casi
exclusivamente en sistemas cuánticos.

\section{Concepto básico de entropía}
Explicaremos el significado general de la entropía, ya sea en sistemas
clásicos, cuánticos, en informática, etc.

\subsection{Ganancia de información}
Primero nos centraremos en el concepto de \emph{ganancia de información}.
Vamos a imaginar que tenemos una máquina un tanto aburrida, por predecible,
que llamaremos ``tres'', y cada vez que pulsamos el botón rojo, nos devolverá
siempre el mismo número, véase la figura~\ref{fig:entr-maquina-tres}.
\begin{figure}[ht]
  \centering
  % Colores
  \newcommand{\colorMaquinaInt}{green!70!black!40}
  \newcommand{\colorMaquinaExt}{black}
  \newcommand{\colorEtiquetaInt}{green!70!black!35}
  \newcommand{\colorEtiquetaExt}{green!45!black}
  \newcommand{\colorBotonInt}{red}
  \newcommand{\colorBotonExt}{black}
  \newcommand{\colorFlechaSalida}{black!60}
  \newcommand{\colorNumeroInt}{orange!40}
  \newcommand{\colorNumeroExt}{black!60}
  %
  \def\scl{1}
  % MACROS
  % Base y altura de la máquina
  \pgfmathsetmacro{\MQBASE}{1.3}
  \pgfmathsetmacro{\MQALT}{1.5}
  % Altura del objetivo interno y externo
  \pgfmathsetmacro{\MQINTALT}{.4}
  \pgfmathsetmacro{\MQEXTALT}{.8}
  % Base del objetivo
  \pgfmathsetmacro{\MQOBJBASE}{.5}
  % Posiciones inferior y superior del objetivo interno
  \pgfmathsetmacro{\MQSUPMED}{(\MQALT + \MQINTALT)/2}
  \pgfmathsetmacro{\MQINFMED}{(\MQALT - \MQINTALT)/2}
  % Posiciones inferior y superior del objetivo externo
  \pgfmathsetmacro{\MQOBJINF}{(\MQALT - \MQEXTALT)/2}
  \pgfmathsetmacro{\MQOBJSUP}{(\MQALT + \MQEXTALT)/2}
  % Posición relativa del botón de la máquina
  \pgfmathsetmacro{\MQBOTONX}{.4 * \MQBASE}
  \pgfmathsetmacro{\MQBOTONY}{.2 * \MQALT}
  % Posición relativa del etiqueta de la máquina
  \pgfmathsetmacro{\MQETIQX}{.5 * \MQBASE}
  \pgfmathsetmacro{\MQETIQY}{.8 * \MQALT}

  % Radio del botón en 'pt'
  \pgfmathsetmacro{\MQBOTONRADIO}{2}
  %
  \begin{tikzpicture}[%
    scale=\scl,
    every node/.style={font=\footnotesize},
    maquina/.style={fill=\colorMaquinaInt,draw=\colorMaquinaExt},
    etiqueta/.style={%
      fill=\colorEtiquetaInt,
      draw=\colorEtiquetaExt,
      line width=.4pt,
      draw,
    },
    boton/.style={fill=\colorBotonInt,draw=\colorBotonExt},
    flecha/.style={%
      -{Latex[round]},
      draw=\colorFlechaSalida,
      shorten >=4pt,
      shorten <= 1.5pt,
    },
    numero/.style={%
      circle,
      draw,
      inner sep=2pt,
      fill=\colorNumeroInt,
      draw=\colorNumeroExt,
      font=\small,
    },
    background/.style={
      line width=\bgborderwidth,
      draw=\bgbordercolor,
      fill=\bgcolor,
    },
    ]
    % COORDENADAS
    % Origen
    \coordinate (o) at (0,0);
    % Máquina
    \coordinate (mqa) at ($(o)$);
    \coordinate (mqb) at ($(mqa)+(\MQBASE,0)$);
    \coordinate (mqc) at ($(mqa)+(\MQBASE,\MQINFMED)$);
    \coordinate (mqod) at ($(mqa)+(\MQBASE + \MQOBJBASE,\MQOBJINF)$);
    \coordinate (mqoe) at ($(mqa)+(\MQBASE + \MQOBJBASE,\MQOBJSUP)$);
    \coordinate (mqd) at ($(mqa)+(\MQBASE,\MQSUPMED)$);
    \coordinate (mqe) at ($(mqa)+(\MQBASE,\MQALT)$);
    \coordinate (mqf) at ($(mqa)+(0,\MQALT)$);
    \coordinate (mqboton) at ($(mqa)+(\MQBOTONX,\MQBOTONY)$);
    \coordinate (mqetiq) at ($(mqa)+(\MQETIQX,\MQETIQY)$);
    \coordinate (mqsalida) at ($(mqa)+(\MQBASE + \MQOBJBASE,\MQALT/2)$);

    % DIBUJO
    % Máquina
    \draw[maquina]
    (mqa)--(mqb)--(mqc)--(mqod)--(mqoe)--(mqd)--(mqe)--(mqf)--cycle;
    % Botón
    \filldraw[boton] (mqboton) circle[radius=\MQBOTONRADIO pt];
    % Etiqueta
    \node[etiqueta] at (mqetiq) {Tres};
    % Flecha salida: número 3
    \draw[flecha] (mqsalida) -- +(0:1) node[right,name=num,numero] {$3$};
    
    % Fondo amarillo
    \coordinate (liminf) at ($(mqa)+(0,-8pt)$);
    \coordinate (limsup) at ($(mqf)+(0,8pt)$);
    \coordinate (limizda) at ($(mqa)+(-8pt,0)$);
    \coordinate (limdcha) at ($(num.east)+(10pt,0)$);
    \begin{scope}[on background layer]
      %% \node [line width=1pt, draw=\backgroundbordercolor,
      %% fill=\backgroundcolor, fit= (O) (letraejex) (letraejey)] {};
      \node [background, fit= (liminf) (limsup) (limizda) (limdcha)] {};
    \end{scope}
  \end{tikzpicture}
  \caption{Máquina que produce el número tres cada vez que se pulsa el
    botón.}
  \label{fig:entr-maquina-tres}
\end{figure}

La probabilidad de obtener el número tres es $p=1$, certeza absoluta.
Si informáramos a una persona acerca de las características de la máquina
y esta pulsara el botón, observaría que sale el tres y no se llevaría
ninguna sorpresa. Diremos que este resultado no le produce ninguna
\emph{ganacia de información}.


Inventaremos una función, $I$, que exprese la ganancia de información.
En esta circunstancia debería dar $I=0$. Hay muchas formas de obtener una
función así, pero la que se ha encontrado más útil es definirla como el menos
logaritmo de la probabilidad de ocurrencia de un suceso
\begin{equation}
  I = -\ln p
\end{equation}
En el caso de nuestra máquina, vemos que se cumple lo que nos proponíamos
\[
  I = -\ln p = -\ln 1 = 0
\]

Para esta máquina, la probabilidad de obtener cualquier otro número vale cero.
Por ejemplo, la probabilidad de obtener el número siete es cero, como se
observa en la figura~\ref{fig:entr-maquina-tres-restoprob-cero}
(podríamos añadir más probabilidades nulas, pero es suficiente con la
del número siete para nuestro propósito).
\begin{figure}[ht]
  \centering
  % Colores
  \newcommand{\colorMaquinaInt}{green!70!black!40}
  \newcommand{\colorMaquinaExt}{black}
  \newcommand{\colorEtiquetaInt}{green!70!black!35}
  \newcommand{\colorEtiquetaExt}{green!45!black}
  \newcommand{\colorBotonInt}{red}
  \newcommand{\colorBotonExt}{black}
  \newcommand{\colorFlechaSalida}{black!60}
  \newcommand{\colorNumeroInt}{orange!40}
  \newcommand{\colorNumeroExt}{black!60}
  \newcommand{\colorNumFalsoInt}{black!10}
  \newcommand{\colorNumFalsoExt}{black!20}
  %
  \def\scl{1}
  % MACROS
  % Base y altura de la máquina
  \pgfmathsetmacro{\MQBASE}{1.3}
  \pgfmathsetmacro{\MQALT}{1.5}
  % Altura del objetivo interno y externo
  \pgfmathsetmacro{\MQINTALT}{.4}
  \pgfmathsetmacro{\MQEXTALT}{.8}
  % Base del objetivo
  \pgfmathsetmacro{\MQOBJBASE}{.5}
  % Posiciones inferior y superior del objetivo interno
  \pgfmathsetmacro{\MQSUPMED}{(\MQALT + \MQINTALT)/2}
  \pgfmathsetmacro{\MQINFMED}{(\MQALT - \MQINTALT)/2}
  % Posiciones inferior y superior del objetivo externo
  \pgfmathsetmacro{\MQOBJINF}{(\MQALT - \MQEXTALT)/2}
  \pgfmathsetmacro{\MQOBJSUP}{(\MQALT + \MQEXTALT)/2}
  % Posición relativa del botón de la máquina
  \pgfmathsetmacro{\MQBOTONX}{.4 * \MQBASE}
  \pgfmathsetmacro{\MQBOTONY}{.2 * \MQALT}
  % Posición relativa del etiqueta de la máquina
  \pgfmathsetmacro{\MQETIQX}{.5 * \MQBASE}
  \pgfmathsetmacro{\MQETIQY}{.8 * \MQALT}
  % Alcance horizontal bolas
  \pgfmathsetmacro{\BOLAX}{\MQBASE + 3}

  % Radio del botón en 'pt'
  \pgfmathsetmacro{\MQBOTONRADIO}{2}
  %
  \begin{tikzpicture}[%
    scale=\scl,
    every node/.style={font=\footnotesize},
    maquina/.style={fill=\colorMaquinaInt,draw=\colorMaquinaExt},
    etiqueta/.style={%
      fill=\colorEtiquetaInt,
      draw=\colorEtiquetaExt,
      line width=.4pt,
      draw,
    },
    boton/.style={fill=\colorBotonInt,draw=\colorBotonExt},
    flecha/.style={%
      -{Latex[round]},
      draw=\colorFlechaSalida,
      shorten >=4pt,
      shorten <= 1.5pt,
    },
    numero/.style={%
      circle,
      draw,
      inner sep=2pt,
      fill=\colorNumeroInt,
      draw=\colorNumeroExt,
      font=\small,
    },
    numFalso/.style={%
      circle,
      draw,
      inner sep=2pt,
      fill=\colorNumFalsoInt,
      draw=\colorNumFalsoExt,
      font=\small,
    },
    background/.style={
      line width=\bgborderwidth,
      draw=\bgbordercolor,
      fill=\bgcolor,
    },
    ]
    % COORDENADAS
    % Origen
    \coordinate (o) at (0,0);
    % Máquina
    \coordinate (mqa) at ($(o)$);
    \coordinate (mqb) at ($(mqa)+(\MQBASE,0)$);
    \coordinate (mqc) at ($(mqa)+(\MQBASE,\MQINFMED)$);
    \coordinate (mqod) at ($(mqa)+(\MQBASE + \MQOBJBASE,\MQOBJINF)$);
    \coordinate (mqoe) at ($(mqa)+(\MQBASE + \MQOBJBASE,\MQOBJSUP)$);
    \coordinate (mqd) at ($(mqa)+(\MQBASE,\MQSUPMED)$);
    \coordinate (mqe) at ($(mqa)+(\MQBASE,\MQALT)$);
    \coordinate (mqf) at ($(mqa)+(0,\MQALT)$);
    \coordinate (mqboton) at ($(mqa)+(\MQBOTONX,\MQBOTONY)$);
    \coordinate (mqetiq) at ($(mqa)+(\MQETIQX,\MQETIQY)$);
    \coordinate (mqsalida) at ($(mqa)+(\MQBASE + \MQOBJBASE,\MQALT/2)$);

    % DIBUJO
    % Máquina
    \draw[maquina]
    (mqa)--(mqb)--(mqc)--(mqod)--(mqoe)--(mqd)--(mqe)--(mqf)--cycle;
    % Botón
    \filldraw[boton] (mqboton) circle[radius=\MQBOTONRADIO pt];
    % Etiqueta
    \node[etiqueta] at (mqetiq) {Tres};
    % Salida: números 3 y 7
    \draw[flecha] (mqsalida) -- +(30:1) node[right,name=tres,numero] {$3$};
    \path (mqsalida) -- +(-30:1) node[right,name=siete,numFalso] {$7$};
    % Probabilidades
    \node[right=2em,name=p1] at (tres) {$p_1=1$};
    \node[right=2em,name=p2] at (siete) {$p_2=0$};
    
    % Fondo amarillo
    \coordinate (liminf) at ($(mqa)+(0,-8pt)$);
    \coordinate (limsup) at ($(mqf)+(0,8pt)$);
    \coordinate (limizda) at ($(mqa)+(-8pt,0)$);
    \coordinate (limdcha) at ($(p1.east)+(10pt,0)$);
    \begin{scope}[on background layer]
      %% \node [line width=1pt, draw=\backgroundbordercolor,
      %% fill=\backgroundcolor, fit= (O) (letraejex) (letraejey)] {};
      \node [background, fit= (liminf) (limsup) (limizda) (limdcha)] {};
    \end{scope}
  \end{tikzpicture}
  \caption{Máquina que proporciona el número tres cada vez que se pulsa el
    botón y además se tiene en cuenta la probabilidad nula de que salga
    cualquier otro número, por ejemplo el siete.}
  \label{fig:entr-maquina-tres-restoprob-cero}
\end{figure}

Aparentemente, esto nos conduce a una incongruencia. Vamos a calcular la
ganancia de información si saliera el número siete, sabiendo que su
probabilidad de salida es nula
\[
  I_2 = -\ln p_2 = -\ln 0 \to \infty
\]

Resulta que sale una ganancia de información infinita. Hay una forma
divertida de explicar este cálculo:
si al pulsar el botón de la máquina del tres, saliera un siete,
¡nuestra sorpresa sería infinita!, pero este resultado sería  absurdo
ya que entonces la probabilidad no sería cero, como pensábamos.
Una forma más realista de explicarlo sería decir que hemos dado con una
discontinuidad de la función $I$ y que esta resultado no tendría sentido
(no podría producirse).

Pero no seamos tan extremos, supongamos nuestra máquina se llama ``tres''
y ``siete'' y que la probabilidad de que salga el tres es muy elevada, por
ejemplo un \SI{99,9}{\percent}
\[
  p_1=0,999
\]
y la de que saliera un siete es muy baja, un $\SI{0,1}{\percent}$
\[
  p_2 = 1 - p_1 = 1-0,999 = 0,001
\]
\begin{figure}[ht]
  \centering
  % Colores
  \newcommand{\colorMaquinaInt}{green!70!black!40}
  \newcommand{\colorMaquinaExt}{black}
  \newcommand{\colorEtiquetaInt}{green!70!black!35}
  \newcommand{\colorEtiquetaExt}{green!45!black}
  \newcommand{\colorBotonInt}{red}
  \newcommand{\colorBotonExt}{black}
  \newcommand{\colorFlechaSalida}{black!60}
  \newcommand{\colorNumeroInt}{orange!40}
  \newcommand{\colorNumeroExt}{black!60}
  \newcommand{\colorNumFalsoInt}{black!10}
  \newcommand{\colorNumFalsoExt}{black!20}
  %
  \def\scl{1}
  % MACROS
  % Base y altura de la máquina
  \pgfmathsetmacro{\MQBASE}{1.3}
  \pgfmathsetmacro{\MQALT}{1.5}
  % Altura del objetivo interno y externo
  \pgfmathsetmacro{\MQINTALT}{.4}
  \pgfmathsetmacro{\MQEXTALT}{.8}
  % Base del objetivo
  \pgfmathsetmacro{\MQOBJBASE}{.5}
  % Posiciones inferior y superior del objetivo interno
  \pgfmathsetmacro{\MQSUPMED}{(\MQALT + \MQINTALT)/2}
  \pgfmathsetmacro{\MQINFMED}{(\MQALT - \MQINTALT)/2}
  % Posiciones inferior y superior del objetivo externo
  \pgfmathsetmacro{\MQOBJINF}{(\MQALT - \MQEXTALT)/2}
  \pgfmathsetmacro{\MQOBJSUP}{(\MQALT + \MQEXTALT)/2}
  % Posición relativa del botón de la máquina
  \pgfmathsetmacro{\MQBOTONX}{.4 * \MQBASE}
  \pgfmathsetmacro{\MQBOTONY}{.2 * \MQALT}
  % Posición relativa del etiqueta de la máquina
  \pgfmathsetmacro{\MQETIQX}{.5 * \MQBASE}
  \pgfmathsetmacro{\MQETIQY}{.8 * \MQALT}
  % Alcance horizontal bolas
  \pgfmathsetmacro{\BOLAX}{\MQBASE + 3}

  % Radio del botón en 'pt'
  \pgfmathsetmacro{\MQBOTONRADIO}{2}
  %
  \begin{tikzpicture}[%
    scale=\scl,
    every node/.style={font=\footnotesize},
    maquina/.style={fill=\colorMaquinaInt,draw=\colorMaquinaExt},
    etiqueta/.style={%
      fill=\colorEtiquetaInt,
      draw=\colorEtiquetaExt,
      line width=.4pt,
      draw,
    },
    boton/.style={fill=\colorBotonInt,draw=\colorBotonExt},
    flecha/.style={%
      -{Latex[round]},
      draw=\colorFlechaSalida,
      shorten >=4pt,
      shorten <= 1.5pt,
    },
    numero/.style={%
      circle,
      draw,
      inner sep=2pt,
      fill=\colorNumeroInt,
      draw=\colorNumeroExt,
      font=\small,
    },
    numFalso/.style={%
      circle,
      draw,
      inner sep=2pt,
      fill=\colorNumFalsoInt,
      draw=\colorNumFalsoExt,
      font=\small,
    },
    background/.style={
      line width=\bgborderwidth,
      draw=\bgbordercolor,
      fill=\bgcolor,
    },
    ]
    % COORDENADAS
    % Origen
    \coordinate (o) at (0,0);
    % Máquina
    \coordinate (mqa) at ($(o)$);
    \coordinate (mqb) at ($(mqa)+(\MQBASE,0)$);
    \coordinate (mqc) at ($(mqa)+(\MQBASE,\MQINFMED)$);
    \coordinate (mqod) at ($(mqa)+(\MQBASE + \MQOBJBASE,\MQOBJINF)$);
    \coordinate (mqoe) at ($(mqa)+(\MQBASE + \MQOBJBASE,\MQOBJSUP)$);
    \coordinate (mqd) at ($(mqa)+(\MQBASE,\MQSUPMED)$);
    \coordinate (mqe) at ($(mqa)+(\MQBASE,\MQALT)$);
    \coordinate (mqf) at ($(mqa)+(0,\MQALT)$);
    \coordinate (mqboton) at ($(mqa)+(\MQBOTONX,\MQBOTONY)$);
    \coordinate (mqetiq) at ($(mqa)+(\MQETIQX,\MQETIQY)$);
    \coordinate (mqsalida) at ($(mqa)+(\MQBASE + \MQOBJBASE,\MQALT/2)$);

    % DIBUJO
    % Máquina
    \draw[maquina]
    (mqa)--(mqb)--(mqc)--(mqod)--(mqoe)--(mqd)--(mqe)--(mqf)--cycle;
    % Botón
    \filldraw[boton] (mqboton) circle[radius=\MQBOTONRADIO pt];
    % Etiqueta
    \node[etiqueta,inner sep=1pt] at (mqetiq) {\scriptsize Tres y siete};
    % Salida: números 3 y 7
    \draw[flecha] (mqsalida) -- +(30:1) node[right,name=tres,numero] {$3$};
    \draw[flecha] (mqsalida) -- +(-30:1) node[right,name=siete,numero] {$7$};
    % Probabilidades
    \node[right=2em,name=p1] at (tres) {$p_1=0,999$};
    \node[right=2em,name=p2] at (siete) {$p_2=0,001$};
    
    % Fondo amarillo
    \coordinate (liminf) at ($(mqa)+(0,-8pt)$);
    \coordinate (limsup) at ($(mqf)+(0,8pt)$);
    \coordinate (limizda) at ($(mqa)+(-8pt,0)$);
    \coordinate (limdcha) at ($(p1.east)+(10pt,0)$);
    \begin{scope}[on background layer]
      %% \node [line width=1pt, draw=\backgroundbordercolor,
      %% fill=\backgroundcolor, fit= (O) (letraejex) (letraejey)] {};
      \node [background, fit= (liminf) (limsup) (limizda) (limdcha)] {};
    \end{scope}
  \end{tikzpicture}
  \caption{Máquina que proporciona el número tres con una probabilidad muy
    alta (novecientas noventa y nueve veces de cada mil tiradas), y el siete
    muy raramente (una de cada mil).}
  \label{fig:entr-maquina-tres-siete}
\end{figure}

En esta máquina, la ganancia de información que obtendríamos si saliera un
siete sería relativamente elevada (porque no lo esperábamos)
\[
  I_2 = -\ln 0,001 \approx 6,908
\]
mientras que la de obtener el número tres sería muy baja, porque no
sería casi sorpresa
\[
  I_1 = -\ln 0,999 \approx 0,001
\]

\subsection{Definición básica de entropía}
\label{subsect:entr-entropia-basica}
Definimos el concepto de \emph{entropía}, que representaremos con el símbolo
$S$
\begin{quote}
  ``La entropía es el promedio de ganancia de información.''
\begin{equation}\label{eq:entr-def-entropia-basica}
  S = \braket{I} = \sum_{i=1}^N p_i I_i = -\sum_{i=1}^N p_i \ln p_i
\end{equation}
\end{quote}
La entropía se podría interpretar como el \emph{grado de ignorancia} que
tenemos sobre un sistema. Volviendo al ejemplo de la máquina anterior
(figura~\ref{fig:entr-maquina-tres}), la entropía será
\[
  S = -p \ln p = -1 \ln 1 = 0
\]
que se puede interpretar como que nuestro grado de ignorancia es cero,
porque ya sabíamos lo que se iba a producir con un \SI{100}{\percent} de
seguridad.

Vamos a aclarar una duda matemática que puede surgir con el ejemplo anterior.
Fijémonos en la figura~\ref{fig:entr-maquina-tres-siete}, donde teníamos
en cuenta, además, la probabilidad de que saliera el siete.
La entropía será
\[
  S = -p_1\ln p_1 - p_2\ln p_2 = -1\ln 1 - 0\ln 0
\]

Podríamos pensar que el factor con logaritmo natural de cero, $\ln 0$ nos iba a
impedir el cálculo de la entropía, debido al \emph{menos infinito que produce}.

Pero ese factor va acompañado de otro que vale cero, y el logaritmo natural
se acerca a menos infinito más despacio que el cero a su valor.
Matemáticamente se trata de calcular el siguiente límite
\[
  "0\ln 0" = \lim_{x\to 0} x\ln x = 0
\]

Para calcularlo, se convierte en una indeterminación del tipo $\infty/\infty$
\[
  \lim_{x\to 0} x\ln x = \lim_{x\to 0} \frac{\ln x}{1/x} = \frac{\infty}{\infty}
\]
y se aplica la regla de L'Hopital
\[
  \lim_{x\to 0} x\ln x
  = \lim_{x\to 0} \frac{1/x}{-1/x^2}
  = - \lim_{x\to 0} \frac{1/\cancelout{x}}{1/x^{\cancelout{\scriptstyle 2}}}
  = -\lim_{x\to 0} x = 0
\]

\subsubsection{Entropía máxima}
Antes de terminar esta sección vamos a analizar cuándo es máxima la entropía.
Razonaremos en base a la máquina que producía un tres o un siete cada vez
que se pulsaba el botón. Supondremos que la probabilidad de producir un tres
es $p_1=1$ y que es nula la de que salga un siete, $p_2=0$; en esta situación
nuestro grado de desconocimiento es cero, sabemos lo que va a salir con
total seguridad. Por tanto la entropía es nula
\[
  S = -p_1\ln p_1 - p_2\ln p_2 = -1\ln 1 - 0\ln 0 = 0
\]
y como buscamos la máxima, tendremos que cambiar las probabilidades.

Bajamos, por tanto,  un poco la probabilidad de que salga el tres y
aumentamos como consecuencia la de obtención de un siete. Por ejemplo,
$P_1=0,75$ y $p_2=0,25$
\[
  S = -p_1\ln P_1 - p_2\ln p_2 = -0,75\ln 0,75 - 0,25\ln 0,25
  \approx 0,5623\cdots
\]
y vemos que ha aumentado la entropía, por lo que seguimos disminuyendo $p_1$.

Supongamos que llegamos a $p_1 = p_2 = 0,5$
\[
  S = -p_1\ln p_1 - p_2\ln p_2 = -0,5\ln 0,5 - 0,5\ln 0,5
  \approx 0,6931\cdots
\]
la entropía es mayor.

Sigamos disminuyendo, ahora muy poco la probabilidad de que salga un tres,
$p_1=0,49$
\[
  S = -p_1\ln p_1 - p_2\ln p_2 = -0,49\ln 0,49 - 0,51\ln 0,51
  \approx 0,6929\cdots  
\]
Comprobamos que ha disminuido. Tenemos, por tanto la intuición matemática
de que la entropía será máxima cuando $p_1=p_2 = 1/2$.

A continuación representamos el valor de la entropía, $S$, en función de la
probabilidad de que salga un tres, $p_1$. Para ello, ponemos la entropía
en función de $p_1$
\[
  S = -p_1\ln p_1 - p_2\ln p_2
\]
Como $p_1 + p_2 = 1$, resulta que $p_2 = 1 - p_1$
\begin{align*}
  S
  &=
    -p_1\ln p_1 - (1-p_1)\ln (1-p_1)
    = -p_1\ln p_1 - \ln (1-p_1) +  p_1\ln (1-p_1)\\
  &=
    p_1\left[\ln(1-x) - \ln x\right] - \ln (1-x)
    = p_1 \ln\frac{1-p_1}{p_1} - \ln (1-p_1)
\end{align*}

Al final queda
\[
  S = \ln\frac{\left(\frac{1-p_1}{p_1}\right)^{p_1}}{1-p_1}
\]

En la gráfica~\ref{fig:entr-p1-S} se observa que la entropía máxima se
obtiene cuando
\[
  p_1 = p_2 = 0,5
\]
\begin{figure}[ht]
\centering
\begin{tikzpicture}[scale=1.00]
\begin{axis}[
	axis background/.style={fill=yellow!18},
	grid style={line width=.2pt,draw=gray!20},
	major grid style={line width=.3pt,draw=gray!50},
	grid=both,
	%enlargelimits={abs=0.5},
	%enlarge x limits=0.0,
	restrict x to domain=0:1,
	%title= CONVERSIÓN EN EL EQUILIBRIO Y TEMPERATURA,
	ylabel style={overlay},
	yticklabel style={overlay},
	%xlabel={$X_{\ce{H2}}$},
	%ylabel={Concentración (\si{\molar})},
	xlabel={Probabilidad de obtener un tres $(p_1)$},
	ylabel={Entropía $(S)$}, %legend style={at={(0.5,0.97)},
	%   anchor=north, legend columns=-1},
	domain=0:1,
	minor y tick num = 3,
	minor x tick num = 4,
	xmin=0,
	xmax=1,
	ymin=0,
	ymax=1,
	%xtick={}
	%ytick={0,20,40,60,80,100,120},
	%grid style={line width=.2pt,draw=gray!20},
	%major grid style={line width=.3pt,draw=gray!50},
	%grid=both
]
\addplot[%
smooth,domain=0:1,color=black,mark=none,samples=500,line width=1.2pt
%] {-x*ln(x)-(1-x)*ln(1-x)};
%] {x*ln((1-x)/x)-ln(1-x)};
% {ln(((1-x)/x)^x)-ln(1-x)};
] {ln((((1-x)/x)^x)/(1-x))};
\end{axis}
\end{tikzpicture}
\caption{Representación gráfica de la entropía de la máquina que produce
  treses y sietes, en función de la probabilidad de obtener un tres.
  Se observa que es máxima cuando $p_1=0,5$.}
\label{fig:entr-p1-S}
\end{figure}


Generalizando a $N$ sucesos probables, la entropía sería máxima cuando
\[
  p_1=p_2=\cdots=p_N = 1/N
\]
\[
  S_{\text{máx}}
  = -\sum_{i} \frac{1}{N} \ln\frac{1}{N}
  = -\frac{1}{N}\,(-\ln N) \sum_i 1
  =  \cancelout{N} \frac{1}{\cancelout{N}}\,\ln N
  = \ln N
\]

Este resultado tiene cierta lógica, porque cuando todos los sucesos fueran
equiprobables, nuestra ignorancia sería máxima. Y si uno de los sucesos
tuviera algo más de probabilidad de ocurrir que otro, entonces tendríamos
mayor grado de conocimiento de lo que se obtendría, y la entropía disminuiría.

El problema de esta conclusión radica en que sólo es válida cuando los sucesos
que se consideren sean disjuntos (u ortogonales). Más adelante se analizará
de una forma más completa.

\section{Traza de una matriz}
La traza de una matriz es la suma de sus elementos diagonales
\[
  \Tr\mmm{A} = \sum_n A_{nn}
\]

\subsection{Propiedades importantes}
Podemos destacar dos propiedades de la traza de matrices:
\begin{enumerate}
\item La traza de la suma de dos matrices es la suma de la traza de
  cada una de ellas
  \begin{equation}\label{eq:entr-traza-de-suma}
    \Tr(\mmm{A} + \mmm{B})
    = \Tr\mmm{A} + \Tr\mmm{B}
  \end{equation}

\item La traza de un escalar por una matriz es el escalar por la traza
  de la matriz
  \begin{equation}\label{eq:entr-traza-escalar-matriz}
    \Tr{(\lambda\mmm{A})}
    = \lambda \Tr\mmm{A}
  \end{equation}
  
\end{enumerate}

\subsection{Traza del producto de dos matrices}
El producto de dos matrices $\mmm{D} = \mmm{A}$ y $\mmm{B}$ se podría
escribir en forma compacta como
\[
  D_{ij} = (AB)_{ij} = \sum_k A_{ik} B_{kj}
\]

Los elementos diagonales de esta matriz serían
\[
  D_{nn} = (AB)_{nn}
\]


La forma compacta de la traza de una matriz $\mmm{D}$ sería la suma
de sus elementos diagonales
\[
  \Tr{\mmm{D}} = \sum_n D_{nn}
\]

La traza del producto de dos matrices sería
\[
  \Tr{(\mmm{A}\mmm{B})} = \sum_n (AB)_{nn} = \sum_n \sum_k A_{nk} B_{kn}
  = \sum_{n,k} A_{nk} B_{kn}
\]

\section{Matriz densidad}
Para estudiar la \emph{matriz densidad} en mecánica cuántica, consideraremos
estados cuánticos con dos grados de libertad, que podremos generalizar
fácilmente a cualquier número $N$ de grados de libertad. No trataremos aquí
con estados con infinitos grados de libertad, que complicarían los cálculos
pero no añadirían nada conceptualmente.

La matriz densidad se utiliza cuando hay \emph{estados mezcla}, que se
estudiarán más adelante.

\subsection{Producto exterior y proyectores}
Sabemos que el producto interno o escalar de dos vectores produce un número
real
\[
  \ket{\phi}\cdot\ket{\psi}
  = \braket{\phi|\psi} \in \symbb{R}
\]

Ahora definimos un producto exterior de dos vectores como
\[
  \ket{\phi}\otimes\ket{\psi}
  = \ket{\phi}\bra{\psi}
\]
que se representa mediante una matriz $N\times N$ en el caso de vectores de un
espacio vectorial de dimensión finita $N$.

Normalmente el producto externo se puede aplicar a un vector columna $\ket{A}$
por la derecha
\[
  \ket{\phi}\bra{\psi} \ket{A}
  = \ket{\phi}\braket{\psi|A}
\]
o bien a un vector fila por la izquierda
\[
  \bra{A} \ket{\phi}\bra{\psi}
  = \braket{A|\phi}\bra{\psi}
\]

El producto externo de un vector de módulo unidad $\ket{\psi}$ por sí mismo se
denomina \emph{proyector} $\ket{\psi}\bra{\psi}$, cuyo resultado al aplicarlo
a un vector $\ket{A}$ es un vector proporcional a $\ket{\psi}$
\[
  \ket{\psi}\bra{\psi} \ket{A}
  = \ket{\psi} \braket{\psi|A}
\]
esto es, proyecta un vector $\ket{A}$ en la dirección definida por
$\ket{\psi}$.

El proyector tiene ciertas propiedades que escribimos sin demostrar:
\begin{itemize}
\item El proyector $\ket{\psi}\bra{\psi}$ es un operador hermítico.
\item El vector $\ket{\psi}$ es un autovector del proyector, con autovalor
  igual a uno
  \[
    \ket{\psi}\bra{\psi} \ket{\psi} = 1 \ket{\psi}
  \]
\item Cualquier vector ortogonal a $\psi$ es autovector del proyector con
  autovalor cero.

  Si $\braket{\psi|\phi} = 0$, entonces
  \[
    \ket{\psi}\bra{\psi} \ket{\phi} = 0 \ket{\phi}
  \]

\item El cuadrado del operador proyector es igual al operador proyector
  \[
    (\ket{\psi}\bra{\psi})^2 = \ket{\psi}\bra{\psi}
  \]

\item La traza de un operador proyector es la suma de sus autovalores
  \[
    \Tr(\ket{\psi}\bra{\psi}) = 1
  \]

\item La suma de los operadores proyección de todos los vectores
  de una base ortonormal es la matriz unidad
  \[
    \sum_{i} \ket{i}\bra{i} = \mmm{I} 
  \]

\item El valor esperado del operador $\mmm{F}$ en el estado $\ket{\psi}$
  es la traza del proyector del vector de estado por el operador $\mmm{F}$
  \begin{equation}\label{eq:entr-valoresperado-como-traza}
    \braket{\mmm{F}} = \braket{\psi|\mmm{F}|\psi}
    = \Tr(\ket{\psi}\bra{\psi} \mmm{F})
  \end{equation}

  Lo demostramos a continuación, recordando que la traza de una matriz es
  la suma de sus elementos diagonales
  $\Tr\mmm{F} = \sum_i \mmm{F}_{ii} = \sum_i \braket{i|\mmm{F}|i}$
  \begin{align*}
    \Tr(\ket{\psi}\bra{\psi} \mmm{F})
    &=
      \sum_i \braket{i|\psi} \braket{\psi|\mmm{F}|i}
      = \sum_i \braket{\psi|\mmm{F}|i} \braket{i|\psi}
      = \sum_i \braket{\psi|\mmm{F}} \ket{i}\bra{i} \ket{\psi}\\
    &=
      \braket{\psi|\mmm{F}} \left(\sum_i\ket{i}\bra{i}\right) \ket{\psi}
      = \braket{\psi|\mmm{F}} \mmm{I} \ket{\psi}
      = \braket{\psi|\mmm{F}|\psi}
      = \braket{\mmm{F}}
  \end{align*}
  
\end{itemize}


\subsection{Definición de matriz densidad}
El estado de un sistema cuántico queda completamente especificado mediante un
vector de estado $\ket{\psi}$, que contiene toda la información que podemos
obtener de él.
Una forma equivalente de especificar el estado de un sistema cuántico
sería mediante la matriz densidad $\mmmg{\rho}$, que se define como
el proyector
\begin{equation}\label{eq:entr-matriz-densidad}
  \mmmg{\rho} = \ket{\psi}\bra{\psi}
\end{equation}

Se puede seguir la evolución dinámica de ambos $\ket{\psi}$ y $\mmmg{\rho}$,
es decir, su evolución con el tiempo.

Una consecuencia de la definición es que la matriz densidad no depende de la
fase global. Supongamos que el vector de estado se escribe como
\[
  \ket{\psi'} = \ket{e^{i\phi}\psi} = e^{i\phi}\ket{\psi}
\]

La matriz densidad no tiene en cuenta la fase global
\[
  \mmmg{\rho}
  = \ket{\psi'}\bra{\psi'}
  = \ket{e^{i\phi}\psi}\bra{e^{i\phi}\psi}
  = e^{i\phi} e^{-i\phi} \ket{\psi}\bra{\psi}
  = \ket{\psi}\bra{\psi}
\]

\subsection{Valor esperado de un observable mecano-cuántico}
Imaginemos que tenemos un sistema cuántico finito\footnotemark{} en un estado
$\ket{\psi}$
\footnotetext{Podríamos considerar también sistemas con infinitas dimensiones,
  en cuyo caso habría que escribir la combinación lineal como una integral
  del tipo $\ket{\psi} = \int dx\,\psi(x) \ket{x}$, pero como no aportaría
  profundidad y sólo se complicaría la notación matemática, nos limitaremos
  a sistemas de dimensiones finitas.}
\[
  \ket{\psi} = a_0 \ket{0} + a_1 \ket{1}
\]
donde hemos representado los vectores de estado que forman la base, $\ket{0}$
y $\ket{1}$, como si fueran \emph{qbits}.

Un \emph{observable} se representa mediante un operador hermítico, que en
el caso de sistemas cuánticos de un número finito de grados de libertad,
se materializan como matrices cuadradas hermíticas
\[
  \mmm{F}^\dagger = \mmm{F}
\]

Hay un teorema que dice que toda matriz hermítica es diagonalizable,
sus valores propios $F_i$ son reales y sus vectores propios
asociados $\ket{F_i}$ forman una base ortonormal.

Además, el \emph{postulado de la medida} establece que al medir $\mmm{F}$,
los valores que se podrán obtener serán los valores propios, con unas ciertas
probabilidades. Y después de obtener uno de ellos $F_i$, el sistema
quedará definido por el estado $\ket{F_i}$ asociado. Las probabilidades de
obtener cada uno de los valores propios en la medida sería
\[
  p_i = |\braket{\psi|F_i}|^2
\]

Como este observable es una matriz, en dos dimensiones sería
\[
  \mmm{F}
  = \begin{pNiceMatrix}
    F_{00} & F_{01}\\
    F_{10} & F_{11}
  \end{pNiceMatrix}
\]

El valor esperado del observable en un estado cuántico $\ket{\psi}$ es
\[
  \braket{\mmm{F}} = \braket{\psi|\mmm{F}|\psi}
\]
que se interpreta como el valor promedio de todos los posibles resultados de
la medida $F_i$, teniendo en cuenta sus probabilidades $p_i$.

Desarrollamos la expresión anterior
\begin{align*}
  \braket{\mmm{F}}
  &=
    \braket{\psi|\mmm{F}|\psi}
    = \begin{pNiceMatrix}
      a_0^* & a_1^*
      \end{pNiceMatrix}
    \begin{pNiceMatrix}
      F_{00} & F_{01}\\
      F_{10} & F_{11}
    \end{pNiceMatrix}
    \begin{pNiceMatrix}
      a_0\\
      a_1
    \end{pNiceMatrix}
    = \begin{pNiceMatrix}
      a_0^* & a_1^*
      \end{pNiceMatrix}
    \begin{pNiceMatrix}
      F_{00} a_0 + F_{01} a_1\\
      F_{10} a_0 + F_{11} a_1
    \end{pNiceMatrix}\\
  &=
    F_{00} a_0 a_0^* + F_{01} a_1 a_0^* + F_{10} a_0 a_1^* + F_{11} a_1 a_1^*\\
  &=
    a_0 a_0^* F_{00} + a_1 a_0^* F_{01} + a_0 a_1^* F_{10} + a_1 a_1^* F_{11}
\end{align*}

Si abreviamos con $\rho_{ij}\equiv a_i a_j^*$, el valor esperado puede
escribirse como
\[
  \braket{\mmm{F}}
  =
  \rho_{00} F_{00} + \rho_{10} F_{01} + \rho_{01} F_{10} + \rho_{11} F_{11}
  = \sum_{n,k} \rho_{nk} F_{kn}
  = \Tr{(\mmmg{\rho}\mmm{F})}
\]

Nótese que el primer factor del producto de matrices depende de las $a_i$,
\begin{equation}\label{eq:entr-elementos-matriz-densidad}
  \rho_{ij} = a_i a_j^*
\end{equation}
que, a su vez, dependen únicamente del estado cuántico del sistema,
$\ket{\psi}$, y que equivale a los elementos de la matriz densidad.
Por otro lado, el segundo factor es la matriz del observable $\mmmg{F}$
que queremos medir, que es independiente del estado.

Por tanto, el valor esperado del observable $\mmm{F}$ es la traza de un
producto de matrices
\begin{equation}\label{eq:entr-valor-esperado-traza}
  \braket{\mmm{F}}
  = \braket{\psi|\mmm{F}|\psi}
  = \Tr{(\mmmg{\rho}\mmm{F})}
\end{equation}
Esto también se puede deducir de la
propiedad~\eqref{eq:entr-valoresperado-como-traza} de los proyectores.

\section{Entropía}
En esta sección utilizaremos una forma más general de calcular la entropía,
la utilizaremos para calcular la en estados puros y mezcla\footnotemark{}
\footnotetext{En el desarrollo de esta sección se pondrán ejemplos de
  estados puros y estados mezcla.}
\begin{equation}\label{eq:entr-def-entropia-general}
  S = -\Tr(\mmmg{\rho}\ln\mmmg{\rho})
\end{equation}
y aprenderemos cuándo equivale a la definición
básica~\eqref{eq:entr-def-entropia-basica}
\[
  S = -\sum_{i} p_i \ln p_i
\]

Empezamos haciendo una serie de consideraciones sobre las implicaciones
acerca del uso de la expresión~\eqref{eq:entr-def-entropia-general}:
\begin{itemize}
\item Hay que destacar que a una misma matriz densidad se puede llegar
  desde infinitos estados mezcla diferentes
  \[
    \mmmg{\rho} = \sum_i w_i \ket{\psi_i}\bra{\psi_i}
  \]
  donde $w_i$ es el peso con que participa cada estado puro
  $\ket{\psi_i}\bra{\psi_i}$, en el estado mezcla.
  
  Pero de esos infinitos estados mezcla sólo hay uno, que escribiremos
  como
  \begin{equation}\label{eq:entr-estados-ortogonales}
    \mmmg{\rho} = \sum_i \lambda_i \ket{u_i}\bra{u_i}
  \end{equation}
  en el que se puede utilizar la definición básica
  ~\eqref{eq:entr-def-entropia-basica}
  \[
    S = -\sum_i\lambda_i\ln\lambda_i
  \]
  y es aquél en el que los estados puros $\ket{u_i}$ que se mezclan
  son \emph{ortogonales}.
  
  Además, cuado los que se mezclan son ortogonales,
  la expresión~\eqref{eq:entr-def-entropia-basica} se minimiza, esto es
  \[
    S < -\sum_i w_i\ln w_i
  \]

  Para obtener la mezcla de estados
  ortogonales~\eqref{eq:entr-estados-ortogonales} a partir de cualquier
  matriz densidad $\mmmg{\rho}$, hay que escribir esta última en su base
  propia, es decir, hay que diagonalizarla.
  Por tanto debemos hallar los valores propios $\lambda_i$ de la matriz.
  Esto se debe a que
  \emph{los vectores propios} $\ket{u_i}$ \emph{son ortogonales} y se
  corresponden con los estados que se mezclan.

\item Lo anterior se puede explicar matemáticamente, en parte, debido a la
  necesidad de calcular el logaritmo de la matriz densidad. El logaritmo de
  una matriz $\mmmg{\rho}$ se puede desarrollar en serie de potencias de
  $\mmmg{\rho} - \mmm{I}$, siempre que la serie sea convergente
  \begin{align*}
    \ln\mmmg{\rho}
    &=
      \mmmg{I}
      + \frac{\mmmg{\rho} - \mmmg{I}}{1!}
      + \frac{(\mmmg{\rho} - \mmmg{I})^2}{2!}
      + \frac{(\mmmg{\rho} - \mmmg{I})^3}{3!}
      + \cdots
  \end{align*}
  
  El logaritmo de una matriz cuadrada se puede hallar en la práctica utilizando
  la propiedad de que el logaritmo de una matriz diagonal es la matriz de los
  logaritmos de sus elementos diagonales.
  Por ejemplo, diagonalizando $\mmmg{\rho}$ en dos dimensiones
  \[
    \mmmg{\rho}
    = \begin{pNiceMatrix}
      \lambda_0 & 0\\
      0 & \lambda_1
    \end{pNiceMatrix}
    \hspace{.5em}\longrightarrow\hspace{.5em}
    \ln\mmmg{\rho}
    = \begin{pNiceMatrix}
      \ln\lambda_0 & 0\\
      0 & \ln\lambda_1
    \end{pNiceMatrix}
  \]
  y como resulta que la matriz densidad es hermítica, tenemos la seguridad de
  que se puede diagonalizar.
  
  Entonces, y sólo entonces
  \[
    \mmmg{\rho}\ln\mmmg{\rho}
    = \begin{pNiceMatrix}
      \lambda_0 & 0\\
      0 & \lambda_1
    \end{pNiceMatrix}
    \begin{pNiceMatrix}
      \ln\lambda_0 & 0\\
      0 & \ln\lambda_1
    \end{pNiceMatrix}
    = \begin{pNiceMatrix}
      \lambda_0\ln\lambda_0 & 0\\
      0 & \lambda_1\ln\lambda_1
      \end{pNiceMatrix}
    \]
    la entropía se podría expresar mediante la fórmula
    básica~\eqref{eq:entr-def-entropia-basica}
    \[
      S = -\Tr(\mmmg{\rho}\ln\mmmg{\rho})
      = -\lambda_0\ln\lambda_0 - \lambda_1\ln\lambda_1
    \]
    donde las $\lambda_i$ son los pesos de los estados mezcla ortogonales
    $\ket{u_0}$ y $\ket{u_1}$.
\end{itemize}


\subsection{Estados puros}
Se dice que un sistema se encuentra en un \emph{estado puro} cuando se
conoce perfectamente. Esto tiene distintas consecuencias según el sistema
sea clásico o cuántico.

La entropía de estos sistemas vale cero porque nuestro grado de desconocimiento
es nulo
\[
  S = 0
\]

\subsubsection{Mecánica clásica}
En mecánica clásica, un estado puro se representa mediante un punto en el
espacio de las fases. El estado, en un instante $t$, de una partícula queda
completamente determinado mediante sus coordenadas $\vvv{r}$ y su momento
lineal $\vvv{p}$.
Conocer el estado de un sistema clásico en un tiempo $t$, implica conocerlo
en cualquier instante posterior siempre que tengamos en cuenta las fuerzas que
actúan sobre el sistema y las ligaduras que constriñen su movimiento.

Los estados posibles que resultan de lanzar una moneda\footnotemark{} son dos
puntos en el espacio de fases: \emph{cara} y \emph{cruz}, que podríamos
describir por $0$ y $1$, respectivamente.
\footnotetext{Descartando la posibilidad de que quede de canto.}
Estos dos estados son mútuamente excluyentes en el sentido de que si una
moneda se encuentra, por ejemplo, en el estado \emph{cara}, es imposible
que la veamos como \emph{cruz}, y viceversa. Diremos que estos estados
son ortogonales.

Si una persona lanza una moneda y nos comunica el resultado, por ejemplo,
\emph{cara} o $0$, la moneda se encontrará en un estado puro que no cambiará
mientras no se interactúe con la moneda para cambiarlo.
Así, cada vez que midamos su estado, encontraremos que está de cara o $0$
siempre, con una probabilidad $p=1$.

La entropía se puede calcular de las dos formas:
\begin{itemize}
\item En estas condiciones, la entropía de la moneda será cero
\[
  S = -p_0\ln p_0 - p_1\ln p_1 = -1 \ln 1 - 0\ln 0 = 0
\]
lo que significa que nuestro grado de desconocimiento de su estado es cero,
lo que implica que está en un estado clásico puro.

\item Por otro lado, como al lanzar la moneda son posibles los valores,
cara $(1, 0)$ y cruz $(0,1)$.
Si sale cara, por ejemplo, el vector de estado de la moneda será
\[
  \ket{0} =
  \begin{pNiceMatrix}
    1\\
    0
  \end{pNiceMatrix}
\]

La matriz de densidad es
\[
  \mmmg{\rho}
  = \ket{0}\bra{0}
  = \begin{pNiceMatrix}
    1\\
    0
  \end{pNiceMatrix}
  \begin{pNiceMatrix}
    1 & 0
  \end{pNiceMatrix}
  = \begin{pNiceMatrix}
    1 & 0\\
    0 & 0
  \end{pNiceMatrix}
\]

Esta matriz es diagonal, calculamos su logaritmo directamente y lo
multiplicamos por la matriz.
Pero no intentamos resolver el logaritmo de cada elemento hasta que no
multipliquemos las matrices
\[
  \mmmg{\rho}\ln\mmmg{\rho}
  =
  \begin{pNiceMatrix}
    1 & 0\\
    0 & 0
  \end{pNiceMatrix}
  \begin{pNiceMatrix}
    \ln 1 & 0\\
    0 & \ln 0
  \end{pNiceMatrix}
  =
  \begin{pNiceMatrix}
    1\ln 1 & 0\\
    0 & 0\ln 0
  \end{pNiceMatrix}
\]

La entropía será la traza de la matriz
\[
  S
  = -\Tr(\mmmg{\rho}\ln\mmmg{\rho})
  = -1\ln 1 - 0\ln 0
  = 0
\]
\end{itemize}


\subsubsection{Mecánica cuántica}
Consideraremos, por sencillez, un sistema cuántico de dos dimensiones, como
puede ser el spin 1/2. Si se mide el spin de una partícula en una
cierta dirección, que llamaremos eje $z$, el resultado será un spin
\emph{hacia arriba} o \emph{hacia abajo}, de acuerdo con el postulado
de la medida. En el primer caso, la partícula quedará en el estado que
representaremos como $\ket{0}=\ket{z_+}$, y en el segundo en
$\ket{1}=\ket{z_-}$, donde hemos utilizado esta vez el $0$ y el $1$ por
analogía con los \emph{qbits}, y porque además, al representarlos mediante
números se podrían utilizar como índices de sumatorios en fórmulas.

Según este postulado, los posibles valores que se pueden obtener (en este
ejemplo dos) se corresponden con estados que son ortogonales entre sí,
esto es, su producto interno es nulo
\[
  \braket{0|1} = \braket{1|0} = 0
\]
Esto se puede interpretar como que los dos estados son
\emph{mútuamente excluyentes},
es decir, si una partícula estuviera en un instante en uno de los dos estados,
por ejemplo $\ket{0}$, sería imposible encontrarla con spin hacia abajo en el
mismo instante\footnotemark{}.
\footnotetext{Si su estado cambiara con el tiempo, sí podría ser posible
  encontrarla más tarde con el spin opuesto.}

La entropía de un sistema cuántico se define en función de la matriz densidad
\[
  S = -\Tr(\mmmg{\rho}\ln\mmmg{\rho})
\]

Cuando consideramos el spin 1/2, la matriz densidad será una matriz
cuadrada $2\times 2$. Un \emph{estado puro} genérico de spin 1/2 de un sistema,
tendría la forma
\[
  \ket{\psi} = a_0\ket{0} + a_1\ket{1}
  = a_0
  \begin{pNiceMatrix}
    1\\
    0
  \end{pNiceMatrix}
  + a_1
  \begin{pNiceMatrix}
    0\\
    1
  \end{pNiceMatrix}
  = \begin{pNiceMatrix}
    a_0\\
    a_1
  \end{pNiceMatrix}
\]

Las amplitudes de probabilidad $a_0$ y $a_1$ nos permiten calcular las
probabilidades de obtener los dos valores posibles del spin en la dirección del
eje $z$, asociados a los estados $\ket{0}$ y $\ket{1}$
\begin{align*}
  p_0 &= a_0 a_0^* = |a_0|^2\\
  p_1 &= a_1 a_1^* = |a_1|^2
\end{align*}

La matriz densidad sería
\begin{equation}\label{eq:entr-matriz-densidad-spin1/2}
  \mmmg{\rho}
  =
  \ket{\psi}\bra{\psi}
  = \begin{pNiceMatrix}
    a_0\\
    a_1
  \end{pNiceMatrix}
  \begin{pNiceMatrix}
    a_0^* & a_1^*
  \end{pNiceMatrix}
  = \begin{pNiceMatrix}
    a_0 a_0^* & a_0 a_1^*\\
    a_1 a_0^* & a_1 a_1^*
  \end{pNiceMatrix}
  = \begin{pNiceMatrix}
    p_0 & a_0 a_1^*\\
    a_1 a_0^* & p_1
    \end{pNiceMatrix}
\end{equation}

Para calcular el logaritmo de la matriz densidad, aprovechamos que
la matriz densidad es hermítica, por lo que sus valores propios serán números
reales no negativos debido a que los valores diagonales son probabilidades.
Una vez diagonalizada, el logaritmo de la matriz sería la matriz de los
logaritmos de los elementos diagonales\footnotemark{}.
\footnotetext{Posteriormente habría que reescribir esta matriz diagonal con
  logaritmos en la base original y multiplicarla por la matriz densidad
  original, pero no necesitaremos hacerlo porque para calcular la entropía
  necesitamos la traza, que sería igual en las dos matrices.}

Supongamos que la hemos diagonalizado y la representamos en la base de
los vectores propios
\[
  \mmmg{\rho}'
  = \begin{pNiceMatrix}
    \lambda_0 & 0\\
    0 & \lambda_1
  \end{pNiceMatrix}
\]

Entonces el logaritmo de la matriz sería
\[
  \ln\mmmg{\rho}'
  = \begin{pNiceMatrix}
    \ln\lambda_0 & 0\\
    0 & \ln\lambda_1
  \end{pNiceMatrix}
\]

Si alguna de las probabilidades $\lambda_i$ fuera cero, su logaritmo no
estaría definido. Pero como lo que necesitamos es $\mmmg{\rho}'\ln\mmmg{\rho}'$,
entonces podremos suponer, como se vió en la
sección~\ref{subsect:entr-entropia-basica} que
\begin{equation}\label{eq:entr-0ln0}
  "0\ln 0" = \lim_{p_i\to 0} p_i\ln p_i = 0
\end{equation}

Por tanto escribimos $\mmmg{\rho}'\ln\mmmg{\rho}'$, obviando de momento
el que algún $p_i'$ fuera cero
\[
  \mmmg{\rho}'\ln\mmmg{\rho}'
  = \begin{pNiceMatrix}
    \lambda_0 & 0\\
    0 & \lambda_1  
  \end{pNiceMatrix}
  \begin{pNiceMatrix}
    \ln \lambda_0 & 0\\
    0 & \ln \lambda_1
  \end{pNiceMatrix}
  = \begin{pNiceMatrix}
    \lambda_0\ln \lambda_0 & 0\\
    0 & \lambda_1\ln \lambda_1
  \end{pNiceMatrix}
\]

El opuesto de la traza de esta matriz sería
\[
  -\Tr(\mmmg{\rho}'\ln\mmmg{\rho}')
  = -\lambda_0\ln \lambda_0 - \lambda_1\ln \lambda_1
\]

Aprovechando que al diagonalizar una matriz se conserva la traza,
la entropía del sistema sería, finalmente
\begin{equation}\label{eq:entr-entropia-cuantica-p0'-p1'}
  S = -\Tr(\mmmg{\rho}\ln\mmmg{\rho})
  = -\Tr(\mmmg{\rho}'\ln\mmmg{\rho}')
  = -\lambda_0\ln \lambda_0 - \lambda_1\ln \lambda_1
\end{equation}

Para nuestro sistema de spin 1/2, diagonalizamos la matriz
densidad~\eqref{eq:entr-matriz-densidad-spin1/2}, planteando su ecuación
de autovalores y autovectores
\[
  \mmmg{\rho} \vvv{u} = \lambda \vvv{u}
\]
\[
  \mmmg{\rho} \vvv{u} - \lambda \vvv{u} = \vvv{0}
\]
\[
  (\mmmg{\rho} - \lambda\mmm{I}) \vvv{u} = \vvv{0}
\]
\begin{align*}
  \begin{pNiceMatrix}
    p_0 - \lambda & a_0 a_1^*\\
    a_1 a_0^* & p_1 - \lambda
  \end{pNiceMatrix}
  \begin{pNiceMatrix}
    u\\
    v
  \end{pNiceMatrix}
  =
  \begin{pNiceMatrix}
    0\\
    0
  \end{pNiceMatrix}
\end{align*}

Para obtener los valores propios, el determinante de la matriz debe anularse,
ya que el sistema de ecuaciones tiene que ser compatible indeterminado
\[
  \begin{vNiceMatrix}
    p_0 - \lambda & a_0 a_1^*\\
    a_1 a_0^* & p_1 - \lambda
  \end{vNiceMatrix}
  = 0
\]

Desarrollando el determinante, obtenemos la ecuación característica, y
teniendo en cuenta que $p_0 + p_1 = 1$
\[
  \left(p_0-\lambda\right) \left(p_1-\lambda\right) -
  a_0 a_1^* a_1 a_0^*
  = 0
\]
\[
  \cancelout{p_0 p_1} - (p_0 + p_1)\lambda + \lambda^2 - \cancelout{p_0 p_1} = 0
\]
\[
  \lambda^2 - \lambda = 0
\]
\[
  \lambda(\lambda - 1) = 0
\]
de donde $\lambda_0 = 1$ y $\lambda_1 = 0$.
La matriz densidad diagonalizada será
\[
  \mmmg{\rho}
  = \begin{pNiceMatrix}
    1 & 0\\
    0 & 0
  \end{pNiceMatrix}
\]

Y según~\eqref{eq:entr-entropia-cuantica-p0'-p1'}, junto
con~\eqref{eq:entr-0ln0}, la entropía del sistema puro $\ket{\psi}$ será cero
\[
  S = -\lambda_0\ln \lambda_0 - \lambda_1\ln \lambda_1
  = -1\ln 1 - 0\ln 0
  = 0
\]
lo que implica que se el estado está completamente determinado por el
vector de estado $\ket{\psi}$. Nótese que el hecho de que no sepamos
con seguridad el resultado de una medida del spin en la dirección $z$,
\emph{es inherente a la mecánica cuántica} y no se debe a ningún
desconocimiento achacable a nosotros.


\subsection{Estados mezcla}
Los \emph{estados mezcla} surgen debido a un \emph{conocimiento incompleto}
del estado de un sistema. Suelen aparecer en mecánica estadística, tanto
clásica como cuántica, porque debido al gran número de partículas implicadas
resulta del todo imposible conocer el estado de todas y cada una de ellas.
También se tienen que tener en cuenta los estados mezcla cuando un sistema
cuántico conocido interacciona con el ambiente de forma incontrolable y por
decoherencia nos hace perder información.

En mecánica cuántica hay solapamientos, y por solapamiento nos referimos a
que no hay un experimento que pueda distinguir entre dos estados no
ortogonales.
Una medida en un estado mezcla nos permite distinguir sólo entre estados
ortogonales; recuérdese que el resultado de la medida de un observable da como
resultado uno de los valores propios de éste.
Pero estos valores propios están asociados a vectores propios ortogonales entre
sí, que se corresponden a estados ortogonales\footnotemark{}.
\footnotetext{Hay técnicas que intentan minimizar el error que se produce al
  intentar distinguir entre estados solapados
  (\emph{quantum measure discrimination}).}

\begin{quote}
  ``Para caracterizar estados mezcla debemos utilizar la matriz densidad
  $\mmmg{\rho}$, pues ya no se puede utilizar el rayo en el espacio de Hilbert
  que representa a un estado puro $\ket{\psi}$''
\end{quote}

Hay infinitas mezclas de estados que producen la misma matriz densidad
(mismo estado), y así, todas ellas son \emph{físicamente indistiguibles}.
Además ocurre que ningún experimento puede discriminar entre estados no
ortogonales.
De éstas, sólo hay una mezcla que implica estados ortogonales $\ket{u_i}$
y sus probabilidades $\lambda_i$, que reproduce la matriz densidad.
Entonces, al realizar una medida en ese estado $\mmmg{\rho}$, en realidad
estamos disciminando entre estos estados ortogonales.

¿Cómo podemos encontrar los estados $\ket{u_i}$ y sus probabilidades
asociadas $\lambda_i$?
Diagonalizando la matriz, esto es, expresándola en función de la
base formada por sus vectores propios.

Además se produce la circunstancia de que el cálculo con los valores propios
es el que produce el menor valor, y éste valor es la entropía
\[
  S = -\sum_i \lambda_i\ln\lambda_i
\]
mientras que si tomáramos los pesos de cualquiera de los otros posibles
estados mezcla que reproducen $\mmmg{\rho}$, el valor sería mayor que la
entropía
\[
  -\sum_i w_i\ln w_i > S
\]

Como en esta sección nos interesa la entropía, tendremos que
calcularla mediante
\[
  S = - \Tr(\mmmg{\rho}\ln\mmmg{\rho})
\]

Cuando los estados que se mezclan son ortonormales, la fórmula anterior
es equivalente a
\[
  S = -\sum_i \lambda_i\ln \lambda_i
\]
    
\subsubsection{Construcción de la matriz densidad de un estado mezcla}
Supongamos una mezcla de dos estados que no tienen porqué ser ortogonales,
\begin{align*}
  \ket{\psi_0} &= a_0 \ket{0} + a_1 \ket{1}\\
  \ket{\psi_1} &= b_0 \ket{0} + b_1 \ket{1}
\end{align*}
donde $\ket{0}$ y $\ket{1}$ forman una base ortonormal.

El valor esperado del observable $\mmm{F}$ en cada uno de los estados sería,
según~\eqref{eq:entr-valor-esperado-traza}
\begin{align*}
  \braket{\mmm{F}}_0
  &=
    \braket{\psi_0|\mmm{F}|\psi_0} = \Tr{(\mmmg{\rho_0} \mmm{F})}\\
  \braket{\mmm{F}}_1
  &=
    \braket{\psi_1|\mmm{F}|\psi_1} = \Tr{(\mmmg{\rho_1} \mmm{F})}
\end{align*}
Como $\mmmg{\rho}$ depende del estado, le ponemos un subíndice $\mmmg{\rho_i}$
para indicar el estado $\ket{\psi_i}$ con el que se corresponde.

Si alguien nos dice que el sistema está descrito por $\ket{\psi_0}$ con
un peso\footnotemark{} $w_0$ y por $\ket{\psi_1}$ con un peso $w_1$,
donde los estados no son necesariamente ortogonales, entonces el valor
esperado de $\mmm{F}$ para el sistema en esta situación sería
\footnotetext{Le podemos llamar probabilidad, pero como no se refiere
  a un estado puro preferimos utilizar otro nombre.}
\[
  \braket{\mmm{F}}
  = w_0 \braket{\mmm{F}}_0 + w_1 \braket{\mmm{F}}_1
  = w_0 \braket{\psi_0|\mmm{F}|\psi_0} + w_1 \braket{\psi_1|\mmm{F}|\psi_1}
\]
y esto es así porque al incluir los valores esperados para cada posible estado
del sistema $\braket{\mmm{F}}_i = \braket{\psi_i|\mmm{F}|\psi_i}$ ya se han
promediado las incertidumbres cuánticas y estamos trabajando de forma
estadística normal, \emph{por encima de lo cuántico}.

Vayamos un poco más alla a ver si llegamos a alguna conclusión que implique
a la matriz densidad. Aplicamos las
propiedades~\eqref{eq:entr-traza-de-suma}
y~\eqref{eq:entr-traza-escalar-matriz}
de la traza de matrices
\begin{align*}
  \braket{\mmm{F}}
  &=
    w_0 \braket{\psi_0|\mmm{F}|\psi_0} + w_1 \braket{\psi_1|\mmm{F}|\psi_1}
    = w_0 \Tr{(\mmmg{\rho_0}\mmm{F})} + w_1 \Tr{(\mmmg{\rho_1}\mmm{F})}\\
  &=
    \Tr{(w_0 \mmmg{\rho_0}\mmm{F})} + \Tr{(w_0 \mmmg{\rho_1}\mmm{F})}
    = \Tr{\left[w_0\mmmg{\rho_0}\mmm{F} + w_1\mmmg{\rho_1}\mmm{F}\right]}\\
  &=
    \Tr{\left[(w_0\mmmg{\rho_0} + w_1\mmmg{\rho_1}) \mmm{F}\right]}
\end{align*}

Comparando el segundo miembro de la expresión anterior
con~\eqref{eq:entr-valor-esperado-traza}, vemos que
$w_0\mmmg{\rho_0} + w_1\mmmg{\rho_1}$ sería la matriz densidad de un
\emph{estado mezcla} formado por varios estados puros $\ket{\psi_0}$ y
$\ket{\psi_1}$, con pesos respectivos $w_0$ y $w_1$.

Por tanto definimos un estado mezcla $\mmmg{\rho}$ como
\begin{equation}\label{eq:entr-estado-mezcla-rho}
  \mmmg{\rho} = w_0\mmmg{\rho_0} + w_1\mmmg{\rho_1}
\end{equation}

Con estados puros no se necesita el formalismo de la matriz densidad, pero
con estados mezclas es obligatorio.

Estos estados mezcla pueden surgir en física estadística, por ejemplo, donde
existe la posibilidad de haya distintos estados de energía posibles,
como $\ket{E_0}$, $\ket{E_1}$, etc., que se pueden ocupar por las partículas
del sistema con unos ciertos pesos o probabilidades $w_0$, $w_1$, etc.,
que dependen de la temperatura. La energía total del sistema contempla esa
mezcla de estados $\ket{E_i}$ (que se llaman estados de Gibbs).
Por tanto, para describir la termodinámica y la cuántica se necesitan estados
mezcla. También se necesitan los estados mezcla en cosas tan avanzadas como
entrelazamiento en el espacio-tiempo.

\subsubsection{Experimentos clásicos}
Vamos a proponer dos experimentos clásicos:
\begin{enumerate}
\item El primer experimento se mezclan estados ortogonales.
  
    Supongamos que alguien a quien no vemos, quizás por estar tras una
    pantalla, utiliza un sistema clásico que podría ser una moneda o cualquier
    otro dispositivo. Prepara un juego que pueda producir dos resultados, que
    llamaremos cara y cruz, y que denotaremos como 0 o 1. Supongamos que se
    trata de una moneda trucada que produce cara en un cuarto de los
    lanzamientos y cruz en tres cuartos de ellos, circunstancia de la que
    nos informa de antemano.

    Entonces, lanza la moneda y produce un resultado que nosotros no podemos
    ver y por tanto tenemos un cierto grado de ignorancia, aunque cualquier
    otra persona que estuviera con la anterior podrá conocerlo.
    Este es un ejemplo de ignoracia clásica, debido a que no conocemos el
    resultado del lanzamiento de la moneda.

    Como los dos estados posibles del experimento son ortogonales, lo
    resolveremos de dos formas:
    \begin{itemize}
    \item Calculamos mediante~\eqref{eq:entr-def-entropia-basica}
      nuestro grado de desconocimiento del estado de la moneda
      \[
        S
        = -p_0\ln p_0 - p_1\ln p_1
        = -\frac{1}{4}\,\ln\frac{1}{4} - \frac{3}{4}\ln\frac{3}{4}
        = 0,5623\cdots
      \]

    \item Seguimos el camino más elaborado que utiliza la matriz densidad.

      Los dos estados que se mezclan son
      \[
        \ket{\psi_0}
        =
        \ket{0}
        =
        \begin{pNiceMatrix}
          1\\
          0
        \end{pNiceMatrix}
        \hspace{.5em}\text{y}\hspace{.5em}
        \ket{\psi_1}
        =
        \ket{1}
        =
        \begin{pNiceMatrix}
          0\\
          1
        \end{pNiceMatrix}
      \]
      donde $\ket{0}$ representa cara y $\ket{1}$ cruz.

      La matriz densidad que describe la mezcla es
      \begin{align*}
        \mmmg{\rho}
        &=
          w_0\mmmg{\rho_0} + w_1\mmmg{\rho_1}
          = w_0\ket{0}\bra{0} + w_1\ket{1}\bra{1}\\
        &=
          \frac{1}{4}
          \begin{pNiceMatrix}
            1\\
            0
          \end{pNiceMatrix}
        \begin{pNiceMatrix}
          1 & 0
        \end{pNiceMatrix}
        + \frac{3}{4}
        \begin{pNiceMatrix}
          0\\
          1
        \end{pNiceMatrix}
        \begin{pNiceMatrix}
          0 & 1
        \end{pNiceMatrix}
        =
          \begin{pNiceMatrix}
            1/4 & 0\\
            0 & 3/4
          \end{pNiceMatrix}
      \end{align*}

      La matriz densidad es diagonal, por lo que se puede calcular
      fácilmente el logaritmo de la matriz, multiplicarlo por la matriz
      densidad, hallar la traza y calcular su entropía
      \[
        S
        = -\Tr(\mmmg{\rho}\ln\mmmg{\rho})
        = -\frac{1}{4}\ln\frac{1}{4} - \frac{3}{4}\ln\frac{3}{4}
        = 0,5623\cdots
      \]
    \end{itemize}

  \item Un ejemplo de mezcla de estados no ortogonales consiste en el
    lanzamiento de un dado, el primer estado es que salga un dos
    y el otro que salga un número par. Obviamente los dos estados no
    son disjuntos, por lo que no se puede utilizar la fórmula básica.

    La probabilidad de que salga un dos es $1/6$, mientras
    la de que salga un número par es $3/6 = 1/2$.
    El estado mezcla sería algo así
    \[
      \begin{array}{lll}
        \ket{\psi_1} && w_1 = 1/6\\[4pt]
        \ket{\psi_2} && w_2 = 1/2\\
      \end{array}
    \]
    
    Los vectores normalizados implicados son
    {\small
      \[
        \ket{\psi_1}
        = \begin{pNiceMatrix}
          0\\
          1\\
          0\\
          0\\
          0\\
          0
        \end{pNiceMatrix}
        \hspace{1.5em}\text{y}\hspace{1.5em}
        \ket{\psi_2}
        = \frac{1}{\sqrt{3}}
        \begin{pNiceMatrix}
          0\\
          1\\
          0\\
          1\\
          0\\
          1
        \end{pNiceMatrix}      
      \]
    }
    
    Construimos la matriz densidad de los estados individuales
    {\small
      \begin{align*}
        \mmmg{\rho_1}
        &=
          \ket{\psi_1}\bra{\psi_1}
          = \begin{pNiceMatrix}
            0\\
            1\\
            0\\
            0\\
            0\\
            0
          \end{pNiceMatrix}
        \begin{pNiceMatrix}
          0 & 1 & 0 & 0 & 0 & 0
        \end{pNiceMatrix}
      = \begin{pNiceMatrix}
        0 & 0 & 0 & 0 & 0 & 0\\
        0 & 1 & 0 & 0 & 0 & 0\\
        0 & 0 & 0 & 0 & 0 & 0\\
        0 & 0 & 0 & 0 & 0 & 0\\
        0 & 0 & 0 & 0 & 0 & 0\\
        0 & 0 & 0 & 0 & 0 & 0
      \end{pNiceMatrix}\\
        \mmmg{\rho_2}
        &=
          \ket{\psi_2}\bra{\psi_2}
          = \frac{1}{\sqrt{3}}
          \begin{pNiceMatrix}
            0\\
            1\\
            0\\
            1\\
            0\\
            1
          \end{pNiceMatrix}
        \frac{1}{\sqrt{3}}
        \begin{pNiceMatrix}
          0 & 1 & 0 & 1 & 0 & 1
        \end{pNiceMatrix}
        = \frac{1}{3}
        \begin{pNiceMatrix}
          0 & 0 & 0 & 0 & 0 & 0\\
          0 & 1 & 0 & 1 & 0 & 1\\
          0 & 0 & 0 & 0 & 0 & 0\\
          0 & 1 & 0 & 1 & 0 & 1\\
          0 & 0 & 0 & 0 & 0 & 0\\
          0 & 1 & 0 & 1 & 0 & 1
        \end{pNiceMatrix}                      
      \end{align*}
    }
    
    La matriz densidad del estado mezcla sería
    {\small
      \begin{align*}
        \mmmg{\rho}
        &=
          w_0\mmmg{\rho_1} + w_1\mmmg{\rho_2}
          = \frac{1}{6}\mmmg{\rho_1} + \frac{1}{2}\mmmg{\rho_2}\\
        &=
          \frac{1}{6}
          \begin{pNiceMatrix}
            0 & 0 & 0 & 0 & 0 & 0\\
            0 & 1 & 0 & 0 & 0 & 0\\
            0 & 0 & 0 & 0 & 0 & 0\\
            0 & 0 & 0 & 0 & 0 & 0\\
            0 & 0 & 0 & 0 & 0 & 0\\
            0 & 0 & 0 & 0 & 0 & 0
          \end{pNiceMatrix}
        + \frac{1}{2}\cdot\frac{1}{3}
        \begin{pNiceMatrix}
          0 & 0 & 0 & 0 & 0 & 0\\
          0 & 1 & 0 & 1 & 0 & 1\\
          0 & 0 & 0 & 0 & 0 & 0\\
          0 & 1 & 0 & 1 & 0 & 1\\
          0 & 0 & 0 & 0 & 0 & 0\\
          0 & 1 & 0 & 1 & 0 & 1
        \end{pNiceMatrix}\\
        &= \frac{1}{6}
          \begin{pNiceMatrix}
            0 & 0 & 0 & 0 & 0 & 0\\
            0 & 2 & 0 & 1 & 0 & 1\\
            0 & 0 & 0 & 0 & 0 & 0\\
            0 & 1 & 0 & 1 & 0 & 1\\
            0 & 0 & 0 & 0 & 0 & 0\\
            0 & 1 & 0 & 1 & 0 & 1
          \end{pNiceMatrix}
      \end{align*}
    }
    
    Los valores propios son
    \[
      \lambda_1 = \frac{2-\sqrt{2}}{6}
      ;\hspace{.25em} \lambda_2 = \frac{2+\sqrt{2}}{6}
      ;\hspace{.25em} \lambda_3 = \lambda_4 = \lambda_5 = \lambda_6 = 0
    \]
    
    Por tanto, la entropía sería
    \begin{align*}
      S
      &=
      -\Tr(\mmmg{\rho}\ln\mmmg{\rho})
      = -\sum_{i=1}^6 \lambda_i \ln\lambda_i\\
      &= - \frac{2-\sqrt{2}}{6}\ln\frac{2-\sqrt{2}}{6}
        - \frac{2+\sqrt{2}}{6}\ln\frac{2+\sqrt{2}}{6}
        + 0 + 0 + 0 + 0\\
      &=
        0,54797\cdots
    \end{align*}

    Obsérvese que si hubiéramos realizado el cálculo incorrecto, hubiera
    dado un resultado mayor
    \[
      -w_1\ln w_1 - w_2\ln w_2
      = -\frac{1}{6} \ln\frac{1}{6} - \frac{1}{2}\ln\frac{1}{2}
      = 0,57762\cdots
    \]
    
    En la sección~\ref{sect:apcua-estadomezcla-dado} del
    apéndice~\ref{chapt:apcua-matriz-densidad-codigo} se lista el código
    para calcular los valores de este ejemplo.
\end{enumerate}



%%% Local Variables:
%%% coding: utf-8
%%% mode: latex
%%% TeX-engine: luatex
%%% TeX-master: "../cuarentena.tex"
%%% End:

