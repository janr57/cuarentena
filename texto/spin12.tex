% spin12.tex
%
% Copyright (C) 2022 José A. Navarro Ramón <janr.devel@gmail.com>
% 1) Código LuaLatex:
%    Licencia GPL-2.
% 2) Producto en pdf, postscript, etc.:
%    Licencia Creative Commons Recognition Share alike. (CC-BY-SA)

\chapter{Spin 1/2}
El spin $S$ es \emph{un momento angular} que \emph{rota} estados cuánticos
de spin alrededor de una cierta dirección espacial. Se trata de un observable
mecanocuántico que \emph{vive} en nuestro mundo $\symbb{R}^3$, con tres
componentes, $S_x$, $S_y$, $S_z$ que se representan mediante operadores
hermíticos. Son hermíticos puesto que los valores que medimos de este
observable son números reales. En el caso del momento angular de spin,
estos operadores se representan mediante matrices cuadradas $N\times N$.

En este capítulo nos centraremos en el spin 1/2 y estudiaremos la relación
entre este, las \emph{matrices de Pauli} y las rotaciones.

\section{Espinores}
 El estado de spin $S$ de un sistema cuántico se representa mediante elementos
del espacio vectorial complejo $\symbb{C}^N$, denominados espinores,
donde $N=2S+1$.

Para el spin 1/2, $N= 2\cdot 1/2 + 1 = 2$ y el estado de spin se representa
mediante vectores de $\symbb{C}^2$
\[
  \symbb{C}^2
  =
  \left\{
    \vvv{z} = \begin{pNiceMatrix}\alpha \\ \beta\end{pNiceMatrix}
    \,\big/\, \alpha, \beta \in \symbb{C}
  \right\}
\]

Estos espinores, en concreto, son \emph{objetos} con spin $1/2$.
Sus componentes, $\alpha$ y $\beta$, son las amplitudes de probabilidad
de que al medir la componente del spin en una cierta dirección, esta tenga
un sentido a favor o en contra de ella.

Los fermiones, como el electrón, son partículas de spin 1/2.
Por tanto, la probabilidad de que al medir el spin de un electrón
en una cierta dirección, por ejemplo el eje $z$, nos dé un spin
\emph{hacia arriba} sería
\[
  |\alpha|^2 = \alpha^* \alpha
\]

La probabilidad de que se encuentre \emph{hacia abajo} sería
\[
  |\beta|^2 = \beta^* \beta
\]

Como no hay más alternativas, la suma de estas probabilidades tiene que
valer la unidad
\[
  |\alpha|^2 + |\beta|^2 = 1
\]

Como consecuencia, el módulo de un espinor debe valer la unidad
\[
  |z|^2
  = z^\dagger z
  = \begin{pNiceMatrix}
    \alpha\\
    \beta
  \end{pNiceMatrix}^\dagger
  \begin{pNiceMatrix}
    \alpha\\
    \beta
  \end{pNiceMatrix}
  = \begin{pNiceMatrix}
    \alpha^* & \beta^*
  \end{pNiceMatrix}
  \begin{pNiceMatrix}
    \alpha\\
    \beta
  \end{pNiceMatrix}
  = \alpha^* \alpha + \beta^* \beta
  = |\alpha|^2 + |\beta|^2
  = 1
\]

Resumiendo:
\begin{quote}
  ``Un spin $1/2$, o espinor, es un objeto con dos componentes complejas que
  representan amplitudes de probabilidad de que el spin esté \emph{arriba} o
  \emph{abajo}.''
\end{quote}

\section{Matrices de Pauli}
Las matrices de Pauli son tres matrices cuadradas, $2\times 2$, linealmente
independientes
\[
  \mmm{\sigma_1}
  = \mmm{\sigma_x}
  = \begin{pNiceMatrix}
    0 & 1\\
    1 & 0
  \end{pNiceMatrix}
  ;\hspace{1em}
  \mmm{\sigma_2}
  = \mmm{\sigma_y}
  = \begin{pNiceMatrix}
    0 & -i\\
    i & 0
  \end{pNiceMatrix}
  ;\hspace{1em}
  \mmm{\sigma_3}
  = \mmm{\sigma_z}
  = \begin{pNiceMatrix}
    1 & 0\\
    0 & -1
  \end{pNiceMatrix}  
\]

Podemos comprobar que sus cuadrados son iguales a la matriz identidad,
$\mmm{I}$
\begin{align*}
  \mmm{\sigma_1}^2
  &=
    \begin{pNiceMatrix}
      0 & 1\\
      1 & 0
    \end{pNiceMatrix}
    \begin{pNiceMatrix}
      0 & 1\\
      1 & 0
    \end{pNiceMatrix}
    = \begin{pNiceMatrix}
      1 & 0\\
      0 & 1
    \end{pNiceMatrix}
    = \mmm{I}\\
  \mmm{\sigma_2}^2
  &=
    \begin{pNiceMatrix}
      0 & -i\\
      i & 0
    \end{pNiceMatrix}
    \begin{pNiceMatrix}
      0 & -i\\
      i & 0
    \end{pNiceMatrix}
    = \begin{pNiceMatrix}
      1 & 0\\
      0 & 1
    \end{pNiceMatrix}
    = \mmm{I}\\
  \mmm{\sigma_3}^2
  &=
    \begin{pNiceMatrix}
      1 & 0\\
      0 & -1
    \end{pNiceMatrix}
    \begin{pNiceMatrix}
      1 & 0\\
      0 & -1
    \end{pNiceMatrix}
    = \begin{pNiceMatrix}
      1 & 0\\
      0 & 1
    \end{pNiceMatrix}
    = \mmm{I}
\end{align*}

Podemos resumir lo anterior en
\begin{equation}\label{eq:spin12-sigmai-sigmai}
  \mmm{\sigma_i}^2 = \mmm{I}\hspace{2em}\text{para } i = 1, 2, 3
\end{equation}

Además, anticonmutan porque son matrices cuadradas
\begin{equation}\label{eq:spin12-sigmai-sigmaj}
  \mmm{\sigma_i} \mmm{\sigma_j} = - \mmm{\sigma_j} \mmm{\sigma_i}
  \hspace{1em}
  (\text{para } i\neq j)
\end{equation}

Estas matrices cumplen las propiedades de un \emph{álgebra de Clifford}.


\section{Matrices de Pauli y rotaciones}
Vamos a fijarnos en un vector cualquiera de $\symbb{R}^3$, por ejemplo
$\vvv{v} = (v_1,v_2,v_3)$. Construimos una combinación lineal de las matrices
de Pauli donde los coeficientes son las coordenadas del vector
\[
  \mmm{A} = v_1\mmm{\sigma_1} + v_2\mmm{\sigma_2} + v_3\mmm{\sigma_3}
\]

Las matrices de Pauli son linealmente independientes y forman la base de un
cierto espacio vectorial, que es subespacio de las matrices $2\times 2$
complejas.
El resultado de la combinación lineal debe ser otra matriz de ese subespacio.

Pero el cuadrado de esta combinación lineal no cumple las
propiedades~\eqref{eq:spin12-sigmai-sigmai} y~\eqref{eq:spin12-sigmai-sigmaj}
\begin{align*}
  \mmm{A}^2
  &=
  (v_1\mmm{\sigma_1} + v_2\mmm{\sigma_2} + v_3\mmm{\sigma_3})^2
  = {\scriptstyle
    (v_1\mmm{\sigma_1} + v_2\mmm{\sigma_2} + v_3\mmm{\sigma_3})
    (v_1\mmm{\sigma_1} + v_2\mmm{\sigma_2} + v_3\mmm{\sigma_3})}\\
  &=
    {\scriptstyle
    v_1^2\mmm{\sigma_1}^2 + v_2^2\mmm{\sigma_2}^2 + v_3^2\mmm{\sigma_3}^2
    + v_1v_2(\mmm{\sigma_1}\mmm{\sigma_2} + \mmm{\sigma_2}\mmm{\sigma_1})
  + v_1v_3(\mmm{\sigma_1}\mmm{\sigma_3} + \mmm{\sigma_3}\mmm{\sigma_1})
  + v_2v_3(\mmm{\sigma_2}\mmm{\sigma_3} + \mmm{\sigma_3}\mmm{\sigma_2})}\\
  &=
    {\scriptstyle
    v_1^2\mmm{I} + v_2^2\mmm{I} + v_3^2\mmm{I}
  + v_1v_2(\mmm{\sigma_1}\mmm{\sigma_2}-\mmm{\sigma_1}\mmm{\sigma_2})
  + v_1v_3(\mmm{\sigma_1}\mmm{\sigma_3} - \mmm{\sigma_1}\mmm{\sigma_3})
  + v_2v_3(\mmm{\sigma_2}\mmm{\sigma_3} - \mmm{\sigma_2}\mmm{\sigma_3})}\\
  &=
    (v_1^2 + v_2^2 + v_3^2)\mmm{I} = |\vvv{v}|^2 \mmm{I}
\end{align*}

La propiedad se cumple si elegimos un vector unitario como
$\xhat{n} = (n_1, n_2, n_3)$, con $|\xhat{n}|^2 = n_1^2 + n_2^2 + n_3^2 = 1$.
Entonces
\[
  \mmm{A}^2
  = (n_1\mmm{\sigma_1} + n_2\mmm{\sigma_2} + n_3\mmm{\sigma_3})^2
  = \left(n_1^2 + n_2^2 + n_3^2\right)^2 \mmm{I}
  = |\xhat{n}|^2 \mmm{I}
  = \mmm{I}
\]

Además, el determinante de $\mmm{A}$ vale -1
\[
  \mmm{A}
  = \left[%
    n_1\begin{pNiceMatrix} 0 & 1\\ 1 & 0\end{pNiceMatrix}
    + n_2\begin{pNiceMatrix} 0 & -i\\ i & 0\end{pNiceMatrix}
    + n_3\begin{pNiceMatrix} 1 & 0\\ 0 & -1\end{pNiceMatrix}
  \right]
  = \begin{pNiceMatrix}
    n_3 & n_1-in_2\\
    n_1+in_2 & -n_3
    \end{pNiceMatrix}
\]
\[
  \det(\mmm{A})
  = \begin{vNiceMatrix}
    n_3 & n_1-in_2\\
    n_1+in_2 & -n_3
  \end{vNiceMatrix}
  = -n_1^2 - n_2^2 - n_3^2 = -1
\]

Pero queremos que el cuadrado de la matriz $\mmm{A}$ valga $-\mmm{I}$,
por analogía con las rotaciones de números complejos \eqref{eq:rot-i2} y
de vectores de $\symbb{R}^2$ \eqref{eq:rot-A2} y que su determinante valga
la unidad, también por analogía con \eqref{eq:rot-modi} y \eqref{eq:rot-detA}.

Esto se consigue incluyendo la unidad imaginaria en la combinación lineal
\begin{equation}\label{eq:rot-generador-rotacion}
  \mmm{A}
  = i (n_1\mmm{\sigma_1} + n_2\mmm{\sigma_2} + n_3\mmm{\sigma_3})
\end{equation}

Ahora, el cuadrado de $\mmm{A}$ tendrá el valor buscado
\begin{equation}\label{eq:spin12-A2}
  \mmm{A}^2
  = i^2 (n_1\mmm{\sigma_1} + n_2\mmm{\sigma_2} + n_3\mmm{\sigma_3})^2
  = (-1) |\xhat{n}|^2 \mmm{I}
  = -\mmm{I}
\end{equation}
Y su determinante también
\[
  \mmm{A}
  = i\left[%
    n_1\begin{pNiceMatrix} 0 & 1\\ 1 & 0\end{pNiceMatrix}
    + n_2\begin{pNiceMatrix} 0 & -i\\ i & 0\end{pNiceMatrix}
    + n_3\begin{pNiceMatrix} 1 & 0\\ 0 & -1\end{pNiceMatrix}
  \right]
  = \begin{pNiceMatrix}
    in_3 & n_2+in_1\\
    -n_2+in_1 & -in_3
    \end{pNiceMatrix}
\]
\begin{equation}\label{eq:spin12-detA}
  \det(\mmm{A})
  = \begin{vNiceMatrix}
    in_3 & n_2+in_1\\
    -n_2+in_1 & -in_3    
  \end{vNiceMatrix}
  = n_1^2 + n_2^2 + n_3^2 = 1
\end{equation}

Lo anterior \eqref{eq:spin12-A2} y \eqref{eq:spin12-detA} nos sugiere que la
matriz~\eqref{eq:rot-generador-rotacion}
\[
  \mmm{A}
  = i (n_1\mmm{\sigma_1} + n_2\mmm{\sigma_2} + n_3\mmm{\sigma_3})
  = i (\xhat{n}\cdot\mmm{\sigma})
\]
es la generadora de una rotación y, de acuerdo con~\eqref{eq:rot-ethetaA-abr},
se puede escribir como
\[
    e^{i\theta \xhat{n}\cdot\mmm{\sigma}}
    = \mmm{I} \cos\theta + i (\xhat{n}\cdot\mmm{\sigma}) \sin\theta
\]

Nótese que $\xhat{n}\cdot\mmm{\sigma}$ se utiliza para abreviar
la notación y, aunque se representa como un producto escalar, no lo es en
realidad, porque este solo se define para vectores de un mismo espacio
vectorial, y resulta que $\xhat{n}$ y $\mmm{\sigma}$ no son elementos del
mismo espacio. Obsérvese también que el producto escalar se define
de manera que su resultado es siempre un escalar y en este caso es una
matriz.

\section{Matrices de Pauli y momento angular}
Se sabe, de mecánica cuántica, que todo objeto que pretenda llamarse momento
angular debe tener tres componentes $\vvv{A}_1$, $\vvv{A}_2$ y $\vvv{A}_3$
que conmutan de la siguiente manera
\begin{align*}
  [\vvv{A}_1,\vvv{A}_2] &= i\vvv{A}_3\\
  [\vvv{A}_2,\vvv{A}_3] &= i\vvv{A}_1\\
  [\vvv{A}_3,\vvv{A}_1] &= i\vvv{A}_2\\
  [\vvv{A}_1,\vvv{A}_1] &= [\vvv{A}_2,\vvv{A}_2] = [\vvv{A}_3,\vvv{A}_3] = 0
\end{align*}
donde las $\vvv{A}_i$ son las que generan las rotaciones.

Estas condiciones se puede resumir en una, recurriendo al símbolo de
Levi-Civita
\begin{equation}
  [\vvv{A}_i,\vvv{A}_j] = i\epsilon_{ijk}\vvv{A}_k
\end{equation}

Entonces para ver si las matrices de Pauli son las componentes de un momento
angular, tenemos que calcular sus conmutadores. Empezamos con uno de ellos
\[
  [\mmm{\sigma_1},\mmm{\sigma_2}]
  =
    \mmm{\sigma_1}\mmm{\sigma_2} - \mmm{\sigma_2}\mmm{\sigma_1}
    = 2\mmm{\sigma_1}\mmm{\sigma_2}
    = 2
    \begin{pNiceMatrix}
      0 & 1\\
      1 & 0
    \end{pNiceMatrix}
    \begin{pNiceMatrix}
      0 & -i\\
      i & 0
    \end{pNiceMatrix}
    = 2i\mmm{\sigma_3}
\]

Vemos que da el doble de lo que se requiere, por lo que las $\mmm{\sigma_i}$
no son las componentes de un momento angular. Pero las matrices
$\mmm{\sigma_i}/2$ sí lo son
\begin{align*}
  \left[\frac{\mmm{\sigma_1}}{2},\frac{\mmm{\sigma_2}}{2}\right]
  &= \frac{1}{4}
    (\mmm{\sigma_1}\mmm{\sigma_2} - \mmm{\sigma_2}\mmm{\sigma_1})
    = \frac{1}{4}\mmm{\sigma_1}\mmm{\sigma_2}
    =
    \frac{1}{4}
    \begin{pNiceMatrix}
      0 & 1\\
      1 & 0
    \end{pNiceMatrix}
    \begin{pNiceMatrix}
      0 & -i\\
      i & 0
    \end{pNiceMatrix}
    = i\frac{\sigma_3}{2}\\
  \left[\frac{\mmm{\sigma_2}}{2},\frac{\mmm{\sigma_3}}{2}\right]
  &=
    \frac{1}{4}
    (\mmm{\sigma_2}\mmm{\sigma_3} - \mmm{\sigma_3}\mmm{\sigma_2})
    = \frac{1}{4}\mmm{\sigma_2}\mmm{\sigma_3}
    =
    \frac{1}{4}
    \begin{pNiceMatrix}
      0 & -i\\
      i & 0
    \end{pNiceMatrix}
    \begin{pNiceMatrix}
      1 & 0\\
      0 & -1
    \end{pNiceMatrix}
    = i\frac{\sigma_1}{2}\\
  \left[\frac{\mmm{\sigma_3}}{2},\frac{\mmm{\sigma_1}}{2}\right]
  &=
    \frac{1}{4}
    (\mmm{\sigma_3}\mmm{\sigma_1} - \mmm{\sigma_1}\mmm{\sigma_3})
    = \frac{1}{4}\mmm{\sigma_3}\mmm{\sigma_1}
    =
    \frac{1}{4}
    \begin{pNiceMatrix}
      1 & 0\\
      0 & -1
    \end{pNiceMatrix}
    \begin{pNiceMatrix}
      0 & 1\\
      1 & 0
    \end{pNiceMatrix}
    = i\frac{\sigma_2}{2}  
\end{align*}

De manera que las $\mmm{\sigma}/2$ forman un momento angular
\[
  \frac{\mmm{\sigma}}{2}
  = \left(%
    \frac{\mmm{\sigma_1}}{2},
    \frac{\mmm{\sigma_2}}{2},
    \frac{\mmm{\sigma_3}}{2}
    \right)
\]

El operador de rotación pasiva correspondiente, es
\[
  \mmm{R}(\xhat{n},\theta)
  =
  e^{i\theta\left(\xhat{n}\cdot\frac{\mmm{\sigma}}{2}\right)}
  =
  e^{i\frac{\theta}{2}(\xhat{n}\cdot\mmm{\sigma})}
\]

Este operador rota estados de spin 1/2, esto es vectores del espacio vectorial
complejo $\symbb{C}^2$.
\begin{quote}
 Cuando se gira el sistema de coordenadas (rotación pasiva), los estados
 de spin 1/2 giran según
  \begin{equation}\label{eq:spin12-e-giro-pasivo-spin}
    \mmm{R}(\xhat{n},\theta)
    = e^{i\frac{\theta}{2}(\xhat{n}\cdot\mmm{\sigma})}
    = \mmm{I} \cos\frac{\theta}{2}
    + i (\xhat{n}\cdot\mmm{\sigma}) \sin\frac{\theta}{2}
  \end{equation}
\end{quote}

Desarrollamos el operador de rotación pasiva (giro de ejes de coordenadas)
%\NiceMatrixOptions{cell-space-top-limit = 1pt,cell-space-bottom-limit = 1pt}
\begin{align*}
  \xhat{n}\cdot\mmm{\sigma}
  &=
    n_1\mmm{\sigma_1} + n_2\mmm{\sigma_2} + n_3\mmm{\sigma_3}
    = n_1\begin{pNiceMatrix}0 & 1\\ 1 & 0\end{pNiceMatrix}
    + n_2\begin{pNiceMatrix}0 & -i\\ i & 0\end{pNiceMatrix}
    + n_3\begin{pNiceMatrix}1 & 0\\ 0 & -1\end{pNiceMatrix}\\
  &=
    n_1\begin{pNiceMatrix}0 & n_1\\ n_1 & 0\end{pNiceMatrix}
    + \begin{pNiceMatrix}0 & -in_2\\ n_2i & 0\end{pNiceMatrix}
    + \begin{pNiceMatrix}n_3 & 0\\ 0 & -n_3\end{pNiceMatrix}\\
  &=
    \begin{pNiceMatrix}
      n_3 & n_1-in_2\\
      n_1+in_2 & -n_3
    \end{pNiceMatrix}
\end{align*}
\begin{align*}
  e^{i\frac{\theta}{2}(\xhat{n}\cdot\mmm{\sigma})}
  &=
    \cos\frac{\theta}{2}\mmm{I}
    + i\sin\frac{\theta}{2}(\xhat{n}\cdot\mmm{\sigma})\\
  &=
    \cos\frac{\theta}{2}
    \begin{pNiceMatrix}
      1 & 0\\
      0 & 1
    \end{pNiceMatrix}
    + i\sin\frac{\theta}{2}
    \begin{pNiceMatrix}
      n_3 & n_1-in_2\\
      n_1+in_2 & -n_3
    \end{pNiceMatrix}\\
  &=
    \begin{pNiceMatrix}
      \cos\frac{\theta}{2} & 0\\
      0 & \cos\frac{\theta}{2}
    \end{pNiceMatrix}
    + \begin{pNiceMatrix}
      in_3\sin\frac{\theta}{2} & i (n_1-in_2) \sin\frac{\theta}{2}\\[.5ex]
      i (n_1+in_2) \sin\frac{\theta}{2} & -in_3\sin\frac{\theta}{2}
    \end{pNiceMatrix}\\
  &=
    \begin{pNiceMatrix}
      \cos\frac{\theta}{2}+in_3\sin\frac{\theta}{2}
      & (n_2+in_1)\sin\frac{\theta}{2}\\[0.5ex]
      (-n_2+in_1)\sin\frac{\theta}{2}
      & \cos\frac{\theta}{2}-in_3\sin\frac{\theta}{2}
    \end{pNiceMatrix}  
\end{align*}

Vamos a poner un ejemplo sencillo. Supongamos que tenemos en el origen
de coordenadas una partícula de spin $1/2$, como el electrón.
Y que el espinor que caracteriza su estado de spin es
\[
  \begin{pNiceMatrix}
    \alpha\\
    \beta
  \end{pNiceMatrix}
\]

La probabilidad de encontrar al electrón con \emph{spin hacia arriba} es
\[
  \alpha^* \alpha = |\alpha|^2
\]

Al girar el electrón alrededor de un eje $\xhat{n}$ (rotación activa) un
ángulo  $\theta$, el electrón no cambiará de posición (porque sigue estando
en el origen de coordenadas), pero su estado de spin sí habrá cambiado
\begin{align*}
  \begin{pNiceMatrix}
    \alpha'\\
    \beta'
  \end{pNiceMatrix}
  &=
    e^{-i\theta (\xhat{n}\cdot\frac{\mmm{\sigma}}{2})}
    \begin{pNiceMatrix}
      \alpha\\
      \beta
    \end{pNiceMatrix}\\
  &=
    \begin{pNiceMatrix}
      \cos\left(-\frac{\theta}{2}\right)
      + in_3\sin\left(-\frac{\theta}{2}\right)
    & (n_2 + in_1)\sin\left(-\frac{\theta}{2}\right)\\[0.5ex]
    (-n_2+in_1)\sin\left(-\frac{\theta}{2}\right)
    & \cos\left(-\frac{\theta}{2}\right)
    -in_3\sin\left(-\frac{\theta}{2}\right)
  \end{pNiceMatrix}
  \begin{pNiceMatrix}
    \alpha\\
    \beta
  \end{pNiceMatrix}\\
  &=
    \begin{pNiceMatrix}
      \cos\frac{\theta}{2} - in_3\sin\frac{\theta}{2}
      & -(n_2 + in_1)\sin\frac{\theta}{2}\\[3pt]
      (n_2-in_1)\sin\frac{\theta}{2}
      & \cos\frac{\theta}{2}+in_3\sin\frac{\theta}{2}
    \end{pNiceMatrix}
    \begin{pNiceMatrix}
      \alpha\\
      \beta
    \end{pNiceMatrix}\\
  &=
    \begin{pNiceMatrix}
      \left(\cos\frac{\theta}{2} - in_3\sin\frac{\theta}{2}\right) \alpha
      - (n_2 + in_1)\sin\frac{\theta}{2}\, \beta\\[.5ex]
      (n_2-in_1)\sin\frac{\theta}{2}\,\alpha
      + \left(\cos\frac{\theta}{2}+in_3\sin\frac{\theta}{2}\right) \beta
    \end{pNiceMatrix}
\end{align*}

Pongamos tres ejemplos concretos:
\begin{itemize}
\item Si girara un ángulo de $\ang{90} = \pi/2\,\si{\radian}$
  alrededor del eje $y$ ($\xhat{n} = (0,1,0)$), el espinor que representaría
  su estado final de spin, sería
  \[
    \begin{pNiceMatrix}
      \alpha'\\
      \beta'
    \end{pNiceMatrix}
    = \begin{pNiceMatrix}
      \cos\frac{\pi/2}{2} \alpha - \sin\frac{\pi/2}{2}\, \beta\\[3pt]
      \sin\frac{\pi/2}{2}\,\alpha + \cos\frac{\theta}{2} \beta
    \end{pNiceMatrix}
    = \begin{pNiceMatrix}
      \cos\frac{\pi}{4} \alpha - \sin\frac{\pi}{4}\, \beta\\[3pt]
      \sin\frac{\pi}{4}\,\alpha + \cos\frac{\pi}{4} \beta
    \end{pNiceMatrix}
    = \begin{pNiceMatrix}
      \frac{1}{\sqrt{2}} (\alpha - \beta)\\
      \frac{1}{\sqrt{2}} (\alpha + \beta)
    \end{pNiceMatrix}
  \]

  La probabilidad de encontrar al electron con \emph{spin hacia arriba}
  sería diferente de la original $|\alpha|^2$, porque hemos cambiado la
  orientación del electrón con respecto al eje $z$ que indicaba el estado hacia
  arriba o hacia abajo del spin
  \[
    |\alpha'|^2
    = (\alpha')^* \alpha'
    = \left[\frac{1}{\sqrt{2}} (\alpha - \beta)\right]^*
    \left[\frac{1}{\sqrt{2}} (\alpha - \beta)\right]
    = \frac{|\alpha - \beta|^2}{2} \neq |\alpha|^2
  \]
  
\item Si girara un ángulo $\ang{60} = \pi/3\,\si{\radian}$ alrededor
  del eje $z$ ($\xhat{n} = (0,0,1)$), el espinor que representaría su estado
  de spin sería diferente
  \[
    \begin{pNiceMatrix}
      \alpha'\\
      \beta'
    \end{pNiceMatrix}
    = \begin{pNiceMatrix}
      \left(\cos\frac{\pi/3}{2}-i\sin\frac{\pi/3}{2}\right)\alpha\\
      \left(\cos\frac{\pi/3}{2}+i\sin\frac{\pi/3}{2}\right)\beta
    \end{pNiceMatrix}
    = \begin{pNiceMatrix}
      \left(\frac{\sqrt{3}}{2}-i\frac{1}{2}\right)\alpha\\
      \left(\frac{\sqrt{3}}{2}+i\frac{1}{2}\right)\beta
    \end{pNiceMatrix}
  \]

  Pero la probabilidad no cambiaría porque no hemos cambiado la orientación
  del electrón con respecto al eje $z$ que es el que nos indicaba el spin
  hacia arriba o hacia abajo de la partícula
  \[
    |\alpha'|^2
    = (\alpha')^* \alpha'
    = \left(\frac{\sqrt{3}}{2}+i\frac{1}{2}\right)\alpha^*
    \left(\frac{\sqrt{3}}{2}-i\frac{1}{2}\right)\alpha
    = \left(\frac{3}{4}+\frac{1}{4}\right) \alpha^* \alpha
    = |\alpha|^2
  \]
  
\item Como curiosidad, si diera una vuelta completa, por ejemplo\footnotemark{}
  alrededor del eje $z$, esto es, si girara un ángulo de
  $\ang{360} = 2\pi\,\si{\radian}$
  \footnotetext{Para hacer el cálculo más sencillo, pero podría ser alrededor
    de cualquier dirección.}
  y volviera a su orientación inicial, el estado de spin sería diferente al
  inicial\footnotemark{}
  \footnotetext{Compruebe el lector que tendría que dar
    \emph{dos vueltas completas} ($\theta=4\pi\,\si{\radian}$) para volver el
    espinor al estado inicial.}
  \[
    \begin{pNiceMatrix}
      \alpha'\\
      \beta'
    \end{pNiceMatrix}
    = \begin{pNiceMatrix}
      \left(\cos\frac{2\pi}{2}-i\sin\frac{2\pi}{2}\right)\alpha\\[0.5ex]
      \left(\cos\frac{2\pi}{2}+i\sin\frac{2\pi}{2}\right)\beta
    \end{pNiceMatrix}
    =
    \begin{pNiceMatrix}
      -\alpha\\
      -\beta
    \end{pNiceMatrix}
  \]
  
  El electrón no tendría el mismo estado de spin, aunque nosotros lo viéramos
  igual, porque la probabilidad de encontrarlo con \emph{spin hacia arriba}
  o \emph{hacia abajo} sería la misma que antes de girarlo
  \[
    |\alpha'|^2 = (-\alpha^*) (-\alpha) = \alpha^* \alpha = |\alpha|^2
  \]

  \begin{quote}
    ``Si un electrón gira dando una vuelta completa (o la dan los ejes de
    coordenadas), la partícula no termina en el mismo estado interno de spin
    en el que estaba inicialmente; sería necesario darle dos vueltas para
    encontrarlo exactamente igual que al principio.''
  \end{quote}
\end{itemize}

\section{Base de estados de spin 1/2}
Como el spin es $1/2$, significa que hay dos posibilidades mútuamente
excluyentes.

Por comodidad, es tradición elegir como base de estados de spin 1/2 los
vectores propios de $\mmm{\sigma_3} = \mmm{\sigma_z}$, ya que la matriz es
diagonal y los vectores propios son\footnotemark{}
\footnotetext{Se podría elegir cualquier par de símbolos para representar
  estos estados; aquí hemos elegido $\ket{z+}$ y $\ket{z_-}$. En otras partes
  de este volumen elegiremos $\ket{0}$ y $\ket{1}$, que nos recordarían los
  estados de un \emph{qbit}.}
\[
  \ket{z_+}
  =
  \begin{pNiceMatrix}
    1\\
    0
  \end{pNiceMatrix}
  \hspace{.5em}
  y\hspace{.5em}
  \ket{z_-}
  =
  \begin{pNiceMatrix}
    0\\
    1
  \end{pNiceMatrix}
\]
y los valores propios respectivos, $+1$ y $-1$
\begin{align*}
  &\begin{pNiceMatrix}
    1 & 0\\
    0 & -1
  \end{pNiceMatrix}
  \begin{pNiceMatrix}
    1\\
    0
  \end{pNiceMatrix}
  = +1\,
  \begin{pNiceMatrix}
    1\\
    0
  \end{pNiceMatrix}\\
  &\begin{pNiceMatrix}
    1 & 0\\
    0 & -1
  \end{pNiceMatrix}
  \begin{pNiceMatrix}
    0\\
    1
  \end{pNiceMatrix}
  = -1\,
  \begin{pNiceMatrix}
    0\\
    1
  \end{pNiceMatrix}  
\end{align*}

Denotaremos por
\[
  \ket{z_+}
  = \begin{pNiceMatrix}
    1\\
    0
  \end{pNiceMatrix}
\]
el estado de un sistema que tiene el spin \emph{hacia arriba};
en otras palabras, una partícula tiene un spin caracterizado por $\ket{z_+}$
cuando la componente $z$ de su spin es $+1$ (apunta hacia arriba).

La otra función propia
\[
  \ket{z_-}
  = \begin{pNiceMatrix}
    0\\
    1
  \end{pNiceMatrix}
\]
describe el estado de un sistema que tiene el spin \emph{hacia abajo}.
Es decir, una partícula tiene un spin caracterizado por $\ket{z_-}$ cuando
la componente $z$ de su spin es $-1$ (apunta hacia abajo).

Estos dos estados forman una base ortonormal de $\symbb{C}^2$
\begin{align}\label{eq:spin12-base-normalizada}
  &\braket{z_+|z_+} = \braket{z_-|z_-} = 1\\
  \label{eq:spin12-base-ortogonal}
  &\braket{z_+|z_-} = \braket{z_-|z_+} = 0
\end{align}
Los dos estados son ortogonales~\eqref{eq:spin12-base-ortogonal}
porque son mutuamente excluyentes.

Cualquier otro estado de spin $1/2$ del sistema se podrá expresar como
combinación lineal de ambos
\[
  \ket{\psi}
  =\begin{pNiceMatrix}
    \alpha\\
    \beta
  \end{pNiceMatrix}
  = \alpha \ket{z_+} + \beta \ket{z_-}
\]
donde los coeficientes son números complejos $\alpha, \beta \in\symbb{C}$.

Ejercicio
¿Cuáles hubieran sido los valores y vectores propios si se hubiera elegido
como base de estados los de la matrices $\mmm{\sigma_1}$, $\mmm{\sigma_2}$
y $\mmm{\sigma_3}$? Ver soluciones en el apéndice.

\section{Helicidad}

\subsection{Matriz de rotación activa}
Recordemos que según~\eqref{eq:spin12-e-giro-pasivo-spin}, la matriz
de giro pasivo de espinores es
\[
  e^{i\frac{\theta}{2}(\xhat{n}\cdot\mmm{\sigma})}
  = \mmm{I} \cos\frac{\theta}{2}
  + i (\xhat{n}\cdot\mmm{\sigma}) \sin\frac{\theta}{2}
\]
Nótese el signo positivo de la exponencial, que interpretaremos como el
indicador de que giran los ejes de coordenadas (giro pasivo) en sentido
antihorario alrededor de la dirección indicada por $\xhat{n}$.

Cuando se encuentra un signo negativo
\begin{equation}\label{eq:spin12-e-giro-activo-spin}
  e^{-i\frac{\theta}{2}(\xhat{n}\cdot\mmm{\sigma})}
  = \mmm{I} \cos\frac{\theta}{2}
  - i (\xhat{n}\cdot\mmm{\sigma}) \sin\frac{\theta}{2}  
\end{equation}
lo interpretamos como la indicación de un giro del espinor (giro activo)
en sentido antihorario alrededor de la dirección $\xhat{n}$, o bien, un
giro de los ejes (giro pasivo) en sentido horario (negativo).

Como ejercicio, rotaremos \ang{180} el espinor $\ket{z+}$ alrededor del eje $y$
para transformarlo en $\ket{z-}$, ver figura~\eqref{fig:spin12-giro180z+}.
A continuación, presentamos los parámetros que se aplicarán en la rotación
\begin{align*}
  &\xhat{n} = (0,1,0)\\
  &\xhat{n}\cdot\mmm{\sigma} = \mmm{\sigma_2}\\
  &\theta = \pi\,\si{\radian}
\end{align*}

Considerando~\eqref{eq:spin12-e-giro-activo-spin}
\begin{align*}
  e^{-i\frac{\pi}{2}(\xhat{n}\cdot\mmm{\sigma})} \ket{z_+}
  &=
    \left(
    \cos\frac{\pi}{2} - i \mmm{\sigma_2} \sin\frac{\pi}{2}
    \right)
    \ket{z_+}
    = -i\mmm{\sigma_2} \ket{z_+}\\
  &=
    -i
    \begin{pNiceMatrix}
      0 & -i\\
      i & 0
    \end{pNiceMatrix}
    \begin{pNiceMatrix}
      1\\
      0
    \end{pNiceMatrix}
  = -i \begin{pNiceMatrix}
      0\\
      i
    \end{pNiceMatrix}
  = \begin{pNiceMatrix}
    0\\
    1
    \end{pNiceMatrix}
    = \ket{z_-}
\end{align*}
\begin{figure}[ht]
  \centering
  \begin{minipage}{.43\linewidth}
  % 
  \newcommand{\colorspinorplus}{green!60!black}
  \newcommand{\colorspinorplustext}{green!45!black}
  \newcommand{\colorspinorminus}{red!80!black}
  \newcommand{\colorspinorminustext}{red!80!black}
  \newcommand{\colorelectron}{blue}
  % Escala
  \def\scl{1}
  % Eje y
  \pgfmathsetmacro{\YMLONG}{-2}
  \pgfmathsetmacro{\YPLONG}{2}
  % Eje z
  \pgfmathsetmacro{\ZMLONG}{-2}
  \pgfmathsetmacro{\ZPLONG}{2}
  % Vectores de estado
  \pgfmathsetmacro{\KETLONG}{.7}
  \pgfmathsetmacro{\KETYCOORD}{0}
  % Vector de estado |0>
  \pgfmathsetmacro{\ZEROKETYCOORD}{\KETYCOORD}
  \pgfmathsetmacro{\ZEROKETZCOORD}{.7}
  \pgfmathsetmacro{\ZEROKETANG}{90}
  % Vector de estado |1>
  \pgfmathsetmacro{\ONEKETYCOORD}{\KETYCOORD}
  \pgfmathsetmacro{\ONEKETZCOORD}{-.7}
  \pgfmathsetmacro{\ONEKETANG}{-90}
  %    % 
  % \centering
  %
  \tikzfading[name=fade out, inner color=transparent!0,
  outer color=transparent!100]

  \begin{tikzpicture}[%
    scale=\scl,
    baseline,
    every node/.style={black,font=\small},
    ejey/.style={->},
    ejez/.style={->, black!50},
    ket/.style={-{Latex[round,width=5pt]}, shorten >=1.2pt, line width=1.2pt},
    particula/.style={fill=red, draw=black},
    electron/.style={\colorelectron,path fading=fade out},
    background/.style={
      line width=\bgborderwidth,
      draw=\bgbordercolor,
      fill=\bgcolor,
    },
    ]
    % Coordenadas
    \coordinate (O) at (0,0);
    % Ejes
    \coordinate (yini) at (\YMLONG, 0);
    \coordinate (yfin) at (\YPLONG, 0);
    \coordinate (zini) at (0, \ZMLONG);
    \coordinate (zfin) at (0, \ZPLONG);
    % Vectores de estado
    \coordinate (zeroketini) at (\KETYCOORD, \ZEROKETZCOORD);
    \path (zeroketini) -- coordinate (zeroketnode) +(\ZEROKETANG:\KETLONG)
    coordinate (zeroketfin);
    \coordinate (oneketini) at (\KETYCOORD, \ONEKETZCOORD);
    \path (oneketini) -- coordinate (oneketnode) +(\ONEKETANG:\KETLONG)
    coordinate (oneketfin);
    % Ejes
    \draw[ejey] (yini) -- (yfin);
    \node[right, name=letraejey] at (yfin) {$y$};
    \draw[ejez] (zini) -- (zfin);
    \node[above, name=letraejez] at (zfin) {$z$};
    % Rotación
    \draw[-{Stealth[round,width=7.5pt,bend]}, black!40, line width=1.8pt]
    (1.7,.2) arc[%
    start angle=30,end angle=330,x radius=.25,y radius=.4
    ];
    \node[above left=6pt and -6pt,black!70] at (1.7,.2) {$\pi\,\si{\radian}$};
    % |0>
    \draw[ket,\colorspinorplus] (zeroketini) -- (zeroketfin);
    %\filldraw[particula] (zeroketini) circle[radius=1.4pt];
    % Nube de probabilidad (estado espacial)
    \fill[electron] (zeroketini) circle [radius=3.5pt];
    %node[below right=4pt and 0pt] {$\scriptstyle e^-$};
    \node[below left=-6pt and 4pt,\colorspinorplustext]
    at (zeroketnode) {$\ket{z_+}$};
    % |z->
    %\draw[ket] (oneketini) -- (oneketfin);
    %\filldraw[particula] (oneketini) circle[radius=1.4pt];
    % Nube de probabilidad (estado espacial)
    %\fill [electron] (oneketini) circle [radius=3.25pt];
    %\node[above left=-6pt and 4pt] at (oneketnode) {$\ket{z_-}$};
    % Fondo amarillo
    \begin{scope}[on background layer]
      % \node [line width=1pt, draw=\backgroundbordercolor,
      % fill=\backgroundcolor, fit= (O) (letraejex) (letraejey)] {};
      \node [background, fit= (yini) (zini) (letraejey) (letraejez)] {};
    \end{scope}
  \end{tikzpicture}
\end{minipage}
\hspace{1em}
\begin{minipage}{.43\linewidth}
  % 
  \newcommand{\colorspinorplus}{green!60!black}
  \newcommand{\colorspinorplustext}{green!45!black}
  \newcommand{\colorspinorminus}{red!80!black}
  \newcommand{\colorspinorminustext}{red!80!black}
  \newcommand{\colorelectron}{blue}
  \newcommand{\colorelectronapagado}{black!20}
  % Escala
  \def\scl{1}
  % Eje y
  \pgfmathsetmacro{\YMLONG}{-2}
  \pgfmathsetmacro{\YPLONG}{2}
  % Eje z
  \pgfmathsetmacro{\ZMLONG}{-2}
  \pgfmathsetmacro{\ZPLONG}{2}
  % Vectores de estado
  \pgfmathsetmacro{\KETLONG}{.7}
  \pgfmathsetmacro{\KETYCOORD}{0}
  % Vector de estado |0>
  \pgfmathsetmacro{\ZEROKETYCOORD}{\KETYCOORD}
  \pgfmathsetmacro{\ZEROKETZCOORD}{.7}
  \pgfmathsetmacro{\ZEROKETANG}{90}
  % Vector de estado |1>
  \pgfmathsetmacro{\ONEKETYCOORD}{\KETYCOORD}
  \pgfmathsetmacro{\ONEKETZCOORD}{-.7}
  \pgfmathsetmacro{\ONEKETANG}{-90}
  % 
  %\centering
  %
  \tikzfading[name=fade out, inner color=transparent!0,
  outer color=transparent!100]
  \begin{tikzpicture}[%
    scale=\scl,
    baseline,
    every node/.style={black,font=\small},
    ejey/.style={->},
    ejez/.style={->, black!50},
    ket/.style={-{Latex[round,width=5pt]}, shorten >=1.2pt, line width=1.2pt},
    particula/.style={fill=red, draw=black},
    electron/.style={\colorelectron,path fading=fade out},
    electronapagado/.style={\colorelectronapagado,path fading=fade out},
    background/.style={
      line width=\bgborderwidth,
      draw=\bgbordercolor,
      fill=\bgcolor,
    },
    ]
    % Coordenadas
    \coordinate (O) at (0,0);
    % Ejes
    \coordinate (yini) at (\YMLONG, 0);
    \coordinate (yfin) at (\YPLONG, 0);
    \coordinate (zini) at (0, \ZMLONG);
    \coordinate (zfin) at (0, \ZPLONG);
    % Vectores de estado
    \coordinate (zeroketini) at (\KETYCOORD, \ZEROKETZCOORD);
    \path (zeroketini) -- coordinate (zeroketnode) +(\ZEROKETANG:\KETLONG)
    coordinate (zeroketfin);
    \coordinate (oneketini) at (\KETYCOORD, \ONEKETZCOORD);
    \path (oneketini) -- coordinate (oneketnode) +(\ONEKETANG:\KETLONG)
    coordinate (oneketfin);
    % Ejes
    \draw[ejey] (yini) -- (yfin);
    \node[right, name=letraejey] at (yfin) {$y$};
    \draw[ejez] (zini) -- (zfin);
    \node[above, name=letraejez] at (zfin) {$z$};
    % Rotación
    \draw[-{Stealth[round,width=7.5pt,bend]}, black!10, line width=1.8pt]
    (1.7,.2) arc[%
    start angle=30,end angle=330,x radius=.25,y radius=.4
    ];
    \node[above left=6pt and -6pt,black!20] at (1.7,.2) {$\pi\,\si{\radian}$};
    % |z+>
    \draw[ket,black!20] (zeroketini) -- (zeroketfin);
    %\filldraw[particula] (zeroketini) circle[radius=1.4pt];
    % Nube de probabilidad (estado espacial)
    \fill[electronapagado] (zeroketini) circle [radius=3.5pt];
    %node[below right=4pt and 0pt] {$\scriptstyle e^-$};
    \node[below left=-6pt and 4pt,black!20] at (zeroketnode) {$\ket{z_+}$};
    % |z->
    \draw[ket,\colorspinorminus] (oneketini) -- (oneketfin);
    %\filldraw[particula] (oneketini) circle[radius=1.4pt];
    % Nube de probabilidad (estado espacial)
    \fill [electron] (oneketini) circle [radius=3.25pt];
    %node[below right=4pt and 0pt] {$\scriptstyle e^-$};
    \node[above left=-6pt and 4pt,\colorspinorminustext]
    at (oneketnode) {$\ket{z_-}$};
    % Fondo amarillo
    \begin{scope}[on background layer]
      % \node [line width=1pt, draw=\backgroundbordercolor,
      % fill=\backgroundcolor, fit= (O) (letraejex) (letraejey)] {};
      \node [background, fit= (yini) (zini) (letraejey) (letraejez)] {};
    \end{scope}
  \end{tikzpicture}
\end{minipage}
  \caption{Al rotar el vector de estado $\ket{z_+}$ un ángulo de \ang{180}
    alrededor del eje $y$, se obtiene $\ket{z_-}$ (para este ángulo,
    $\pi~\si{rad}$, daría lo mismo una rotación activa o pasiva).
    En azul se representa un electrón.}
  \label{fig:spin12-giro180z+}
\end{figure}

\subsection{Componente del spin en una dirección arbitraria}
Imaginemos un electrón situado en el origen de coordenadas.
Deseamos medir su spin según el eje $z$, por ejemplo.
Se puede elegir el eje $z$ en la dirección que nos parezca.
Supondremos que ya hemos elegido esa dirección arbitraria, junto con el resto
de ejes cartesianos.
\begin{figure}[ht]
  \centering
  % COLORES
  \newcommand{\colordirectrizfront}{black}
  \newcommand{\colordirectrizback}{black!50}
  \newcommand{\colormomento}{blue!85!black}
  \newcommand{\colorlinaux}{black!15}
  \newcommand{\colorndirector}{black!60}
  \newcommand{\colorelectron}{blue}
  \newcommand{\colorhelicplus}{green!60!black}
  \newcommand{\colorhelicplustext}{green!45!black}
  \newcommand{\colorhelicminus}{red!80!black}
  \newcommand{\colorhelicminustext}{red!80!black}
  %
  \def\scl{1.2}
  % EJES
  % x
  \pgfmathsetmacro{\LONGEJEPOSX}{3}
  \pgfmathsetmacro{\LONGEJENEGX}{3}
  % y
  \pgfmathsetmacro{\LONGEJEPOSY}{3}
  \pgfmathsetmacro{\LONGEJENEGY}{1.5}
  % z
  \pgfmathsetmacro{\LONGEJEPOSZ}{2}
  \pgfmathsetmacro{\LONGEJENEGZ}{1.5}

  % VECTOR DIRECTOR
  \pgfmathsetmacro{\PMOD}{2.2}
  \pgfmathsetmacro{\NMOD}{.5}
  \pgfmathsetmacro{\THETA}{55}
  \pgfmathsetmacro{\PHI}{50}
  % 
  \tdplotsetmaincoords{60}{110}
  % MOMENTO LINEAL
  \pgfmathsetmacro{\PR}{\PMOD}
  \pgfmathsetmacro{\PTHETA}{\THETA}
  \pgfmathsetmacro{\PPHI}{\PHI}
  % RECTA DE ACCIÓN INICIO
  \pgfmathsetmacro{\ACCINIR}{3.5}
  \pgfmathsetmacro{\ACCINITHETA}{180-\THETA}
  \pgfmathsetmacro{\ACCINIPHI}{\PHI + 180}
  \pgfmathsetmacro{\ACCINIX}{\ACCINIR * sin(\ACCINITHETA) * cos(\ACCINIPHI)}
  \pgfmathsetmacro{\ACCINIY}{\ACCINIR * sin(\ACCINITHETA) * sin(\ACCINIPHI)}
  \pgfmathsetmacro{\ACCINIZ}{\ACCINIR * cos(\ACCINITHETA)}
  % RECTA DE ACCIÓN FIN
  \pgfmathsetmacro{\ACCFINR}{7.0}
  \pgfmathsetmacro{\ACCFINTHETA}{\THETA}
  \pgfmathsetmacro{\ACCFINPHI}{\PHI}
  \pgfmathsetmacro{\ACCFINX}{\ACCFINR * sin(\ACCFINTHETA) * cos(\ACCFINPHI)}
  \pgfmathsetmacro{\ACCFINY}{\ACCFINR * sin(\ACCFINTHETA) * sin(\ACCFINPHI)}
  \pgfmathsetmacro{\ACCFINZ}{\ACCFINR * cos(\ACCFINTHETA)}
  %
  \tikzfading[name=fade out, inner color=transparent!0,
  outer color=transparent!100]
  
  \begin{tikzpicture}[
    scale=\scl,
    tdplot_main_coords,
    every node/.style={font=\small},
    linaux/.style={ultra thin, color=\colorlinaux},
    directrizfront/.style={dashed,ultra thin,\colordirectrizfront},
    directrizback/.style={dashed,ultra thin,\colordirectrizback},
    ndirector/.style={-{Latex[round]},line width=0.8pt,\colorndirector},
    ejeposit/.style={thick,->},
    ejenegat/.style={ultra thin},
    %electron/.style={fill=\colorelectron, draw=black},
    electron/.style={\colorelectron,path fading=fade out},
    ndirector/.style={%
      -{Latex[round]},color=\colorndirector,line width=1.1pt
    },
    background/.style={
      line width=\bgborderwidth,
      draw=\bgbordercolor,
      fill=\bgcolor,
    },
    ]
    % COORDENADAS
    \coordinate (o) at (0,0,0);
    \coordinate (-dx) at (-\LONGEJENEGX,0,0);
    \coordinate (dx) at (\LONGEJEPOSX,0,0);
    \coordinate (-dy) at (0,-\LONGEJENEGY,0);
    \coordinate (dy) at (0,\LONGEJEPOSY,0);
    \coordinate (-dz) at (0,0,-\LONGEJENEGZ);
    \coordinate (dz) at (0,0,\LONGEJEPOSZ);
    % Recta de acción
    \coordinate (accini) at (\ACCINIX,\ACCINIY,\ACCINIZ);
    \coordinate (accfin) at (\ACCFINX,\ACCFINY,\ACCFINZ);

    % DIBUJO
    % Líneas débiles ejes
    %\draw[ejenegat] (o) -- (-dx);
    %\draw[ejenegat] (o) -- (-dy);
    %\draw[ejenegat] (o) -- (-dz);
    % Ejes
    \draw[ejeposit] (o) -- (dx);
    \draw[ejeposit] (o) -- (dy);
    \draw[ejeposit] (o) -- (dz);
    \node[anchor=north east,name=letrax] at (dx) {$x$};
    \node[anchor=north west,name=letray] at (dy) {$y$};
    \node[anchor=south,name=letraz] at (dz) {$z$};
    
    %  % Giro eje x
    % \tdplotsetthetaplanecoords{-90}
    %  % Notice you have to tell tiks-3dplot you are now in rotated coords
    %  % Since tikz-3dplot swaps the planes in tdplotsetthetaplanecoords,
    %  % the former y axis is now the z axis.
    % \tdplotdrawarc[tdplot_rotated_coords,line width=2pt,
    % -{Latex[length=11pt,width=7pt,flex=1]},color=black!50]
    % {(0,0,1.9)}{0.3}{380}{40}{anchor=south west,color=black}{}
    %  % Reponer parte derecha del eje y
    % \draw[ultra thick,->] (1.9,0,0) -- (\longeje,0,0);
    %% \tdplotdrawarc[tdplot_rotated_coords,line width=2pt,
    %% -{Latex[round,length=6pt,width=4pt]},color=black!50]
    %% {(0,0,1.9)}{0.3}{380}{40}{anchor=south west,color=black}{}
    %%  % Reponer parte derecha del eje y
    %% \draw[ultra thick,->] (1.9,0,0) -- (\longeje,0,0);

    % RECTA DE ACCIÓN
    \draw[directrizback] (accini) -- (o);
    \draw[directrizfront] (o) -- (accfin);

    % VECTOR DIRECTOR n
    \tdplotsetcoord{P}{\PR}{\PTHETA}{\PPHI}
    % draw a vector from origin to point (P)
    % \draw[-{Stealth[width=7pt]},color=red!80!black,ultra thick] (O) -- (P)
    % node[above right=0pt and 0pt] {\footnotesize $\braket{\vec{S}}$};
    % Momento del electrón en el origen
    % Líneas auxiliares por debajo del momento
    %\draw[linaux] (Pxy) -- (Px);
    %\draw[linaux] (O) -- (Pxy);
    
    \draw[ndirector] (o) -- (P);
    %\node[right=0pt,\colormomento] at (P) {\footnotesize $\vvv{p}$};
    \node[above left=0pt and -2pt,\colorndirector]
    at (P) {\footnotesize $\xhat{n}$};
    % Líneas auxiliares por encima del momento
    \draw[linaux] (P) -- (Pxy);
    \draw[linaux] (Pxy) -- (Py);
    \draw[linaux] (Pxy) -- (Px);
    \draw[linaux,thick,-{Latex[round]}] (o) -- (Pxy);

    %\draw[linaux] (P) -- (Pz);
    \draw[linaux] (Py) -- (Pyz);
    \draw[linaux] (Pyz) -- (Pz);
    \draw[linaux] (Pyz) -- (P);
    \draw[linaux] (Pz) -- (Pxz);
    \draw[linaux] (Pxz) -- (P);
    \draw[linaux] (Pxz) -- (Px);

    % draw theta arc and label, using rotated coordinate system
    \tdplotdrawarc
    {(0,0,0)}{0.6}{0}{\PPHI}{below=-2pt}
    {\footnotesize $\phi$}
    
    \tdplotsetthetaplanecoords{\PTHETA}
    
    % draw the angle \phi, and label it
    % syntax:
    % \tdplotdrawarc[coordinate frame, draw options]
    % {center point}{r}{angle}{label options}{label}
    % \tdplotdrawarc{(O)}{0.8}{0}{\phivec}{anchor=north}{\footnotesize $\pi/2$}
    
    % set the rotated coordinate system so the x'-y' plane lies within the
    % "theta plane" of the main coordinate system
    % syntax: \tdplotsetthetaplanecoords{\phi}
    \tdplotsetthetaplanecoords{\PPHI}
    
    % draw theta arc and label, using rotated coordinate system
    \tdplotdrawarc[tdplot_rotated_coords]
    {(0,0,0)}{0.6}{0}{\PTHETA}{above right=-6pt and 0pt}
    {\footnotesize $\theta$}
    
    % Electrón en el origen
    %\filldraw[electron] (o) circle[radius=2pt];
    % Nube de probabilidad (estado espacial)
    \fill[electron] (o) circle [radius=4pt];
    % node[below right=4pt and 0pt] {$\scriptstyle e^-$};

    %\node at (0,.5,3.0) {\ACCINIX};
    %\node at (0,.5,2.6) {\ACCINIY};
    %\node at (0,.5,2.2) {\ACCINIZ};
    % Fondo amarillo
    \begin{scope}[on background layer]
      % \node [line width=1pt, draw=\backgroundbordercolor,
      % fill=\backgroundcolor, fit= (O) (letraejex) (letraejey)] {};
      \node [background, fit= (accini) (accfin) (letrax) (letraz) (letray)] {};
    \end{scope}
  \end{tikzpicture}
  \caption{Se pretende medir la componente del spin del electrón situado en
    el origen de coordenadas, a lo largo de una dirección arbitraria
    representada por el versor $\xhat{n}$.}
  \label{fig:spin12-electron-spin-direccion-n}
\end{figure}

Si el vector de estado de spin del electrón fuera $\ket{\Psi} = \ket{z_+}$
y midiéramos la componente $z$ de su spin, obtendríamos con certeza el valor
$+1$ (spin hacia arriba).
En cambio, si fuera $\ket{\Psi} = \ket{z_-}$, al medir su spin en el eje $z$
obtendríamos con seguridad el valor $-1$ (spin hacia abajo).
También es posible que su estado fuera una combinación lineal de los
anteriores\footnotemark{},
\footnotetext{El spin del electrón estaría en una superposición cuántica de
$\ket{z_+}$ y $\ket{z_-}$.}
$\ket{\Psi} = \alpha \ket{z_+} + \beta \ket{z_-}$,
en cuyo caso obtendríamos la componente $z$ del spin hacia arriba con una
probabilidad $|\alpha|^2$ y otra $|\beta|^2 = 1 - |\alpha|^2$ de encontrarlo
hacia abajo.

Pero no estamos obligados a medir el spin en la dirección del eje $z$.
Podríamos preparar un experimento para medirlo en cualquier otra
dirección, por ejemplo la indicada por el versor $\xhat{n}$, ver
figura~\ref{fig:spin12-electron-spin-direccion-n}.
En este caso, como resultado de la medida obtendríamos un spin
\emph{hacia arriba} $\ket{n_+}$ o \emph{hacia abajo} $\ket{n_-}$ en esa
dirección,
\emph{que es diferente del estado hacia arriba o hacia abajo en la dirección
  $z$}, ver figura~\ref{fig:spin12-electron-estados-spin-direccion-n}.
\begin{figure}[ht]
  \centering
  % COLORES
  \newcommand{\colordirectrizfront}{black}
  \newcommand{\colordirectrizback}{black!50}
  \newcommand{\colormomento}{blue!85!black}
  \newcommand{\colorhelicplus}{green!60!black}
  \newcommand{\colorhelicplustext}{green!45!black}
  \newcommand{\colorhelicminus}{red!80!black}
  \newcommand{\colorhelicminustext}{red!80!black}
  \newcommand{\colorlinaux}{black!15}
  \newcommand{\colorndirector}{black!60}
  \newcommand{\colorelectron}{blue}
  %
  \def\scl{1.2}
  % EJES
  % x
  \pgfmathsetmacro{\LONGEJEPOSX}{3}
  \pgfmathsetmacro{\LONGEJENEGX}{3}
  % y
  \pgfmathsetmacro{\LONGEJEPOSY}{3}
  \pgfmathsetmacro{\LONGEJENEGY}{1.5}
  % z
  \pgfmathsetmacro{\LONGEJEPOSZ}{2}
  \pgfmathsetmacro{\LONGEJENEGZ}{1.5}

  % 
  \pgfmathsetmacro{\PMOD}{2.2}
  \pgfmathsetmacro{\NMOD}{.5}
  \pgfmathsetmacro{\THETA}{55}
  \pgfmathsetmacro{\PHI}{50}
  % 
  \tdplotsetmaincoords{60}{110}
  % HELICIDAD +
  \pgfmathsetmacro{\PR}{\PMOD}
  \pgfmathsetmacro{\PTHETA}{\THETA}
  \pgfmathsetmacro{\PPHI}{\PHI}
  % HELICIDAD -
  \pgfmathsetmacro{\HELICMINR}{\PMOD}
  \pgfmathsetmacro{\HELICMINTHETA}{180 - \THETA}
  \pgfmathsetmacro{\HELICMINPHI}{\PHI + 180}
  \pgfmathsetmacro{\HELICMINX}{\HELICMINR*sin(\HELICMINTHETA)*cos(\HELICMINPHI)}
  \pgfmathsetmacro{\HELICMINY}{\HELICMINR*sin(\HELICMINTHETA)*sin(\HELICMINPHI)}
  \pgfmathsetmacro{\HELICMINZ}{\HELICMINR*cos(\HELICMINTHETA)}
  % VECTOR DIRECTOR INICIO
  \pgfmathsetmacro{\DIRINIR}{4.2}
  \pgfmathsetmacro{\DIRINITHETA}{\THETA}
  \pgfmathsetmacro{\DIRINIPHI}{\PHI}
  \pgfmathsetmacro{\DIRINIX}{\DIRINIR * sin(\DIRINITHETA) * cos(\DIRINIPHI)}
  \pgfmathsetmacro{\DIRINIY}{\DIRINIR * sin(\DIRINITHETA) * sin(\DIRINIPHI)}
  \pgfmathsetmacro{\DIRINIZ}{\DIRINIR * cos(\DIRINITHETA)}
  % VECTOR DIRECTOR FIN
  \pgfmathsetmacro{\DIRFINR}{5.5}
  \pgfmathsetmacro{\DIRFINTHETA}{\THETA}
  \pgfmathsetmacro{\DIRFINPHI}{\PHI}
  \pgfmathsetmacro{\DIRFINX}{\DIRFINR * sin(\DIRFINTHETA) * cos(\DIRFINPHI)}
  \pgfmathsetmacro{\DIRFINY}{\DIRFINR * sin(\DIRFINTHETA) * sin(\DIRFINPHI)}
  \pgfmathsetmacro{\DIRFINZ}{\DIRFINR * cos(\DIRFINTHETA)}
  % RECTA DE ACCIÓN INICIO
  \pgfmathsetmacro{\ACCINIR}{3.5}
  \pgfmathsetmacro{\ACCINITHETA}{180 - \THETA}
  \pgfmathsetmacro{\ACCINIPHI}{\PHI + 180}
  \pgfmathsetmacro{\ACCINIX}{\ACCINIR * sin(\ACCINITHETA) * cos(\ACCINIPHI)}
  \pgfmathsetmacro{\ACCINIY}{\ACCINIR * sin(\ACCINITHETA) * sin(\ACCINIPHI)}
  \pgfmathsetmacro{\ACCINIZ}{\ACCINIR * cos(\ACCINITHETA)}
  % RECTA DE ACCIÓN FIN
  \pgfmathsetmacro{\ACCFINR}{7.0}
  \pgfmathsetmacro{\ACCFINTHETA}{\THETA}
  \pgfmathsetmacro{\ACCFINPHI}{\PHI}
  \pgfmathsetmacro{\ACCFINX}{\ACCFINR * sin(\ACCFINTHETA) * cos(\ACCFINPHI)}
  \pgfmathsetmacro{\ACCFINY}{\ACCFINR * sin(\ACCFINTHETA) * sin(\ACCFINPHI)}
  \pgfmathsetmacro{\ACCFINZ}{\ACCFINR * cos(\ACCFINTHETA)}
  %
  \tikzfading[name=fade out, inner color=transparent!0,
  outer color=transparent!100]
  
  \begin{tikzpicture}[
    scale=\scl,
    tdplot_main_coords,
    every node/.style={font=\small},
    linaux/.style={ultra thin, color=\colorlinaux},
    directrizfront/.style={dashed,ultra thin,\colordirectrizfront},
    directrizback/.style={dashed,ultra thin,\colordirectrizback},
    ndirector/.style={-{Latex[round]},line width=0.8pt,\colorndirector},
    ejeposit/.style={thick,->},
    ejenegat/.style={ultra thin, black!20},
    electron/.style={fill=\colorelectron, draw=black},
    momento/.style={%
      %-{Latex[width=5pt,length=7pt]},color=blue!90!black,line width=1.1pt
      -{Latex[round]},color=\colormomento,line width=1.1pt
    },
    helicidadplus/.style={%
      %-{Latex[width=5pt,length=7pt]},color=blue!90!black,line width=1.1pt
      -{Latex[round]},color=\colorhelicplus,line width=1.1pt
    },
    helicidadminus/.style={%
      %-{Latex[width=5pt,length=7pt]},color=blue!90!black,line width=1.1pt
      -{Latex[round]},color=\colorhelicminus,line width=1.1pt
    },
    background/.style={
      line width=\bgborderwidth,
      draw=\bgbordercolor,
      fill=\bgcolor,
    },
    ]
    % COORDENADAS
    \coordinate (o) at (0,0,0);
    \coordinate (-dx) at (-\LONGEJENEGX,0,0);
    \coordinate (dx) at (\LONGEJEPOSX,0,0);
    \coordinate (-dy) at (0,-\LONGEJENEGY,0);
    \coordinate (dy) at (0,\LONGEJEPOSY,0);
    \coordinate (-dz) at (0,0,-\LONGEJENEGZ);
    \coordinate (dz) at (0,0,\LONGEJEPOSZ);
    % Recta de acción
    \coordinate (accini) at (\ACCINIX,\ACCINIY,\ACCINIZ);
    \coordinate (accfin) at (\ACCFINX,\ACCFINY,\ACCFINZ);
    % Vector helicidad
    \coordinate (helic) at (\HELICMINX,\HELICMINY,\HELICMINZ);

    % Vector director n
    \coordinate (dirini) at (\DIRINIX,\DIRINIY,\DIRINIZ);
    \coordinate (dirfin) at (\DIRFINX,\DIRFINY,\DIRFINZ);
    
    %\coordinate (accfin) at (FX,FY,FZ);
    % DIBUJO
    % Líneas débiles ejes
    %\draw[ejenegat] (o) -- (-dx);
    %\draw[ejenegat] (o) -- (-dy);
    %\draw[ejenegat] (o) -- (-dz);
    % Ejes
    \draw[ejeposit] (o) -- (dx);
    \draw[ejeposit] (o) -- (dy);
    \draw[ejeposit] (o) -- (dz);
    \node[anchor=north east,name=letrax] at (dx) {$x$};
    \node[anchor=north west,name=letray] at (dy) {$y$};
    \node[anchor=south,name=letraz] at (dz) {$z$};
    
    %  % Giro eje x
    % \tdplotsetthetaplanecoords{-90}
    %  % Notice you have to tell tiks-3dplot you are now in rotated coords
    %  % Since tikz-3dplot swaps the planes in tdplotsetthetaplanecoords,
    %  % the former y axis is now the z axis.
    % \tdplotdrawarc[tdplot_rotated_coords,line width=2pt,
    % -{Latex[length=11pt,width=7pt,flex=1]},color=black!50]
    % {(0,0,1.9)}{0.3}{380}{40}{anchor=south west,color=black}{}
    %  % Reponer parte derecha del eje y
    % \draw[ultra thick,->] (1.9,0,0) -- (\longeje,0,0);
    %% \tdplotdrawarc[tdplot_rotated_coords,line width=2pt,
    %% -{Latex[round,length=6pt,width=4pt]},color=black!50]
    %% {(0,0,1.9)}{0.3}{380}{40}{anchor=south west,color=black}{}
    %%  % Reponer parte derecha del eje y
    %% \draw[ultra thick,->] (1.9,0,0) -- (\longeje,0,0);

    % RECTA DE ACCIÓN
    \draw[directrizback] (accini) -- (o);
    \draw[directrizfront] (o) -- (accfin);

    % VECTOR DIRECTOR
    \draw[ndirector] (dirini) -- (dirfin);
    \node[above left,\colorndirector] at (dirfin) {$\xhat{n}$};

    % VECTOR MOMENTO
    \tdplotsetcoord{P}{\PR}{\PTHETA}{\PPHI}
    % draw a vector from origin to point (P)
    % \draw[-{Stealth[width=7pt]},color=red!80!black,ultra thick] (O) -- (P)
    % node[above right=0pt and 0pt] {\footnotesize $\braket{\vec{S}}$};
    % Momento del electrón en el origen
    % Líneas auxiliares por debajo del momento
    \draw[linaux] (Pxy) -- (Px);
    %\draw[linaux] (O) -- (Pxy);
    
    \draw[helicidadplus] (o) -- (P);
    %\node[right=0pt,\colormomento] at (P) {\footnotesize $\vvv{p}$};
    \node[above left=0pt and -6pt,\colorhelicplustext]
    at (P) {\footnotesize $\ket{n_+}$};
    \draw[helicidadminus] (o) -- (helic);
    \node[above left=0pt and -6pt,\colorhelicminustext]
    at (helic) {\footnotesize $\ket{n_-}$};    
    % Líneas auxiliares por encima del momento
    \draw[linaux] (P) -- (Pxy);

    \draw[linaux] (Pxy) -- (Py);
    %
    \draw[linaux] (Py) -- (Pyz);
    \draw[linaux] (Pyz) -- (Pz);
    \draw[linaux] (Pyz) -- (P);
    \draw[linaux] (Pz) -- (Pxz);
    \draw[linaux] (Pxz) -- (P);
    \draw[linaux] (Pxz) -- (Px);
    \draw[linaux,line width=.8pt,-{Latex}] (o) -- (Pxy);
    
    \tdplotsetthetaplanecoords{\PTHETA}
    
    % draw the angle \phi, and label it
    % syntax:
    % \tdplotdrawarc[coordinate frame, draw options]
    % {center point}{r}{angle}{label options}{label}
    % \tdplotdrawarc{(O)}{0.8}{0}{\phivec}{anchor=north}{\footnotesize $\pi/2$}
    
    % set the rotated coordinate system so the x'-y' plane lies within the
    % "theta plane" of the main coordinate system
    % syntax: \tdplotsetthetaplanecoords{\phi}
    \tdplotsetthetaplanecoords{\PPHI}
    
    % draw theta arc and label, using rotated coordinate system
    % \tdplotdrawarc[tdplot_rotated_coords]
    % {(0,0,0)}{0.8}{0}{\thetavec}{above right=1pt and 1pt}{\footnotesize $\theta$}
    
    % Electrón en el origen
    %\filldraw[electron] (o) circle[radius=2pt];

    %\node at (0,.5,3.0) {\ACCINIX};
    %\node at (0,.5,2.6) {\ACCINIY};
    %\node at (0,.5,2.2) {\ACCINIZ};
    
    % Nube de probabilidad (estado espacial)
    \fill [blue,path fading=fade out] (0,0,0) circle [radius=4pt];
    % node[below right=4pt and 0pt] {$\scriptstyle e^-$};
    % Fondo amarillo
    \begin{scope}[on background layer]
      % \node [line width=1pt, draw=\backgroundbordercolor,
      % fill=\backgroundcolor, fit= (O) (letraejex) (letraejey)] {};
      \node [background, fit= (accini) (accfin) (letrax) (letraz) (letray)] {};
    \end{scope}
  \end{tikzpicture}
  \caption{Se representan, en verde y rojo, los dos posibles estados de spin
    que se pueden medir en la dirección elegida, $\xhat{n}$.}
  \label{fig:spin12-electron-estados-spin-direccion-n}
\end{figure}

Estos nuevos estados \emph{arriba} y \emph{abajo} según la dirección $\xhat{n}$
se pueden escribir como combinaciones lineales de la base $\ket{z_+}$ y
$\ket{z_-}$
\begin{align*}
  \ket{n_+} &= \alpha \ket{z_+} + \beta \ket{z_-}\\
  \ket{n_-} &= \gamma \ket{z_+} + \delta \ket{z_-}
\end{align*}

Estos espinores se pueden deducir de varias formas. Aquí vamos a
obtenerlos rotando los vectores $\ket{z+}$ y $\ket{z-}$.
Primero analizaremos las rotaciones necesarias para obtener
$\ket{n+}$ a partir de $\ket{z+}$.

Según la figura~\ref{fig:spin12-rotacion-z+-to-n+},
partimos de $\ket{z_+}$, lo rotamos un ángulo $\theta$ alrededor del
eje $y$, después rotamos el vector resultante un ángulo $\phi$ alrededor del
eje $z$, obteniendo finalmente $\ket{n_+}$ (para obtener $\ket{n-}$ serían
necesarias las mismas transformaciones a partir de $\ket{z-}$).
\begin{figure}[ht]
  % COLORES
  \newcommand{\colordirectrizfront}{black}
  \newcommand{\colordirectrizback}{black!50}
  \newcommand{\colormomento}{blue!85!black}
  \newcommand{\colorspinorplus}{green!60!black}
  \newcommand{\colorspinorplustext}{green!45!black}
  \newcommand{\colorspinorfantasma}{black!30}
  \newcommand{\colorspinorfantasmatext}{black!50}
  \newcommand{\colorhelicminus}{red!80!black}
  \newcommand{\colorhelicminustext}{red!80!black}
  \newcommand{\colorlinaux}{black!15}
  \newcommand{\colorndirector}{black!60}
  \newcommand{\colorelectron}{blue}
  %
  \def\scl{.84}
  % EJES
  % x
  \pgfmathsetmacro{\LONGEJEPOSX}{2.5}
  \pgfmathsetmacro{\LONGEJENEGX}{2}
  % y
  \pgfmathsetmacro{\LONGEJEPOSY}{2.5}
  \pgfmathsetmacro{\LONGEJENEGY}{1.5}
  % z
  \pgfmathsetmacro{\LONGEJEPOSZ}{2.5}
  \pgfmathsetmacro{\LONGEJENEGZ}{1.5}
  % 
  \pgfmathsetmacro{\PMOD}{2.2}
  \pgfmathsetmacro{\NMOD}{1.4}
  \pgfmathsetmacro{\THETA}{55}
  \pgfmathsetmacro{\PHI}{50}
  % 
  \tdplotsetmaincoords{60}{110}
  % HELICIDAD +
  \pgfmathsetmacro{\PR}{\PMOD}
  \pgfmathsetmacro{\PTHETA}{\THETA}
  \pgfmathsetmacro{\PPHI}{\PHI}
  % HELICIDAD -
  \pgfmathsetmacro{\HELMINR}{\PMOD}
  \pgfmathsetmacro{\HELMINTHETA}{180 - \THETA}
  \pgfmathsetmacro{\HELMINPHI}{\PHI + 180}
  \pgfmathsetmacro{\HELMINX}{\HELMINR*sin(\HELMINTHETA)*cos(\HELMINPHI)}
  \pgfmathsetmacro{\HELMINY}{\HELMINR*sin(\HELMINTHETA)*sin(\HELMINPHI)}
  \pgfmathsetmacro{\HELMINZ}{\HELMINR*cos(\HELMINTHETA)}
  % VECTOR Z+
  \pgfmathsetmacro{\ZPLUSX}{0}
  \pgfmathsetmacro{\ZPLUSY}{0}
  \pgfmathsetmacro{\ZPLUSZ}{\NMOD}
  % VECTOR Z-
  \pgfmathsetmacro{\ZMINUSX}{0}
  \pgfmathsetmacro{\ZMINUSY}{0}
  \pgfmathsetmacro{\ZMINUSZ}{-\NMOD}    
  % VECTOR DIRECTOR INICIO
  \pgfmathsetmacro{\DIRINIR}{4.2}
  \pgfmathsetmacro{\DIRINITHETA}{\THETA}
  \pgfmathsetmacro{\DIRINIPHI}{\PHI}
  \pgfmathsetmacro{\DIRINIX}{\DIRINIR * sin(\DIRINITHETA) * cos(\DIRINIPHI)}
  \pgfmathsetmacro{\DIRINIY}{\DIRINIR * sin(\DIRINITHETA) * sin(\DIRINIPHI)}
  \pgfmathsetmacro{\DIRINIZ}{\DIRINIR * cos(\DIRINITHETA)}
  % VECTOR DIRECTOR FIN
  \pgfmathsetmacro{\DIRFINR}{5.5}
  \pgfmathsetmacro{\DIRFINTHETA}{\THETA}
  \pgfmathsetmacro{\DIRFINPHI}{\PHI}
  \pgfmathsetmacro{\DIRFINX}{\DIRFINR * sin(\DIRFINTHETA) * cos(\DIRFINPHI)}
  \pgfmathsetmacro{\DIRFINY}{\DIRFINR * sin(\DIRFINTHETA) * sin(\DIRFINPHI)}
  \pgfmathsetmacro{\DIRFINZ}{\DIRFINR * cos(\DIRFINTHETA)}
  % RECTA DE ACCIÓN INICIO
  \pgfmathsetmacro{\ACCINIR}{3.5}
  \pgfmathsetmacro{\ACCINITHETA}{180 - \THETA}
  \pgfmathsetmacro{\ACCINIPHI}{\PHI + 180}
  \pgfmathsetmacro{\ACCINIX}{\ACCINIR * sin(\ACCINITHETA) * cos(\ACCINIPHI)}
  \pgfmathsetmacro{\ACCINIY}{\ACCINIR * sin(\ACCINITHETA) * sin(\ACCINIPHI)}
  \pgfmathsetmacro{\ACCINIZ}{\ACCINIR * cos(\ACCINITHETA)}
  % RECTA DE ACCIÓN FIN
  \pgfmathsetmacro{\ACCFINR}{7.0}
  \pgfmathsetmacro{\ACCFINTHETA}{\THETA}
  \pgfmathsetmacro{\ACCFINPHI}{\PHI}
  \pgfmathsetmacro{\ACCFINX}{\ACCFINR * sin(\ACCFINTHETA) * cos(\ACCFINPHI)}
  \pgfmathsetmacro{\ACCFINY}{\ACCFINR * sin(\ACCFINTHETA) * sin(\ACCFINPHI)}
  \pgfmathsetmacro{\ACCFINZ}{\ACCFINR * cos(\ACCFINTHETA)}
  % 
  \tikzfading[name=fade out, inner color=transparent!0,
  outer color=transparent!100]
  \begin{minipage}{.32\linewidth}
    %\hspace*{2.3em}
    \begin{tikzpicture}[%
      scale=\scl,
      tdplot_main_coords,
      every node/.style={font=\small},
      linaux/.style={ultra thin, color=\colorlinaux},
      directrizfront/.style={dashed,ultra thin,\colordirectrizfront},
      directrizback/.style={dashed,ultra thin,\colordirectrizback},
      ndirector/.style={-{Latex[round]},line width=0.8pt,\colorndirector},
      ejeposit/.style={thick,->},
      ejenegat/.style={ultra thin,black!20},
      electron/.style={fill=\colorelectron, draw=black},
      momento/.style={%
        % -{Latex[width=5pt,length=7pt]},color=blue!90!black,line width=1.1pt
        -{Latex[round]},color=\colormomento,line width=1.1pt
      },
      spinorplus/.style={%
        % -{Latex[width=5pt,length=7pt]},color=blue!90!black,line width=1.1pt
        -{Latex[round]},color=\colorspinorplus,line width=1.1pt
      },
      spinorminus/.style={%
        % -{Latex[width=5pt,length=7pt]},color=blue!90!black,line width=1.1pt
        -{Latex[round]},color=\colorspinorminus,line width=1.1pt
      },
      spinorfantasma/.style={%
        % -{Latex[width=5pt,length=7pt]},color=blue!90!black,line width=1.1pt
        -{Latex[round]},color=\colorspinorfantasma,line width=1.1pt
      },
      background/.style={
        line width=\bgborderwidth,
        draw=\bgbordercolor,
        fill=\bgcolor,
      },
      ]
      % COORDENADAS
      \coordinate (o) at (0,0,0);
      \coordinate (-dx) at (-\LONGEJENEGX,0,0);
      \coordinate (dx) at (\LONGEJEPOSX,0,0);
      \coordinate (-dy) at (0,-\LONGEJENEGY,0);
      \coordinate (dy) at (0,\LONGEJEPOSY,0);
      \coordinate (-dz) at (0,0,-\LONGEJENEGZ);
      \coordinate (dz) at (0,0,\LONGEJEPOSZ);
      % Vector z+
      \coordinate (z+) at (\ZPLUSX,\ZPLUSY,\ZPLUSZ);
      % Vector z-
      \coordinate (z-) at (\ZMINUSX,\ZMINUSY,\ZMINUSZ);
      % Recta de acción
      \coordinate (accini) at (\ACCINIX,\ACCINIY,\ACCINIZ);
      \coordinate (accfin) at (\ACCFINX,\ACCFINY,\ACCFINZ);
      % Vector helicidad
      \coordinate (helic) at (\HELMINX,\HELMINY,\HELMINZ);
      % Vector director n
      \coordinate (dirini) at (\DIRINIX,\DIRINIY,\DIRINIZ);
      \coordinate (dirfin) at (\DIRFINX,\DIRFINY,\DIRFINZ);
      
      % DIBUJO
      % Líneas débiles ejes
      % \draw[ejenegat] (o) -- (-dx);
      % \draw[ejenegat] (o) -- (-dy);
      % \draw[ejenegat] (o) -- (-dz);
      % Ejes
      \draw[ejeposit] (o) -- (dx);
      \draw[ejeposit] (o) -- (dy);
      \draw[ejeposit] (o) -- (dz);
      \node[anchor=north east,name=letrax] at (dx) {$x$};
      \node[anchor=north west,name=letray] at (dy) {$y$};
      \node[anchor=south,name=letraz] at (dz) {$z$};
      
      % Giro eje y
      \tdplotsetthetaplanecoords{0}
      % Notice you have to tell tiks-3dplot you are now in rotated coords
      % Since tikz-3dplot swaps the planes in tdplotsetthetaplanecoords,
      % the former y axis is now the z axis.
      \tdplotdrawarc[tdplot_rotated_coords,line width=1pt,
      {Latex[round,length=6pt,width=4pt]}-,color=black!50]
      {(0,0,2.0)}{0.3}{440}{100}{anchor=south west,color=black}{}
      % Reponer parte derecha del eje y
      \draw[->] (0,2.0,0) -- (dy);
      %\filldraw[fill=red,draw=black] (0,2.0,0) circle[radius=.5pt];
      \node[above right=7pt and -4pt]
      at (0,2.0,0) {\footnotesize $\theta\,\si{\radian}$};
      
      % RECTA DE ACCIÓN
      % \draw[directrizback] (accini) -- (o);
      % \draw[directrizfront] (o) -- (accfin);
      % VECTOR DIRECTOR
      % \draw[ndirector] (dirini) -- (dirfin);
      % \node[above left,\colorndirector] at (dirfin) {$\xhat{n}$};
      % VECTOR z+
      \draw[spinorplus] (o) -- (z+);
      \node[right,\colorspinorplustext]
      at (z+) {\footnotesize $\ket{z_+}$};
      % VECTOR z+ ROTADO \theta
      % \tdplotsetcoord{P}{\PR}{\PTHETA}{0}
      % \draw[spinorplus] (o) -- (P);
      % \node[right=0pt,\colormomento] at (P) {\footnotesize $\vvv{p}$};
      % \node[above left=0pt and -6pt,\colorhelicplustext]
      % at (P) {\footnotesize $\ket{n_+}$};
      % VECTOR HELICIDAD -
      % \draw[helicidadminus] (o) -- (helic);
      % \node[above left=0pt and -6pt,\colorhelicminustext]
      % at (helic) {\footnotesize $\ket{n_-}$};
      % Nube de probabilidad (estado espacial)
      \fill [blue,path fading=fade out] (0,0,0) circle [radius=5pt];
      %node[below right=4pt and 0pt] {$\scriptstyle e^-$};
      \begin{scope}[on background layer]
        % \node [line width=1pt, draw=\backgroundbordercolor,
        % fill=\backgroundcolor, fit= (O) (letraejex) (letraejey)] {};
        \node [background, fit= (accini)(accfin)(letrax)(letraz)(letray)] {};
      \end{scope}
    \end{tikzpicture}
  \end{minipage}
  % \hspace*{.5em}
  \begin{minipage}{.32\linewidth}
    %\hspace*{.5em}
    \begin{tikzpicture}[%
      scale=\scl,
      tdplot_main_coords,
      every node/.style={font=\small},
      linaux/.style={ultra thin, color=\colorlinaux},
      directrizfront/.style={dashed,ultra thin,\colordirectrizfront},
      directrizback/.style={dashed,ultra thin,\colordirectrizback},
      ndirector/.style={-{Latex[round]},line width=0.8pt,\colorndirector},
      ejeposit/.style={thick,->},
      ejenegat/.style={ultra thin,black!20},
      electron/.style={fill=\colorelectron, draw=black},
      momento/.style={%
        % -{Latex[width=5pt,length=7pt]},color=blue!90!black,line width=1.1pt
        -{Latex[round]},color=\colormomento,line width=1.1pt
      },
      spinorplus/.style={%
        % -{Latex[width=5pt,length=7pt]},color=blue!90!black,line width=1.1pt
        -{Latex[round]},color=\colorspinorplus,line width=1.1pt
      },
      spinorminus/.style={%
        % -{Latex[width=5pt,length=7pt]},color=blue!90!black,line width=1.1pt
        -{Latex[round]},color=\colorspinorminus,line width=1.1pt
      },
      spinorfantasma/.style={%
        % -{Latex[width=5pt,length=7pt]},color=blue!90!black,line width=1.1pt
        -{Latex[round]},color=\colorspinorfantasma,line width=1.1pt
      },
      background/.style={
        line width=\bgborderwidth,
        draw=\bgbordercolor,
        fill=\bgcolor,
      },
      ]
      % COORDENADAS
      \coordinate (o) at (0,0,0);
      \coordinate (-dx) at (-\LONGEJENEGX,0,0);
      \coordinate (dx) at (\LONGEJEPOSX,0,0);
      \coordinate (-dy) at (0,-\LONGEJENEGY,0);
      \coordinate (dy) at (0,\LONGEJEPOSY,0);
      \coordinate (-dz) at (0,0,-\LONGEJENEGZ);
      \coordinate (dz) at (0,0,\LONGEJEPOSZ);
      % Vector z+
      \coordinate (z+) at (\ZPLUSX,\ZPLUSY,\ZPLUSZ);
      % Vector z-
      \coordinate (z-) at (\ZMINUSX,\ZMINUSY,\ZMINUSZ);
      % Recta de acción
      \coordinate (accini) at (\ACCINIX,\ACCINIY,\ACCINIZ);
      \coordinate (accfin) at (\ACCFINX,\ACCFINY,\ACCFINZ);
      % Vector helicidad
      \coordinate (helic) at (\HELMINX,\HELMINY,\HELMINZ);
      % Vector director n
      \coordinate (dirini) at (\DIRINIX,\DIRINIY,\DIRINIZ);
      \coordinate (dirfin) at (\DIRFINX,\DIRFINY,\DIRFINZ);
      
      % DIBUJO
      % Líneas débiles ejes
      % \draw[ejenegat] (o) -- (-dx);
      % \draw[ejenegat] (o) -- (-dy);
      % \draw[ejenegat] (o) -- (-dz);
      % Ejes
      \draw[ejeposit] (o) -- (dx);
      \draw[ejeposit] (o) -- (dy);
      \draw[ejeposit] (o) -- (dz);
      \node[anchor=north east,name=letrax] at (dx) {$x$};
      \node[anchor=north west,name=letray] at (dy) {$y$};
      \node[anchor=south,name=letraz] at (dz) {$z$};
      % Giro eje z
      % BUG: 'Missing character: There is no ;' warning in 'gruposlie.log'
      \tdplotdrawarc[line width=1pt,{Latex[round,length=6pt,width=4pt]}-,
      color=black!50] {(0,0,2.0)}{0.3}{40}{-290}{}{};
      % Reponer parte superior del eje z
      \draw[->] (0,0,2.0) -- (dz);
      \node[above right=-2pt and 6pt]
      at (0,0,2.0) {\footnotesize $\phi\,\si{\radian}$};
      %\filldraw[fill=red,draw=black] (0,0,2.0) circle[radius=1pt];
      

      \tdplotsetthetaplanecoords{0}
      % draw theta arc and label, using rotated coordinate system
      \tdplotdrawarc[tdplot_rotated_coords,black!50]
      {(0,0,0)}{.7}{0}{\PTHETA}{above left=-5pt and -2pt}
      {\footnotesize $\color{black!50}{\theta}$}

      
      % RECTA DE ACCIÓN
      % \draw[directrizback] (accini) -- (o);
      % \draw[directrizfront] (o) -- (accfin);
      % VECTOR DIRECTOR
      % \draw[ndirector] (dirini) -- (dirfin);
      % \node[above left,\colorndirector] at (dirfin) {$\xhat{n}$};
      % VECTOR z+ ROTADO \theta
      \tdplotsetcoord{P}{\PR}{\PTHETA}{0}
      \draw[spinorplus] (o) -- (P);
      \node[left] at (P) {\footnotesize $\ket{u}$};
      % \node[right=0pt,\colormomento] at (P) {\footnotesize $\vvv{p}$};
      % \node[above left=0pt and -6pt,\colorhelicplustext]
      % at (P) {\footnotesize $\ket{n_+}$};
      % VECTOR HELICIDAD -
      % \draw[helicidadminus] (o) -- (helic);
      % \node[above left=0pt and -6pt,\colorhelicminustext]
      % at (helic) {\footnotesize $\ket{n_-}$};
      \draw[linaux] (P) -- (Px);
      \draw[linaux] (P) -- (Pz);
      % Reponer eje x sin flecha
      \draw (o) -- (dx);
      %\filldraw[fill=black,draw=black] (Px) circle[radius=.3pt];
      %
      % VECTOR z+ original
      \draw[spinorfantasma] (o) -- (z+);
      \node[right,\colorspinorfantasmatext] at (z+) {$\ket{z_+}$};

      % Nube de probabilidad (estado espacial)
      \fill [blue,path fading=fade out] (0,0,0) circle [radius=5pt];
      %node[below right=4pt and 0pt] {$\scriptstyle e^-$};
      \begin{scope}[on background layer]
        % \node [line width=1pt, draw=\backgroundbordercolor,
        % fill=\backgroundcolor, fit= (O) (letraejex) (letraejey)] {};
        \node [background, fit= (accini)(accfin)(letrax)(letraz)(letray)] {};
      \end{scope}
    \end{tikzpicture}
  \end{minipage}
  % \hspace{.5em}
  \begin{minipage}{.32\linewidth}
    %\hspace*{.5em}
    \begin{tikzpicture}[%
      scale=\scl,
      tdplot_main_coords,
      every node/.style={font=\small},
      linaux/.style={ultra thin, color=\colorlinaux},
      directrizfront/.style={dashed,ultra thin,\colordirectrizfront},
      directrizback/.style={dashed,ultra thin,\colordirectrizback},
      ndirector/.style={-{Latex[round]},line width=0.8pt,\colorndirector},
      ejeposit/.style={thick,->},
      ejenegat/.style={ultra thin,black!20},
      electron/.style={fill=\colorelectron, draw=black},
      momento/.style={%
        % -{Latex[width=5pt,length=7pt]},color=blue!90!black,line width=1.1pt
        -{Latex[round]},color=\colormomento,line width=1.1pt
      },
      spinorplus/.style={%
        % -{Latex[width=5pt,length=7pt]},color=blue!90!black,line width=1.1pt
        -{Latex[round]},color=\colorspinorplus,line width=1.1pt
      },
      spinorminus/.style={%
        % -{Latex[width=5pt,length=7pt]},color=blue!90!black,line width=1.1pt
        -{Latex[round]},color=\colorspinorminus,line width=1.1pt
      },
      spinorfantasma/.style={%
        % -{Latex[width=5pt,length=7pt]},color=blue!90!black,line width=1.1pt
        -{Latex[round]},color=\colorspinorfantasma,line width=1.1pt
      },
      background/.style={
        line width=\bgborderwidth,
        draw=\bgbordercolor,
        fill=\bgcolor,
      },
      ]
      % COORDENADAS
      \coordinate (o) at (0,0,0);
      \coordinate (-dx) at (-\LONGEJENEGX,0,0);
      \coordinate (dx) at (\LONGEJEPOSX,0,0);
      \coordinate (-dy) at (0,-\LONGEJENEGY,0);
      \coordinate (dy) at (0,\LONGEJEPOSY,0);
      \coordinate (-dz) at (0,0,-\LONGEJENEGZ);
      \coordinate (dz) at (0,0,\LONGEJEPOSZ);
      % Vector z+
      \coordinate (z+) at (\ZPLUSX,\ZPLUSY,\ZPLUSZ);
      % Vector z-
      \coordinate (z-) at (\ZMINUSX,\ZMINUSY,\ZMINUSZ);
      % Recta de acción
      \coordinate (accini) at (\ACCINIX,\ACCINIY,\ACCINIZ);
      \coordinate (accfin) at (\ACCFINX,\ACCFINY,\ACCFINZ);
      % Vector helicidad
      \coordinate (helic) at (\HELMINX,\HELMINY,\HELMINZ);
      % Vector director n
      \coordinate (dirini) at (\DIRINIX,\DIRINIY,\DIRINIZ);
      \coordinate (dirfin) at (\DIRFINX,\DIRFINY,\DIRFINZ);
      % DIBUJO
      % Líneas débiles ejes
      % \draw[ejenegat] (o) -- (-dx);
      % \draw[ejenegat] (o) -- (-dy);
      % \draw[ejenegat] (o) -- (-dz);
      % Ejes
      \draw[ejeposit] (o) -- (dx);
      \draw[ejeposit] (o) -- (dy);
      \draw[ejeposit] (o) -- (dz);
      \node[anchor=north east,name=letrax] at (dx) {$x$};
      \node[anchor=north west,name=letray] at (dy) {$y$};
      \node[anchor=south,name=letraz] at (dz) {$z$};
      % RECTA DE ACCIÓN
      % \draw[directrizback] (accini) -- (o);
      % \draw[directrizfront] (o) -- (accfin);
      % VECTOR DIRECTOR
      % \draw[ndirector] (dirini) -- (dirfin);
      % \node[above left,\colorndirector] at (dirfin) {$\xhat{n}$};

      % Líneas auxiliares
      \draw[linaux] (P) -- (Px);
      \draw[linaux] (P) -- (Pz);
      
      % VECTOR HELICIDAD +
      \tdplotsetcoord{P}{\PR}{\PTHETA}{\PPHI}
      \draw[spinorplus] (o) -- (P);
      % \node[right=0pt,\colormomento] at (P) {\footnotesize $\vvv{p}$};
      \node[above right=-4pt and -2pt,\colorspinorplustext]
      at (P) {\footnotesize $\ket{n_+}$};

      \tdplotsetthetaplanecoords{\PPHI}
      % draw theta arc and label, using rotated coordinate system
      \tdplotdrawarc[tdplot_rotated_coords]
      {(0,0,0)}{.8}{0}{\PTHETA}{above right=-6pt and 0pt}
      {\footnotesize $\theta$}

      % Líneas auxiliares X+
      \draw[linaux] (P) -- (Pxy);
      \draw[linaux] (Pxy) -- (Px);
      \draw[linaux] (Pxy) -- (Py);
      \draw[line width=.7pt,green!27.5,-{Latex}] (o) -- (Pxy);
      % VECTOR z+ ROTADO \theta
      \tdplotsetcoord{P}{\PR}{\PTHETA}{0}
      % draw theta arc and label, using rotated coordinate system
      \tdplotdrawarc
      {(0,0,0)}{0.6}{0}{\PPHI}{below=-2pt}
      {\footnotesize $\phi$}

      \draw[spinorfantasma] (o) -- (P);
      \node[left, \colorspinorfantasma] at (P) {\footnotesize $\ket{u}$};
      % VECTOR HELICIDAD -
      % \draw[helicidadminus] (o) -- (helic);
      % \node[above left=0pt and -6pt,\colorhelicminustext]
      % at (helic) {\footnotesize $\ket{n_-}$};
      % Nube de probabilidad (estado espacial)
      \fill [blue,path fading=fade out] (0,0,0) circle [radius=5pt];
      %node[below right=4pt and 0pt] {$\scriptstyle e^-$};
      \begin{scope}[on background layer]
        % \node [line width=1pt, draw=\backgroundbordercolor,
        % fill=\backgroundcolor, fit= (O) (letraejex) (letraejey)] {};
        \node [background, fit= (accini)(accfin)(letrax)(letraz)(letray)] {};
      \end{scope}
    \end{tikzpicture}
  \end{minipage}    
  \caption{Obtención del vector de estado $\ket{n_+}$, \emph{hacia arriba} en
    una dirección arbitraria $\xhat{n}$, de coordenadas polares
    $(1,\theta,\phi)$, a partir del vector de estado de spin
    \emph{hacia arriba} en el eje $z$, mediante dos rotaciones consecutivas.}
  \label{fig:spin12-rotacion-z+-to-n+}
\end{figure}

Esta transformación se puede representar matemáticamente como
\begin{align}
  \ket{n_+} &= \mmm{R_2}(\phi, n_z) \mmm{R_1}(\theta, n_y) \ket{z_+}
              = \mmm{R_2} \mmm{R_1} \ket{z_+}
              = \mmm{R} \ket{z_+}\\
  \ket{n_-} &= \mmm{R_2}(\phi, n_z) \mmm{R_1}(\theta, n_y) \ket{z_-}
              = \mmm{R_2} \mmm{R_1} \ket{z_-}
              = \mmm{R} \ket{z_-}
\end{align}
Nótese que hemos combinado las dos rotaciones de la figura en una $\mmm{R}$.

Seguiremos los siguientes pasos, que implican los cálculos:

{\parindent=0pt
  1) Matriz de rotación $\mmm{R_1}$}

{\parindent=0pt
  2) Matriz de rotación $\mmm{R_2}$}

{\parindent=0pt
3) Matriz de rotación combinada $\mmm{R} = \mmm{R_2}\mmm{R_1}$}

{\parindent=0pt
4) Cálculo de los vectores de estado $\ket{n_+} = \mmm{R}\ket{z_+}$
y $\ket{n_-} = \mmm{R}\ket{z_-}$}

\begin{description}
\item[1) ] Cálculo de la matriz de rotación $\mmm{R_1}$
  \begin{align*}
    \mmm{R_1}
    &=
      e^{-i\frac{\theta}{2}\,\mmm{\sigma_y}}
      = \mmm{I}\cos\frac{\theta}{2} - i \mmm{\sigma_y}\sin\frac{\theta}{2}
      = \begin{pNiceMatrix}
        1 & 0\\
        0 & 1
      \end{pNiceMatrix}
      \cos\frac{\theta}{2}
      - i \begin{pNiceMatrix}
        0 & -i\\
        i & 0
      \end{pNiceMatrix}
      \sin\frac{\theta}{2}\\
    &=
      \begin{pNiceMatrix}
        \cos\frac{\theta}{2} & 0\\
        0 & \cos\frac{\theta}{2}
      \end{pNiceMatrix}
      - \begin{pNiceMatrix}
        0 & \sin\frac{\theta}{2}\\
        -\sin\frac{\theta}{2} & 0
      \end{pNiceMatrix}
    = \begin{pNiceMatrix}
        \cos\frac{\theta}{2} & -\sin\frac{\theta}{2}\\[4pt]
        \sin\frac{\theta}{2} & \cos\frac{\theta}{2}
      \end{pNiceMatrix}
  \end{align*}
  
\item[2) ] Cálculo de la matriz de rotación $\mmm{R_2}$
  \begin{align*}
    \mmm{R_2}
    &=
      e^{-i\frac{\phi}{2}\,\mmm{\sigma_z}}
      = \mmm{I}\cos\frac{\phi}{2} - i \mmm{\sigma_z}\sin\frac{\phi}{2}
      = \begin{pNiceMatrix}
        1 & 0\\
        0 & 1
      \end{pNiceMatrix}
      \cos\frac{\phi}{2}
      - i \begin{pNiceMatrix}
        1 & 0\\
        0 & -1
      \end{pNiceMatrix}
      \sin\frac{\phi}{2}\\
    &=
      \begin{pNiceMatrix}
        \cos\frac{\phi}{2} & 0\\
        0 & \cos\frac{\phi}{2}
      \end{pNiceMatrix}
      - i\begin{pNiceMatrix}
        \sin\frac{\phi}{2} & 0\\
        0 & -\sin\frac{\phi}{2}
      \end{pNiceMatrix}\\
    &=
      \begin{pNiceMatrix}
        \cos\frac{\phi}{2}-i\sin\frac{\phi}{2} & 0\\
        0 & \cos\frac{\phi}{2}+i\sin\frac{\phi}{2}
      \end{pNiceMatrix}
      = \begin{pNiceMatrix}
        e^{-i\frac{\phi}{2}} & 0\\
        0 & e^{i\frac{\phi}{2}}
      \end{pNiceMatrix}\\
      &= e^{-i\frac{\phi}{2}}
      \begin{pNiceMatrix}
        1 & 0\\
        0 & e^{i\phi}
      \end{pNiceMatrix}
  \end{align*}

\item[3) ] Cálculo de la matriz de rotación combinada
  $\mmm{R} = \mmm{R_2}\mmm{R_1}$
  \begin{align*}
    \mmm{R}
    &=
      \mmm{R_2}\mmm{R_1}
      = e^{-i\frac{\phi}{2}}
      \begin{pNiceMatrix}
        1 & 0\\
        0 & e^{i\phi}
      \end{pNiceMatrix}
      \begin{pNiceMatrix}
        \cos\frac{\theta}{2} & -\sin\frac{\theta}{2}\\[4pt]
        \sin\frac{\theta}{2} & \cos\frac{\theta}{2}
      \end{pNiceMatrix}\\
    &=
      e^{-i\frac{\phi}{2}}
      \begin{pNiceMatrix}      
        \cos\frac{\theta}{2} & -\sin\frac{\theta}{2}\\[4pt]
        e^{i\phi} \sin\frac{\theta}{2} & e^{i\phi} \cos\frac{\theta}{2}
      \end{pNiceMatrix}
  \end{align*}

%  El factor $e^{-i\frac{\phi}{2}}$ del operador de rotación se puede eliminar,
%  porque solo añade una fase global a los espinores que se obtienen de esta
%  transformación, pero no afecta a las probabilidades que se deducen del
%  espinor. En otras palabras, modifica las amplitudes de probabilidad de
%  los espinores pero no las probabilidades. 

\item[4) ] Cálculo de los estados posibles de una medida del spin
  en la dirección $\xhat{n}$.

  \begin{itemize}
  \item Cálculo de $\ket{n_+}$
  \begin{align*}
    \ket{n_+}
    &=
      R \ket{z_+}
      = e^{-i\frac{\phi}{2}}
      \begin{pNiceMatrix}
        \cos\frac{\theta}{2} & -\sin\frac{\theta}{2}\\[4pt]
        e^{i\phi} \sin\frac{\theta}{2} & e^{i\phi} \cos\frac{\theta}{2}
      \end{pNiceMatrix}
      \begin{pNiceMatrix}
        1\\
        0
      \end{pNiceMatrix}
    = e^{-i\frac{\phi}{2}}
    \begin{pNiceMatrix}
      \cos\frac{\theta}{2}\\
      e^{i\phi} \sin\frac{\theta}{2}
    \end{pNiceMatrix}
  \end{align*}
  
  El factor $e^{-i\frac{\phi}{2}}$ de un espinor es prescindible porque solo añade
  una fase global a las amplitudes de probabilidad, pero no afecta a las
  probabilidades que se deducen de estas.
  Resulta que en mecánica cuántica, si dos vectores de estado $\ket{\Psi}$
  y $\ket{\Psi'}$ se diferencian únicamente en una fase global
  \[
    \ket{\Psi'} = e^{i\alpha} \ket{\Psi}
  \]
  entonces conducen a la misma física, porque de ellas se deducen las mismas
  probabilidades.

  Así, $\ket{n_+}$ lo podemos dejar como
  \begin{align*}
    \ket{n_+}
    &=
      \begin{pNiceMatrix}
        \cos\frac{\theta}{2}\\
        e^{i\phi} \sin\frac{\theta}{2}
      \end{pNiceMatrix}
    = \cos\frac{\theta}{2}
    \begin{pNiceMatrix}
      1\\
      0
    \end{pNiceMatrix}
    + e^{i\phi} \sin\frac{\theta}{2}
    \begin{pNiceMatrix}
      0\\
      1
    \end{pNiceMatrix}\\
    &=
      \cos\frac{\theta}{2} \ket{z_+} + e^{i\phi} \sin\frac{\theta}{2} \ket{z_-}
  \end{align*}

  \item Cálculo de $\ket{n_-}$
  \begin{align*}
    \ket{n_-}
    &=
      R \ket{z_-}
      = e^{-i\frac{\phi}{2}}
      \begin{pNiceMatrix}
        \cos\frac{\theta}{2} & -\sin\frac{\theta}{2}\\
        e^{i\phi} \sin\frac{\theta}{2} & e^{i\phi} \cos\frac{\theta}{2}
      \end{pNiceMatrix}
      \begin{pNiceMatrix}
        0\\
        1
      \end{pNiceMatrix}
      = e^{-i\frac{\phi}{2}}
      \begin{pNiceMatrix}
        -\sin\frac{\theta}{2}\\
        e^{i\phi} \cos\frac{\theta}{2}
      \end{pNiceMatrix}\\
    &=
      e^{-i\frac{\phi}{2}}
      e^{i\phi}
      \begin{pNiceMatrix}
        e^{-i\phi}\sin\frac{\theta}{2}\\
        \cos\frac{\theta}{2}
      \end{pNiceMatrix}
    = e^{i\frac{\phi}{2}}
      \begin{pNiceMatrix}
        e^{-i\phi}\sin\frac{\theta}{2}\\
        \cos\frac{\theta}{2}
      \end{pNiceMatrix}    
  \end{align*}
  Se ha sacado el factor $e^{i\phi}$ en el penúltimo paso para que el
  resultado coincidiera con el que se presentó en las lecciones de vídeo.
  
  Como antes, prescindimos de la fase global del espinor
  \begin{align*}
    \ket{n_-}
    &=
      \begin{pNiceMatrix}
        -e^{-i\phi}\sin\frac{\theta}{2}\\
        \cos\frac{\theta}{2}
      \end{pNiceMatrix}
    = -e^{-i\phi}\sin\frac{\theta}{2}
    \begin{pNiceMatrix}
      1\\
      0
    \end{pNiceMatrix}
    + \cos\frac{\theta}{2}
    \begin{pNiceMatrix}
      0\\
      1
    \end{pNiceMatrix}\\
    &=
      -e^{-i\phi}\sin\frac{\theta}{2}
      \ket{z_+}
      + \cos\frac{\theta}{2}
      \ket{z_-}
  \end{align*}
\end{itemize}
%  Aunque este resultado es correcto, no coincide con el que se presentó
%  en las lecciones de vídeo. Sacamos el factor $e^{i\phi}$ del espinor
%  \[
%    \ket{n_-}
%    =
%    e^{i\phi}
%    \begin{pNiceMatrix}
%      -e^{-i\phi}\sin\frac{\theta}{2}\\
%      \cos\frac{\theta}{2}
%    \end{pNiceMatrix}
%  \]
%  \begin{align*}
%    &=
%      \begin{pNiceMatrix}
%        -e^{-i\phi}\sin\frac{\theta}{2}\\
%      \cos\frac{\theta}{2}
%    \end{pNiceMatrix}
%    = -\sin\frac{\theta}{2}
%    \begin{pNiceMatrix}
%      1\\
%      0
%    \end{pNiceMatrix}
%    + e^{i\phi} \cos\frac{\theta}{2}
%    \begin{pNiceMatrix}
%      0\\
%      1
%    \end{pNiceMatrix}\\
%    &=
%      -\sin\frac{\theta}{2} \ket{z_+} + e^{i\phi} \cos\frac{\theta}{2} \ket{z_-}
%  \end{align*}
\end{description}

Cabe destacar que los posibles estados de spin en una dirección arbitraria
$\xhat{n}$ son ortogonales, lo que quiere decir que son excluyentes entre sí
\begin{align*}
  \braket{n_+|n_-}
  &=
    \left(
    \cos\frac{\theta}{2} \ket{z_+} + e^{i\phi} \sin\frac{\theta}{2}\ket{z_-}
    \right)^*
    \left(
    -e^{-i\phi}\sin\frac{\theta}{2} \ket{z_+} + \cos\frac{\theta}{2}\ket{z_-}
    \right)\\
  &=
    \left(
    \cos\frac{\theta}{2} \bra{z_+} + e^{-i\phi} \sin\frac{\theta}{2}\bra{z_-}
    \right)
    \left(
    -e^{-i\phi} \sin\frac{\theta}{2} \ket{z_+} + \cos\frac{\theta}{2}\ket{z_-}
    \right)\\
  &=
    -e^{-i\phi} \sin\frac{\theta}{2}\cos\frac{\theta}{2} \braket{z_+|z_+}
    + e^{-i\phi} \sin\frac{\theta}{2}\cos\frac{\theta}{2} \braket{z_-|z_-}\\
  &=
    -\frac{e^{-i\phi}}{2} \sin\theta + \frac{e^{-i\phi}}{2} \sin\theta
    = 0
\end{align*}
donde se ha tenido en cuenta que los vectores de estado $\ket{z_+}$ y
$\ket{z_-}$ forman una base ortonormal.

También se puede demostrar que están normalizados. Lo comprobaremos para
$\ket{n_+}$
\begin{align*}
  \braket{n_+|n_+}
  &=
    \left(
    \cos\frac{\theta}{2} \ket{z_+} + e^{i\phi} \sin\frac{\theta}{2}\ket{z_-}
    \right)^*
    \left(
    \cos\frac{\theta}{2} \ket{z_+} + e^{i\phi} \sin\frac{\theta}{2}\ket{z_-}
    \right)\\
  &=
    \left(
    \cos\frac{\theta}{2} \bra{z_+} + e^{-i\phi} \sin\frac{\theta}{2}\bra{z_-}
    \right)
    \left(
    \cos\frac{\theta}{2} \ket{z_+} + e^{i\phi} \sin\frac{\theta}{2}\ket{z_-}
    \right)\\
  &=
    \cos^2\frac{\theta}{2} + \sin^2\frac{\theta}{2}
    = 1
\end{align*}
Así que $\ket{n_+}$ y $\ket{n_-}$ también forman una base ortonormal de
$\symbb{C}^2$.

\subsection{Concepto de helicidad}
En la figura~\ref{fig:spin12-electron-trimomento}
se observa un electrón desplazándose en línea recta con un
trimomento\footnotemark{} constante $\vvv{p}$, encontrándose en un cierto
instante en el origen de coordenadas
\footnotetext{Donde $\gamma = \dfrac{1}{\sqrt{1-(v/c)^2}}$ es el factor de
  Lorentez.
  Para una partícula no relativista, $\gamma$ sería aproximadamente la unidad
  y el momento sería prácticamente el momento clásico $\vvv{p} = m\vvv{v}$.}
\[
  \vvv{p} = \gamma m \vvv{v}
\]
La dirección del movimiento viene dada por el vector unitario
\[
  \xhat{n} = \frac{\vvv{p}}{|\vvv{p}|}
\]

\begin{figure}[ht]
  \centering
  % COLORES
  \newcommand{\colordirectrizfront}{blue}
  \newcommand{\colordirectrizback}{blue!50}
  \newcommand{\colormomento}{blue!85!black}
  \newcommand{\colorlinaux}{blue!20}
  \newcommand{\colorndirector}{black!60}
  \newcommand{\colorelectron}{blue}
  %
  \def\scl{1.2}
  % EJES
  % x
  \pgfmathsetmacro{\LONGEJEPOSX}{3}
  \pgfmathsetmacro{\LONGEJENEGX}{3}
  % y
  \pgfmathsetmacro{\LONGEJEPOSY}{3}
  \pgfmathsetmacro{\LONGEJENEGY}{1.5}
  % z
  \pgfmathsetmacro{\LONGEJEPOSZ}{2}
  \pgfmathsetmacro{\LONGEJENEGZ}{1.5}

  % 
  \pgfmathsetmacro{\PMOD}{2.2}
  \pgfmathsetmacro{\NMOD}{.5}
  \pgfmathsetmacro{\THETA}{55}
  \pgfmathsetmacro{\PHI}{50}
  % 
  \tdplotsetmaincoords{60}{110}
  % MOMENTO LINEAL
  \pgfmathsetmacro{\PR}{\PMOD}
  \pgfmathsetmacro{\PTHETA}{\THETA}
  \pgfmathsetmacro{\PPHI}{\PHI}
  % VECTOR DIRECTOR INICIO
  \pgfmathsetmacro{\DIRINIR}{4.2}
  \pgfmathsetmacro{\DIRINITHETA}{\THETA}
  \pgfmathsetmacro{\DIRINIPHI}{\PHI}
  \pgfmathsetmacro{\DIRINIX}{\DIRINIR * sin(\DIRINITHETA) * cos(\DIRINIPHI)}
  \pgfmathsetmacro{\DIRINIY}{\DIRINIR * sin(\DIRINITHETA) * sin(\DIRINIPHI)}
  \pgfmathsetmacro{\DIRINIZ}{\DIRINIR * cos(\DIRINITHETA)}
  % VECTOR DIRECTOR FIN
  \pgfmathsetmacro{\DIRFINR}{5.5}
  \pgfmathsetmacro{\DIRFINTHETA}{\THETA}
  \pgfmathsetmacro{\DIRFINPHI}{\PHI}
  \pgfmathsetmacro{\DIRFINX}{\DIRFINR * sin(\DIRFINTHETA) * cos(\DIRFINPHI)}
  \pgfmathsetmacro{\DIRFINY}{\DIRFINR * sin(\DIRFINTHETA) * sin(\DIRFINPHI)}
  \pgfmathsetmacro{\DIRFINZ}{\DIRFINR * cos(\DIRFINTHETA)}
  % RECTA DE ACCIÓN INICIO
  \pgfmathsetmacro{\ACCINIR}{3.5}
  \pgfmathsetmacro{\ACCINITHETA}{180-\THETA}
  \pgfmathsetmacro{\ACCINIPHI}{\PHI + 180}
  \pgfmathsetmacro{\ACCINIX}{\ACCINIR * sin(\ACCINITHETA) * cos(\ACCINIPHI)}
  \pgfmathsetmacro{\ACCINIY}{\ACCINIR * sin(\ACCINITHETA) * sin(\ACCINIPHI)}
  \pgfmathsetmacro{\ACCINIZ}{\ACCINIR * cos(\ACCINITHETA)}
  % RECTA DE ACCIÓN FIN
  \pgfmathsetmacro{\ACCFINR}{7.0}
  \pgfmathsetmacro{\ACCFINTHETA}{\THETA}
  \pgfmathsetmacro{\ACCFINPHI}{\PHI}
  \pgfmathsetmacro{\ACCFINX}{\ACCFINR * sin(\ACCFINTHETA) * cos(\ACCFINPHI)}
  \pgfmathsetmacro{\ACCFINY}{\ACCFINR * sin(\ACCFINTHETA) * sin(\ACCFINPHI)}
  \pgfmathsetmacro{\ACCFINZ}{\ACCFINR * cos(\ACCFINTHETA)}
  %
  \tikzfading[name=fade out, inner color=transparent!0,
  outer color=transparent!100]
  
  \begin{tikzpicture}[
    scale=\scl,
    tdplot_main_coords,
    every node/.style={font=\small},
    linaux/.style={ultra thin, color=\colorlinaux},
    directrizfront/.style={dashed,ultra thin,\colordirectrizfront},
    directrizback/.style={dashed,ultra thin,\colordirectrizback},
    ndirector/.style={-{Latex[round]},line width=0.8pt,\colorndirector},
    ejeposit/.style={thick,->},
    ejenegat/.style={ultra thin,black!20},
    electron/.style={fill=\colorelectron, draw=black},
    momento/.style={%
      %-{Latex[width=5pt,length=7pt]},color=blue!90!black,line width=1.1pt
      -{Latex[round]},color=\colormomento,line width=1.1pt
    },
    background/.style={
      line width=\bgborderwidth,
      draw=\bgbordercolor,
      fill=\bgcolor,
    },
    ]
    % COORDENADAS
    \coordinate (o) at (0,0,0);
    \coordinate (-dx) at (-\LONGEJENEGX,0,0);
    \coordinate (dx) at (\LONGEJEPOSX,0,0);
    \coordinate (-dy) at (0,-\LONGEJENEGY,0);
    \coordinate (dy) at (0,\LONGEJEPOSY,0);
    \coordinate (-dz) at (0,0,-\LONGEJENEGZ);
    \coordinate (dz) at (0,0,\LONGEJEPOSZ);
    % Recta de acción
    \coordinate (accini) at (\ACCINIX,\ACCINIY,\ACCINIZ);
    \coordinate (accfin) at (\ACCFINX,\ACCFINY,\ACCFINZ);
    % Vector director n
    \coordinate (dirini) at (\DIRINIX,\DIRINIY,\DIRINIZ);
    \coordinate (dirfin) at (\DIRFINX,\DIRFINY,\DIRFINZ);
    
    %\coordinate (accfin) at (FX,FY,FZ);
    % DIBUJO
    % Líneas débiles ejes
    \draw[ejenegat] (o) -- (-dx);
    \draw[ejenegat] (o) -- (-dy);
    \draw[ejenegat] (o) -- (-dz);
    % Ejes
    \draw[ejeposit] (o) -- (dx);
    \draw[ejeposit] (o) -- (dy);
    \draw[ejeposit] (o) -- (dz);
    \node[anchor=north east,name=letrax] at (dx) {$x$};
    \node[anchor=north west,name=letray] at (dy) {$y$};
    \node[anchor=south,name=letraz] at (dz) {$z$};
    
    %  % Giro eje x
    % \tdplotsetthetaplanecoords{-90}
    %  % Notice you have to tell tiks-3dplot you are now in rotated coords
    %  % Since tikz-3dplot swaps the planes in tdplotsetthetaplanecoords,
    %  % the former y axis is now the z axis.
    % \tdplotdrawarc[tdplot_rotated_coords,line width=2pt,
    % -{Latex[length=11pt,width=7pt,flex=1]},color=black!50]
    % {(0,0,1.9)}{0.3}{380}{40}{anchor=south west,color=black}{}
    %  % Reponer parte derecha del eje y
    % \draw[ultra thick,->] (1.9,0,0) -- (\longeje,0,0);
    %% \tdplotdrawarc[tdplot_rotated_coords,line width=2pt,
    %% -{Latex[round,length=6pt,width=4pt]},color=black!50]
    %% {(0,0,1.9)}{0.3}{380}{40}{anchor=south west,color=black}{}
    %%  % Reponer parte derecha del eje y
    %% \draw[ultra thick,->] (1.9,0,0) -- (\longeje,0,0);

    % RECTA DE ACCIÓN
    \draw[directrizback] (accini) -- (o);
    \draw[directrizfront] (o) -- (accfin);

    % VECTOR DIRECTOR
    \draw[ndirector] (dirini) -- (dirfin);
    \node[above left,\colorndirector] at (dirfin) {$\xhat{n}$};

    % VECTOR MOMENTO
    \tdplotsetcoord{P}{\PR}{\PTHETA}{\PPHI}
    % draw a vector from origin to point (P)
    % \draw[-{Stealth[width=7pt]},color=red!80!black,ultra thick] (O) -- (P)
    % node[above right=0pt and 0pt] {\footnotesize $\braket{\vec{S}}$};
    % Momento del electrón en el origen
    % Líneas auxiliares por debajo del momento
    \draw[linaux] (Pxy) -- (Px);
    %\draw[linaux] (O) -- (Pxy);
    
    \draw[momento] (o) -- (P);
    %\node[right=0pt,\colormomento] at (P) {\footnotesize $\vvv{p}$};
    \node[above left=0pt and -2pt,\colormomento]
    at (P) {\footnotesize $\vvv{p}$};
    % Líneas auxiliares por encima del momento
    \draw[linaux] (P) -- (Pxy);

    \draw[linaux] (Pxy) -- (Py);
    %
    \draw[linaux] (Py) -- (Pyz);
    \draw[linaux] (Pyz) -- (Pz);
    \draw[linaux] (Pyz) -- (P);
    \draw[linaux] (Pz) -- (Pxz);
    \draw[linaux] (Pxz) -- (P);
    \draw[linaux] (Pxz) -- (Px);
    
    \tdplotsetthetaplanecoords{\PTHETA}
    
    % draw the angle \phi, and label it
    % syntax:
    % \tdplotdrawarc[coordinate frame, draw options]
    % {center point}{r}{angle}{label options}{label}
    % \tdplotdrawarc{(O)}{0.8}{0}{\phivec}{anchor=north}{\footnotesize $\pi/2$}
    
    % set the rotated coordinate system so the x'-y' plane lies within the
    % "theta plane" of the main coordinate system
    % syntax: \tdplotsetthetaplanecoords{\phi}
    \tdplotsetthetaplanecoords{\PPHI}
    
    % draw theta arc and label, using rotated coordinate system
    % \tdplotdrawarc[tdplot_rotated_coords]
    % {(0,0,0)}{0.8}{0}{\thetavec}{above right=1pt and 1pt}{\footnotesize $\theta$}
    
    % Electrón en el origen
    \filldraw[electron] (o) circle[radius=2pt];

    %\node at (0,.5,3.0) {\ACCINIX};
    %\node at (0,.5,2.6) {\ACCINIY};
    %\node at (0,.5,2.2) {\ACCINIZ};
    % Fondo amarillo
    \begin{scope}[on background layer]
      % \node [line width=1pt, draw=\backgroundbordercolor,
      % fill=\backgroundcolor, fit= (O) (letraejex) (letraejey)] {};
      \node [background, fit= (accini) (accfin) (letrax) (letraz) (letray)] {};
    \end{scope}
  \end{tikzpicture}
  \caption{Electrón en el origen, animado con un vector trimomento $\vvv{p}$,
    en la dirección $\xhat{n}$.}
  \label{fig:spin12-electron-trimomento}
\end{figure}

La helicidad de una partícula, como el electrón de la figura, es la
componente medida de su spin
\emph{hacia arriba} $\ket{\chi_+}$ o \emph{hacia abajo} $\ket{\chi_-}$,
en la dirección del momento.

En la figura~\ref{fig:spin12-electron-estados-helicidad} se representan estos
dos posibles valores de la helicidad de una partícula de spin 1/2, como el
electrón.
\begin{figure}[ht]
  \centering
  % COLORES
  \newcommand{\colordirectrizfront}{black}
  \newcommand{\colordirectrizback}{black!50}
  \newcommand{\colormomento}{blue!85!black}
  \newcommand{\colorhelicplus}{green!60!black}
  \newcommand{\colorhelicplustext}{green!45!black}
  \newcommand{\colorhelicminus}{red!80!black}
  \newcommand{\colorhelicminustext}{red!80!black}
  \newcommand{\colorlinaux}{black!15}
  \newcommand{\colorndirector}{black!60}
  \newcommand{\colorelectron}{blue}
  %
  \def\scl{1.2}
  % EJES
  % x
  \pgfmathsetmacro{\LONGEJEPOSX}{3}
  \pgfmathsetmacro{\LONGEJENEGX}{3}
  % y
  \pgfmathsetmacro{\LONGEJEPOSY}{3}
  \pgfmathsetmacro{\LONGEJENEGY}{1.5}
  % z
  \pgfmathsetmacro{\LONGEJEPOSZ}{2}
  \pgfmathsetmacro{\LONGEJENEGZ}{1.5}

  % 
  \pgfmathsetmacro{\PMOD}{2.2}
  \pgfmathsetmacro{\NMOD}{.5}
  \pgfmathsetmacro{\THETA}{55}
  \pgfmathsetmacro{\PHI}{50}
  % 
  \tdplotsetmaincoords{60}{110}
  % HELICIDAD +
  \pgfmathsetmacro{\PR}{\PMOD}
  \pgfmathsetmacro{\PTHETA}{\THETA}
  \pgfmathsetmacro{\PPHI}{\PHI}
  % HELICIDAD -
  \pgfmathsetmacro{\HELICMINR}{\PMOD}
  \pgfmathsetmacro{\HELICMINTHETA}{180 - \THETA}
  \pgfmathsetmacro{\HELICMINPHI}{\PHI + 180}
  \pgfmathsetmacro{\HELICMINX}{\HELICMINR*sin(\HELICMINTHETA)*cos(\HELICMINPHI)}
  \pgfmathsetmacro{\HELICMINY}{\HELICMINR*sin(\HELICMINTHETA)*sin(\HELICMINPHI)}
  \pgfmathsetmacro{\HELICMINZ}{\HELICMINR*cos(\HELICMINTHETA)}
  % VECTOR DIRECTOR INICIO
  \pgfmathsetmacro{\DIRINIR}{4.2}
  \pgfmathsetmacro{\DIRINITHETA}{\THETA}
  \pgfmathsetmacro{\DIRINIPHI}{\PHI}
  \pgfmathsetmacro{\DIRINIX}{\DIRINIR * sin(\DIRINITHETA) * cos(\DIRINIPHI)}
  \pgfmathsetmacro{\DIRINIY}{\DIRINIR * sin(\DIRINITHETA) * sin(\DIRINIPHI)}
  \pgfmathsetmacro{\DIRINIZ}{\DIRINIR * cos(\DIRINITHETA)}
  % VECTOR DIRECTOR FIN
  \pgfmathsetmacro{\DIRFINR}{5.5}
  \pgfmathsetmacro{\DIRFINTHETA}{\THETA}
  \pgfmathsetmacro{\DIRFINPHI}{\PHI}
  \pgfmathsetmacro{\DIRFINX}{\DIRFINR * sin(\DIRFINTHETA) * cos(\DIRFINPHI)}
  \pgfmathsetmacro{\DIRFINY}{\DIRFINR * sin(\DIRFINTHETA) * sin(\DIRFINPHI)}
  \pgfmathsetmacro{\DIRFINZ}{\DIRFINR * cos(\DIRFINTHETA)}
  % RECTA DE ACCIÓN INICIO
  \pgfmathsetmacro{\ACCINIR}{3.5}
  \pgfmathsetmacro{\ACCINITHETA}{180 - \THETA}
  \pgfmathsetmacro{\ACCINIPHI}{\PHI + 180}
  \pgfmathsetmacro{\ACCINIX}{\ACCINIR * sin(\ACCINITHETA) * cos(\ACCINIPHI)}
  \pgfmathsetmacro{\ACCINIY}{\ACCINIR * sin(\ACCINITHETA) * sin(\ACCINIPHI)}
  \pgfmathsetmacro{\ACCINIZ}{\ACCINIR * cos(\ACCINITHETA)}
  % RECTA DE ACCIÓN FIN
  \pgfmathsetmacro{\ACCFINR}{7.0}
  \pgfmathsetmacro{\ACCFINTHETA}{\THETA}
  \pgfmathsetmacro{\ACCFINPHI}{\PHI}
  \pgfmathsetmacro{\ACCFINX}{\ACCFINR * sin(\ACCFINTHETA) * cos(\ACCFINPHI)}
  \pgfmathsetmacro{\ACCFINY}{\ACCFINR * sin(\ACCFINTHETA) * sin(\ACCFINPHI)}
  \pgfmathsetmacro{\ACCFINZ}{\ACCFINR * cos(\ACCFINTHETA)}
  %
  \tikzfading[name=fade out, inner color=transparent!0,
  outer color=transparent!100]
  
  \begin{tikzpicture}[
    scale=\scl,
    tdplot_main_coords,
    every node/.style={font=\small},
    linaux/.style={ultra thin, color=\colorlinaux},
    directrizfront/.style={dashed,ultra thin,\colordirectrizfront},
    directrizback/.style={dashed,ultra thin,\colordirectrizback},
    ndirector/.style={-{Latex[round]},line width=0.8pt,\colorndirector},
    ejeposit/.style={thick,->},
    ejenegat/.style={ultra thin, black!20},
    electron/.style={fill=\colorelectron, draw=black},
    momento/.style={%
      %-{Latex[width=5pt,length=7pt]},color=blue!90!black,line width=1.1pt
      -{Latex[round]},color=\colormomento,line width=1.1pt
    },
    helicidadplus/.style={%
      %-{Latex[width=5pt,length=7pt]},color=blue!90!black,line width=1.1pt
      -{Latex[round]},color=\colorhelicplus,line width=1.1pt
    },
    helicidadminus/.style={%
      %-{Latex[width=5pt,length=7pt]},color=blue!90!black,line width=1.1pt
      -{Latex[round]},color=\colorhelicminus,line width=1.1pt
    },
    background/.style={
      line width=\bgborderwidth,
      draw=\bgbordercolor,
      fill=\bgcolor,
    },
    ]
    % COORDENADAS
    \coordinate (o) at (0,0,0);
    \coordinate (-dx) at (-\LONGEJENEGX,0,0);
    \coordinate (dx) at (\LONGEJEPOSX,0,0);
    \coordinate (-dy) at (0,-\LONGEJENEGY,0);
    \coordinate (dy) at (0,\LONGEJEPOSY,0);
    \coordinate (-dz) at (0,0,-\LONGEJENEGZ);
    \coordinate (dz) at (0,0,\LONGEJEPOSZ);
<    % Recta de acción
    \coordinate (accini) at (\ACCINIX,\ACCINIY,\ACCINIZ);
    \coordinate (accfin) at (\ACCFINX,\ACCFINY,\ACCFINZ);
    % Vector helicidad
    \coordinate (helic) at (\HELICMINX,\HELICMINY,\HELICMINZ);

    % Vector director n
    \coordinate (dirini) at (\DIRINIX,\DIRINIY,\DIRINIZ);
    \coordinate (dirfin) at (\DIRFINX,\DIRFINY,\DIRFINZ);
    
    %\coordinate (accfin) at (FX,FY,FZ);
    % DIBUJO
    % Líneas débiles ejes
    %\draw[ejenegat] (o) -- (-dx);
    %\draw[ejenegat] (o) -- (-dy);
    %\draw[ejenegat] (o) -- (-dz);
    % Ejes
    \draw[ejeposit] (o) -- (dx);
    \draw[ejeposit] (o) -- (dy);
    \draw[ejeposit] (o) -- (dz);
    \node[anchor=north east,name=letrax] at (dx) {$x$};
    \node[anchor=north west,name=letray] at (dy) {$y$};
    \node[anchor=south,name=letraz] at (dz) {$z$};
    
    %  % Giro eje x
    % \tdplotsetthetaplanecoords{-90}
    %  % Notice you have to tell tiks-3dplot you are now in rotated coords
    %  % Since tikz-3dplot swaps the planes in tdplotsetthetaplanecoords,
    %  % the former y axis is now the z axis.
    % \tdplotdrawarc[tdplot_rotated_coords,line width=2pt,
    % -{Latex[length=11pt,width=7pt,flex=1]},color=black!50]
    % {(0,0,1.9)}{0.3}{380}{40}{anchor=south west,color=black}{}
    %  % Reponer parte derecha del eje y
    % \draw[ultra thick,->] (1.9,0,0) -- (\longeje,0,0);
    %% \tdplotdrawarc[tdplot_rotated_coords,line width=2pt,
    %% -{Latex[round,length=6pt,width=4pt]},color=black!50]
    %% {(0,0,1.9)}{0.3}{380}{40}{anchor=south west,color=black}{}
    %%  % Reponer parte derecha del eje y
    %% \draw[ultra thick,->] (1.9,0,0) -- (\longeje,0,0);

    % RECTA DE ACCIÓN
    \draw[directrizback] (accini) -- (o);
    \draw[directrizfront] (o) -- (accfin);

    % VECTOR DIRECTOR
    \draw[ndirector] (dirini) -- (dirfin);
    \node[above left,\colorndirector] at (dirfin) {$\xhat{n}$};

    % VECTOR MOMENTO
    \tdplotsetcoord{P}{\PR}{\PTHETA}{\PPHI}
    % draw a vector from origin to point (P)
    % \draw[-{Stealth[width=7pt]},color=red!80!black,ultra thick] (O) -- (P)
    % node[above right=0pt and 0pt] {\footnotesize $\braket{\vec{S}}$};
    % Momento del electrón en el origen
    % Líneas auxiliares por debajo del momento
    \draw[linaux] (Pxy) -- (Px);
    %\draw[linaux] (O) -- (Pxy);
    
    \draw[helicidadplus] (o) -- (P);
    %\node[right=0pt,\colormomento] at (P) {\footnotesize $\vvv{p}$};
    \node[above left=0pt and -6pt,\colorhelicplustext]
    at (P) {\footnotesize $\ket{\chi_+}$};
    \draw[helicidadminus] (o) -- (helic);
    \node[above left=0pt and -6pt,\colorhelicminustext]
    at (helic) {\footnotesize $\ket{\chi_-}$};    
    % Líneas auxiliares por encima del momento
    \draw[linaux] (P) -- (Pxy);

    \draw[linaux] (Pxy) -- (Py);
    %
    \draw[linaux] (Py) -- (Pyz);
    \draw[linaux] (Pyz) -- (Pz);
    \draw[linaux] (Pyz) -- (P);
    \draw[linaux] (Pz) -- (Pxz);
    \draw[linaux] (Pxz) -- (P);
    \draw[linaux] (Pxz) -- (Px);
    \draw[linaux,line width=.8pt,-{Latex}] (o) -- (Pxy);
    
    \tdplotsetthetaplanecoords{\PTHETA}
    
    % draw the angle \phi, and label it
    % syntax:
    % \tdplotdrawarc[coordinate frame, draw options]
    % {center point}{r}{angle}{label options}{label}
    % \tdplotdrawarc{(O)}{0.8}{0}{\phivec}{anchor=north}{\footnotesize $\pi/2$}
    
    % set the rotated coordinate system so the x'-y' plane lies within the
    % "theta plane" of the main coordinate system
    % syntax: \tdplotsetthetaplanecoords{\phi}
    \tdplotsetthetaplanecoords{\PPHI}
    
    % draw theta arc and label, using rotated coordinate system
    % \tdplotdrawarc[tdplot_rotated_coords]
    % {(0,0,0)}{0.8}{0}{\thetavec}{above right=1pt and 1pt}{\footnotesize $\theta$}
    
    % Electrón en el origen
    %\filldraw[electron] (o) circle[radius=2pt];

    %\node at (0,.5,3.0) {\ACCINIX};
    %\node at (0,.5,2.6) {\ACCINIY};
    %\node at (0,.5,2.2) {\ACCINIZ};
    
    % Nube de probabilidad (estado espacial)
    \fill [blue,path fading=fade out] (0,0,0) circle [radius=4pt];
    % node[below right=4pt and 0pt] {$\scriptstyle e^-$};
    % Fondo amarillo
    \begin{scope}[on background layer]
      % \node [line width=1pt, draw=\backgroundbordercolor,
      % fill=\backgroundcolor, fit= (O) (letraejex) (letraejey)] {};
      \node [background, fit= (accini) (accfin) (letrax) (letraz) (letray)] {};
    \end{scope}
  \end{tikzpicture}
  \caption{Los dos posibles estados de helicidad, $\ket{\chi_+}$ y
    $\ket{\chi_-}$, de una partícula de spin 1/2. La partícula se representa
    en azul.}
  \label{fig:spin12-electron-estados-helicidad}
\end{figure}

Desde otro punto de vista, estos dos estados de helicidad de partículas de
spin 1/2 se corresponden con los vectores propios y valores propios
de la matriz $\xhat{n}\cdot\mmm{\sigma}/2$
\begin{align*}
  &\frac{\xhat{n}\cdot\mmm{\sigma}}{2}\ket{\chi_+}=+\frac{1}{2}\ket{\chi_+}\\
  &\frac{\xhat{n}\cdot\mmm{\sigma}}{2}\ket{\chi_-}=-\frac{1}{2}\ket{\chi_-}
\end{align*}

Al operador $\xhat{n}\cdot\mmm{\sigma}/2$ se le llama operador
\emph{helicidad}, $\mmm{\hat{\Sigma}}$
\[
  \mmm{\hat{\Sigma}}
  = \frac{1}{2}\xhat{n}\cdot\mmm{\sigma}
  = \frac{\vvv{p}}{2|\vvv{p}|}\cdot\mmm{\sigma}
\]

Los valores y vectores propios del operador helicidad serían
\begin{align} \label{eq:spin12-right-handed}
  &\mmm{\hat{\Sigma}}\ket{\chi_+} = +\frac{1}{2}\ket{\chi_+}\\
  \label{eq:spin12-left-handed}
  &\mmm{\hat{\Sigma}}\ket{\chi_-} = -\frac{1}{2}\ket{\chi_-}
\end{align}
que representan los estados dextrógiro\footnote{Right handed}
y levógiro\footnote{Left handed} de una partícula de spin 1/2.

En Física fundamental se utiliza un sistema natural de unidades donde
la constante de Planck, la velocidad de propagación de la luz en el vacío y
la constante de gravitación universal valen la unidad
\[
  \hbar = c = G = 1
\]

En mecánica cuántica no relativista se suele utilizar el SI de unidades,
y entonces habría que modificar el operador helicidad
\[
  \mmm{\hat{\Sigma}}
  = \frac{\hbar\vvv{p}\cdot\mmm{\sigma}}{2|\vvv{p}|}
\]

La matriz de rotación activa de espinores sería
\[
  e^{-i\hbar\frac{\theta}{2} (\xhat{n}\cdot\mmm{\sigma})}
\]









Para concretar ideas, si el estado del electrón está descrito por
\[
  \ket{\Psi} = \ket{z_+}
\]
sabemos que si midiéramos el spin en la dirección $z$, obtendríamos con
seguridad el spin hacia arriba.
Pero como nuestro experimento no utiliza esa dirección, obtendríamos como
resultado un spin hacia arriba, $\ket{n_+}$, o hacia abajo,
$\ket{n_-}$, en la dirección del movimiento, con unas ciertas
probabilidades.

Esto se debe a que ahora no obtendremos los valores propios de
$\mmm{\sigma_z}$, sino los de otra matriz
\[
  \mmm{\sigma}_{\xhat{n}}
  = n_x\mmm{\sigma_x} + n_y\mmm{\sigma_y} + n_z\mmm{\sigma_z}
\]
cuyos valores propios serían diferentes de los de $\mmm{\sigma_z}$
\begin{align*}
  &\sigma_{\xhat{n}} \ket{n_+}
  = +1 \ket{n_+}\\
  &\sigma_{\xhat{n}} \ket{n_-}
  = -1 \ket{n_-}  
\end{align*}
Y el resultado de la medida ahora sería
\emph{uno de los valores propios de $\mmm{\sigma_{\xhat{n}}}$} que se
corresponden con los nuevos estados propios $\ket{n_+}$ y
$\ket{1}$.




%%% Local Variables:
%%% coding: utf-8
%%% mode: latex
%%% TeX-engine: luatex
%%% TeX-master: "../cuarentena.tex"
%%% End:

