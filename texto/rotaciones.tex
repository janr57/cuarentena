% rotaciones.tex
%
% Copyright (C) 2022 José A. Navarro Ramón <janr.devel@gmail.com>
% 1) Código LuaLatex:
%    Licencia GPL-2.
% 2) Producto en pdf, postscript, etc.:
%    Licencia Creative Commons Recognition Share alike. (CC-BY-SA)

\chapter{Rotaciones}

\section{Fórmula de Euler}\label{sect:rot-taylor}
Empezaremos demostrando la fórmula de Euler
\begin{equation}\label{eq:rot-Euler}
  e^{i\theta} = \cos\theta + i \sin\theta
\end{equation}
mediante un desarrollo en serie de Taylor de la exponencial alrededor de
$\theta=0$
\begin{align*} 
  e^{i\theta}
  &=
    \frac{(i\theta)^0}{0!}
    + \frac{(i\theta)^1}{1!}
    + \frac{(i\theta)^2}{2!}
    + \frac{(i\theta)^3}{3!}
    + \frac{(i\theta)^4}{4!}
    + \frac{(i\theta)^5}{5!}
    + \frac{(i\theta)^6}{6!}
    + \frac{(i\theta)^7}{7!}
    + \cdots\\
  &=
    1
    + i\theta
    - \frac{\theta^2}{2!}
    - i\,\frac{\theta^3}{3!}
    + \frac{\theta^4}{4!}
    + i\,\frac{\theta^5}{5!}
    - \frac{\theta^6}{6!}
    - i\,\frac{\theta^7}{7!}
    + \cdots\\
  &=
    \left(
    1 - \frac{\theta^2}{2} + \frac{\theta^4}{4!} - \frac{\theta^6}{6!}
    + \cdots
    \right)
    + i\left(
    \theta - \frac{\theta^3}{3!} + \frac{\theta^5}{5!} - \frac{\theta^7}{7!}
    + \cdots
    \right)\\
  &=
    \cos\theta + i \sin\theta
\end{align*}

\section{Rotación de números complejos}
En esta sección veremos que para rotar un número complejo $z\in\symbb{C}$,
hay que multiplicarlo por otro de módulo unidad
$w\in\symbb{C}$, $|w|=1$, que en forma exponencial sería
$w=e^{i\theta}$
\begin{equation}\label{eq:rot-rotacion-complejo}
  z' = w z = e^{i\theta} z
\end{equation}

Se puede intuir esta propiedad expresando el complejo $z$ en forma
exponencial, con módulo $|z| = r$ y argumento $\alpha$,
$z=e^{i\alpha}$ (ver figuras~\ref{fig:rot-z} y~\ref{fig:rot-z-rotado}),
\begin{equation}\label{eq:rot-moduloconservado}
  z' = e^{i\theta} z
  = e^{i\theta} \,re^{i\alpha}
  = r e^{i(\alpha+\theta)}
\end{equation}
Sabemos además, que en toda rotación hay un invariante (una magnitud que se
conserva). En el caso de la rotación en el plano complejo se conserva el módulo
del complejo. Esto se puede comprobar en \eqref{eq:rot-moduloconservado},
pues $|z'| = |z| = r$.

Vamos a volver a demostrar esta propiedad de una forma algo más elaborada,
calculando el módulo de un complejo genérico expresado como parte real e
imaginaria, $z = x + iy$
\[
  |z|^2 = z^* z = (x+iy)^* (x+iy) = (x-iy) (x+iy) = x^2 + y^2
\]

Y vemos que es el mismo que el del complejo rotado mediante $e^{i\theta}$,
calculado en~\eqref{eq:rot-modulo-z'},
$z'=(x\cos\theta-y\sin\theta) + i(x\sin\theta + y\cos\theta)$
{\small
\begin{align*}
  |z'|^2
  &= (z')^{*} z'\\
  &=
    \left[(x\cos\theta - y\sin\theta) - i(x\sin\theta + y\cos\theta)\right]
    \left[(x\cos\theta - y\sin\theta) + i(x\sin\theta + y\cos\theta)\right]\\
  &=
    x^2\cos^2\theta + y^2\sin^2\theta - \cancelout{2xy\sin\theta\cos\theta}
    +x^2\sin^2\theta + y^2\cos^2\theta + \cancelout{2xy\sin\theta\cos\theta}\\
  &=
    x^2 (\sin^2\theta + \cos^2\theta) + y^2 (\sin^2\theta + \cos^2\theta)
  =
    x^2 + y^2
\end{align*}
}

\begin{figure}[ht]
  \centering
  \begin{minipage}{0.40\linewidth}
    % Escala
    \def\scl{1}
    % Eje x
    \pgfmathsetmacro{\XMLONG}{0}
    \pgfmathsetmacro{\XPLONG}{3}
    % Eje y
    \pgfmathsetmacro{\YMLONG}{0}
    \pgfmathsetmacro{\YPLONG}{2.7}
    % Complejo
    \pgfmathsetmacro{\ZMOD}{2.5}
    \pgfmathsetmacro{\ZANG}{25}
    %
    \centering
    \begin{tikzpicture}[%
      scale=\scl,
      baseline,
      every node/.style={black,font=\small},
      eje/.style={->},
      complejo/.style={-{Latex}, shorten >=1.2pt, line width=.8pt},
      background/.style={
        line width=\bgborderwidth,
        draw=\bgbordercolor,
        fill=\bgcolor,
      },
      ]
      % Coordenadas
      \coordinate (O) at (0,0);
      \coordinate (xini) at (-\XMLONG cm,0);
      \coordinate (xfin) at (\XPLONG cm,0);
      \coordinate (yini) at (0,-\YMLONG cm);
      \coordinate (yfin) at (0,\YPLONG cm);
      \coordinate (z) at (\ZANG:\ZMOD cm);
      \path (O) -- coordinate (Ozmidway) (z);
      % Ángulo alfa
      \path (xfin) -- (O) -- (z)
      pic [draw=black,fill=black!20,"$\alpha$",angle radius=8mm,
      angle eccentricity=1.3]
      {angle = xfin--O--z};
%      \path (xfin) -- (O) -- (z)
%      pic [draw=black,fill=black!20,"$\alpha$",angle radius=8mm,
%      angle eccentricity=1.3,-{Latex[length=1.4mm,scale=.9,bend]}]
%      {angle = xfin--O--z};
      % Ejes
      \draw[eje] (xini) -- (xfin);
      \node[right, name=letraejex] at (xfin) {$\operatorname{Re}$};
      \draw[eje] (yini) -- (yfin);
      \node[above, name=letraejey] at (yfin) {$\operatorname{Im}$};
      % Vector de posición del punto P
      %\draw[black!70,-{Latex},shorten >=1.2pt] (O) -- (P);
      \draw[complejo] (O) -- (z);
      \node[above=3pt] at (Ozmidway) {$\vvv{r}$};
      % Punto
      \fill[fill=red,draw=black] (z) circle [radius=1.4pt];
      \node[above right] at (z) {$z$};
      % Origen
      \filldraw (O) circle [radius=.2pt];
      % Fondo amarillo
      \begin{scope}[on background layer]
%        \node [line width=1pt, draw=\backgroundbordercolor,
%        fill=\backgroundcolor, fit= (O) (letraejex) (letraejey)] {};
        \node [background, fit= (O) (letraejex) (letraejey)] {};
      \end{scope}
    \end{tikzpicture}
    \caption{Número complejo $z$ de módulo $|z| = r$ y argumento $\alpha$ en el
      plano complejo.}
    \label{fig:rot-z}
  \end{minipage}
  \hspace{1em}
  \begin{minipage}{0.45\linewidth}
    % Escala
    \def\scl{1}
    % Eje x
    \pgfmathsetmacro{\XMLONG}{0}
    \pgfmathsetmacro{\XPLONG}{3}
    % Eje y
    \pgfmathsetmacro{\YMLONG}{0}
    \pgfmathsetmacro{\YPLONG}{2.7}
    % Complejo z
    \pgfmathsetmacro{\ZMOD}{2.5}
    \pgfmathsetmacro{\ZORIGANG}{25}
    \pgfmathsetmacro{\ANGROT}{30}
    \pgfmathsetmacro{\ZANG}{\ZORIGANG + \ANGROT}
    %
    \centering
    \begin{tikzpicture}[%
      scale=\scl,
      baseline,
      every node/.style={black,font=\small},
      eje/.style={->},
      complejo/.style={-{Latex}, shorten >=1.2pt, line width=.8pt},
      crotado/.style={-{Latex}, shorten >=1.2pt, line width=.8pt, black!30},
      background/.style={
        line width=\bgborderwidth,
        draw=\bgbordercolor,
        fill=\bgcolor,
      },
      ]
      % Coordenadas
      \coordinate (O) at (0,0);
      \coordinate (xini) at (-\XMLONG cm,0);
      \coordinate (xfin) at (\XPLONG cm,0);
      \coordinate (yini) at (0,-\YMLONG cm);
      \coordinate (yfin) at (0,\YPLONG cm);
      \coordinate (zorig) at (\ZORIGANG:\ZMOD cm);
      \coordinate (z) at (\ZANG:\ZMOD cm);
      \path (O) -- coordinate (Ozmidway) (z);
      % Ángulo alfa
      \path (xfin) -- (O) -- (zorig)
      pic [draw=gray!50,angle radius=25mm, angle eccentricity=1.3,
      -{Latex[bend]},shorten <=3pt,shorten >=3pt]
      {angle = zorig--O--z};
      % Ángulo alfa
      \path (xfin) -- (O) -- (zorig)
      pic [draw=black!50,fill=black!20,"$\alpha$",angle radius=8mm,
      angle eccentricity=1.3]
      {angle = xfin--O--zorig};
      % Ángulo theta
      \path (zorig) -- (O) -- (z)
      pic [draw=black,fill=green!20,"$\theta$",angle radius=8mm,
      angle eccentricity=1.3,-{Latex[length=1.4mm,scale=.9,bend]}]
      {angle = zorig--O--z};
      % Ejes
      \draw[eje] (xini) -- (xfin);
      \node[right, name=letraejex] at (xfin) {$\operatorname{Re}$};
      \draw[eje] (yini) -- (yfin);
      \node[above, name=letraejey] at (yfin) {$\operatorname{Im}$};
      % Vector de posición del punto P
      %\draw[black!70,-{Latex},shorten >=1.2pt] (O) -- (P);
      \draw[crotado] (O) -- (zorig);
      \draw[complejo] (O) -- (z);
      \node[above left=4pt and -4pt] at (Ozmidway) {$\vvv{r}$};
      % Puntos
      \fill[fill=black!40,draw=black!70] (zorig) circle [radius=1.4pt];
      \fill[fill=red,draw=black] (z) circle [radius=1.4pt];
      \node[above right] at (zorig) {$z$};
      \node[above right] at (z) {$z' = re^{i(\alpha+\theta)}$};
      % Origen
      \filldraw (O) circle [radius=.2pt];
      % Fondo amarillo
      \begin{scope}[on background layer]
%        \node [line width=1pt, draw=\backgroundbordercolor,
%        fill=\backgroundcolor, fit= (O) (letraejex) (letraejey)] {};
        \node [background, fit= (O) (letraejex) (letraejey)] {};
      \end{scope}
    \end{tikzpicture}
    \caption{Vector $z=re^{i\alpha}$ rotado un ángulo $\theta$ al
      multiplicarlo por $e^{i\theta}$. Obsérvese que el módulo $|z| = r$ se
      mantiene constante.}
    \label{fig:rot-z-rotado}
  \end{minipage}
\end{figure}

Ahora volvemos a escribir el complejo con su parte real e imaginaria, $z=x+iy$
y le aplicamos una rotación
\[
  x'+iy'
  = e^{i\theta} (x+iy)
    = (\cos\theta + i\sin\theta) (x + iy)
\]
\begin{equation}\label{eq:rot-modulo-z'}
  x'+iy'
   = (x\cos\theta - y\sin\theta) + i (x\sin\theta + y\cos\theta)
\end{equation}

Obtenemos la parte real e imaginaria del complejo rotado
\begin{align*}
  x' &= x\cos\theta - y\sin\theta\\
  y' &= x\sin\theta + y\cos\theta
\end{align*}

Escribimos la rotación activa\footnotemark{} anterior en forma matricial
\footnotetext{En una rotación activa se gira el objeto, mientras que en
  una  pasiva los que se transforman son los ejes de coordenadas.
  Cada una es inversa de la otra. Una rotación pasiva en un sentido
  equivale a otra activa en sentido contrario.}
\[
  \begin{pNiceMatrix}
    x'\\
    y'
  \end{pNiceMatrix}
  =
  \begin{pNiceMatrix}
    \cos\theta & -\sin\theta\\
    \sin\theta & \cos\theta
  \end{pNiceMatrix}
  \begin{pNiceMatrix}
    x\\
    y
  \end{pNiceMatrix}
\]

Si comparamos la ecuación anterior con \eqref{eq:rot-rotacion-complejo}
llegamos a la conclusión de que $e^{i\theta}$ se puede expresar en forma
matricial
\begin{equation}\label{eq:rot-complejos-matriz}
  e^{i\theta}
  =
  \begin{pNiceMatrix}
    \cos\theta & -\sin\theta\\
    \sin\theta & \cos\theta
  \end{pNiceMatrix}
\end{equation}

¿Por qué $e^{i\theta}$ describe una rotación?
La clave radica en que el módulo de $i$ vale la unidad
\begin{equation}\label{eq:rot-modi}
  |i|=1
\end{equation}
y su cuadrado es \emph{menos uno}
\begin{equation}\label{eq:rot-i2}
  i^2 = -1
\end{equation}

\section{Rotación de vectores}
Haremos algo parecido con matrices, en lugar de la unidad imaginaria $i$.
En este caso, nos centramos en matrices reales $2\times 2$.
Nos \emph{inventaremos} una matriz $\mmm{A}$ antisimétrica, con
determinante unidad
\begin{equation}\label{eq:rot-matrizA}
  \mmm{A}
  = \begin{pNiceMatrix}
    0 & -1\\
    1 & 0
  \end{pNiceMatrix}
\end{equation}
Que sea antisimétrica significa que su traspuesta es igual a su opuesta
\[
  \mmm{A}^\trasp = -\mmm{A}
\]
Según esta propiedad, toda matriz antisimétrica debe tener sus elementos
diagonales nulos, y los no diagonales, con índices $i\neq j$ den ser opuestos:
$a_{ij} = -a_{ji}$.

El que una matriz sea antisimétrica y tenga determinante unidad, guarda una
cierta similitud con~\eqref{eq:rot-modi}
\begin{equation}\label{eq:rot-detA}
  \det\mmm{A}
  = \begin{vNiceMatrix}
    0 & -1\\
    1 & 0
  \end{vNiceMatrix}
  = 1
\end{equation}

Además, al calcular el cuadrado de la matriz observaremos el parecido
con~\eqref{eq:rot-i2}
\begin{equation}\label{eq:rot-A2}
  \mmm{A}^2
  = \begin{pNiceMatrix}
    0 & -1\\
    1 & 0
  \end{pNiceMatrix}
  \begin{pNiceMatrix}
    0 & -1\\
    1 & 0
  \end{pNiceMatrix}
    = \begin{pNiceMatrix}
    -1 & 0\\
    0 & -1
  \end{pNiceMatrix}
  = -
  \begin{pNiceMatrix}
    1 & 0\\
    0 & 1
  \end{pNiceMatrix}
  = -\mmm{I}
\end{equation}

Lo anterior nos sugiere poder representar una rotación de algún tipo mediante
$e^{\theta\!\mmm{A}}$.
Por tanto, desarrollamos $e^{\theta\!\mmm{A}}$ en serie de potencias en un
entorno de $\theta = 0$, tal y como hicimos en la
sección~\ref{sect:rot-taylor}
\begin{align*}
  e^{\theta\!\mmm{A}}
  &=
    \frac{(\theta\mmm{A})^0}{0!}
    + \frac{(\theta\mmm{A})^1}{1!}
    + \frac{(\theta\mmm{A})^2}{2!}
    + \frac{(\theta\mmm{A})^3}{3!}
    + \frac{(\theta\mmm{A})^4}{4!}
    + \frac{(\theta\mmm{A})^5}{5!}
    + \frac{(\theta\mmm{A})^6}{6!}
    + \cdots\\
  &=
    \mmm{I}
    + \theta\mmm{A}
    - \frac{\theta^2}{2!}\,\mmm{I}
    - \frac{\theta^3}{3!}\,\mmm{A}
    + \frac{\theta^4}{4!}\,\mmm{I}
    + \frac{\theta^5}{5!}\,\mmm{A}
    - \frac{\theta^6}{6!}\,\mmm{I}
    - \frac{\theta^7}{7!}\,\mmm{A}
    + \cdots\\
  &=
    \left(
    1 - \frac{\theta^2}{2} + \frac{\theta^4}{4!} - \frac{\theta^6}{6!}
    + \cdots
    \right)\,\mmm{I}
    + \left(
    \theta - \frac{\theta^3}{3!} + \frac{\theta^5}{5!} - \frac{\theta^7}{7!}
    + \cdots
    \right)\mmm{A}
\end{align*}
donde hemos calculado y generalizado las potencias de la matriz $\mmm{A}$
\begin{align*}
  \mmm{A}^3
  &= \mmm{A}^2 \mmm{A} = -\mmm{I} \mmm{A} = -\mmm{A}\\
  \mmm{A}^4
  &= \mmm{A}^2 \mmm{A}^2 = -\mmm{I} (-\mmm{I}) = \mmm{I}\\
  \mmm{A}^5
  &= \mmm{A}^4 \mmm{A} = \mmm{I} \mmm{A} = \mmm{A}\\
  \mmm{A}^6
  &= \mmm{A}^5 \mmm{A} = \mmm{A} \mmm{A} = \mmm{A}^2 = -\mmm{I}\\
  \mmm{A}^7
  &= \mmm{A}^6 \mmm{A} = -\mmm{I} \mmm{A} = -\mmm{A}\\
  \cdots
\end{align*}
de manera que obtenemos una relación que se comparara con la identidad de
Euler~\eqref{eq:rot-Euler}
\begin{equation}\label{eq:rot-ethetaA-abr}
  e^{\theta\!\mmm{A}}
  = \mmm{I} \cos\theta + \mmm{A} \sin\theta
\end{equation}

Desarrollamos las matrices $\mmm{A}$ e $\mmm{I}$ en la expresión
\eqref{eq:rot-ethetaA-abr}
\[
  e^{\theta\!\mmm{A}}
  = \cos\theta
  \begin{pNiceMatrix}
    1 & 0\\
    0 & 1
  \end{pNiceMatrix}
  + \sin\theta
  \begin{pNiceMatrix}
    0 & -1\\
    1 & 0
  \end{pNiceMatrix}
  =
  \begin{pNiceMatrix}
    \cos\theta & 0\\
    0 & \cos\theta
  \end{pNiceMatrix}
  +
  \begin{pNiceMatrix}
    0 & -\sin\theta\\
    \sin\theta & 0
  \end{pNiceMatrix}
\]

Al final obtenemos
\begin{equation}\label{eq:rot-vectores-matriz}
  e^{\theta\!\mmm{A}}
  = \begin{pNiceMatrix}
    \cos\theta & -\sin\theta\\
    \sin\theta & \cos\theta
  \end{pNiceMatrix}
\end{equation}
y llegamos a la conclusión de que ambas exponenciales,
\eqref{eq:rot-complejos-matriz} y \eqref{eq:rot-vectores-matriz} describen una
rotación en dos dimensiones. Una de ellas, $e^{i\theta}$, rota números
complejos en el plano complejo y la otra, $e^{\theta\!\mmm{A}}$, vectores de dos
dimensiones en el plano euclídeo $\symbb{R}^2$
\[
  e^{\theta\!\mmm{A}} = e^{i\theta}
  = \begin{pNiceMatrix}
    \cos\theta & -\sin\theta\\
    \sin\theta & \cos\theta
  \end{pNiceMatrix}
\]

Ya dijimos anteriormente que en toda rotación hay un invariante (una magnitud
que se conserva).
En el caso de la rotación en el plano complejo, vimos que se conservaba,
$|z'| = |z|$.
De forma similar, se puede demostrar que el módulo de un vector se mantiene
invariante cuando se rota mediante la exponencial $\exp\{\theta\mmm{A}\}$
de la matriz antisimétrica $\mmm{A}$.
El módulo de un vector genérico en $\symbb{R}^2$ es
\begin{equation}\label{eq:rot-modvecsinrotar}
  |r|^2
  = \vvv{r}\cdot\vvv{r}
  = \vvv{r}^\trasp \vvv{r}
  = \begin{pNiceMatrix}
    x & y
  \end{pNiceMatrix}
  \begin{pNiceMatrix}
    x\\
    y
  \end{pNiceMatrix}
  = x^2 + y^2
\end{equation}
Donde hemos utilizado la propiedad de que, cuando un vector en el espacio
euclídeo se representa como un vector columna, su producto escalar es el
producto de la traspuesta del vector multiplicada por el propio vector,
$\vvv{r}\cdot\vvv{r} = \vvv{r}^\trasp \vvv{r}$.

Rotamos el vector
\[
  \vvv{r'}
  = e^{\theta\!\mmm{A}} \vvv{r}
  = \begin{pNiceMatrix}
    \cos\theta & -\sin\theta\\
    \sin\theta & \cos\theta
  \end{pNiceMatrix}
  \begin{pNiceMatrix}
    x\\
    y
  \end{pNiceMatrix}
  = \begin{pNiceMatrix}
    x\cos\theta - y\sin\theta\\
    x\sin\theta + y\cos\theta
  \end{pNiceMatrix}
\]

Y comprobamos que tiene el mismo módulo que el vector sin rotar \eqref{eq:rot-modvecsinrotar}
{\small
\begin{align*}
  |r'|^2
  &=
    \vvv{r'}\cdot\vvv{r'}
    = (\vvv{r'})^\trasp \vvv{r'}\\
  &=
    \begin{pNiceMatrix}
      x\cos\theta - y\sin\theta & x\sin\theta + y\cos\theta
    \end{pNiceMatrix}
    \begin{pNiceMatrix}
      x\cos\theta - y\sin\theta\\
      x\sin\theta + y\cos\theta
    \end{pNiceMatrix}\\
  &=
    (x\cos\theta-y\sin\theta)^2 + (x\sin\theta+y\cos\theta)^2\\
  &=
    x^2\cos^2\theta + y^2\sin^2\theta - \cancelout{2xy\sin\theta\cos\theta}
    + x^2\sin^2\theta + y^2\cos^2\theta + \cancelout{2xy\sin\theta\cos\theta}\\
  &=
    x^2(\cos^2\theta+\sin^2\theta) + y^2(\sin^2\theta+\cos^2\theta)
    = x^2 + y^2
\end{align*}
}

La matriz $\mmm{A}$ definida en~\eqref{eq:rot-matrizA} es el generador
de una rotación activa, que es la opuesta del generador de rotaciones
pasivas, $\mmm{G}$, en $\symbb{R}^2$
\[
  \mmm{A}
  = \begin{pNiceMatrix}
    0 & -1\\
    1 & 0
  \end{pNiceMatrix}
  = -\begin{pNiceMatrix}
    0 & 1\\
    -1 & 0
  \end{pNiceMatrix}
  = -\mmm{G}
\]
La matriz de rotación activa que hemos utilizado se puede expresar en función
del generador  de rotaciones pasivas\footnotemark{}
\footnotetext{Una rotación pasiva de números complejos se representaría como
  $e^{-i\theta}$.}
\[
  \mmm{R}(\theta) = e^{\theta\!\mmm{A}}
  = e^{-\theta\!\mmm{G}}
\]
donde el generador de la rotación pasiva es
\[
  \mmm{G}
  = \begin{pNiceMatrix}
    0 & 1\\
    -1 & 0
    \end{pNiceMatrix}
\]
y la rotación pasiva se representa como $e^{\theta\!\mmm{G}}$.

\section{Transformaciones de Lorentz}
Vamos a intentar ir un poco más allá. ¿Y si buscamos una matriz $\mmm{B}$,
cuyo cuadrado no sea $-\mmm{I}$, sino $+\mmm{I}$?

Tomamos una matriz $\mmm{B}$ simétrica, $\mmm{B}^\trasp = \mmm{B}$, cuyo
determinante valga -1. No nos sirve la matriz unidad, porque su determinante
vale +1. Hagamos
\[
  \mmm{B}
  = \begin{pNiceMatrix}
    0 & 1\\
    1 & 0
  \end{pNiceMatrix}
\]

Su determinante tiene el valor que buscamos
\[
  \det\mmm{B}
  = \begin{vNiceMatrix}
    0 & 1\\
    1 & 0
  \end{vNiceMatrix}
  = -1
\]

Y comprobamos que su cuadrado vale $\mmm{I}$
\[
  \mmm{B}^2
  = \begin{pNiceMatrix}
    0 & 1\\
    1 & 0
  \end{pNiceMatrix}
  \begin{pNiceMatrix}
    0 & 1\\
    1 & 0
  \end{pNiceMatrix}
  = \begin{pNiceMatrix}
    1 & 0\\
    0 & 1
  \end{pNiceMatrix}
  = \mmm{I}
\]

Ahora podríamos construir un operador de transformación como $e^{\eta\!\mmm{B}}$,
donde $\eta\in\symbb{R}$ y la Desarrollamos en serie de Taylor alrededor $\eta=0$
\begin{align*}
  e^{\eta\!\mmm{B}}
  &=
    \frac{(\eta\mmm{B})^0}{0!}
    + \frac{(\eta\mmm{B})^1}{1!}
    + \frac{(\eta\mmm{B})^2}{2!}
    + \frac{(\eta\mmm{B})^3}{3!}
    + \frac{(\eta\mmm{B})^4}{4!}
    + \frac{(\eta\mmm{B})^5}{5!}
    + \cdots\\
  &=
    \mmm{I}
    + \eta\mmm{B}
    + \frac{\eta^2}{2!}\,\mmm{I}
    + \frac{\eta^3}{3!}\,\mmm{B}
    + \frac{\eta^4}{4!}\,\mmm{I}
    + \frac{\eta^5}{5!}\,\mmm{B}
    + \cdots\\
  &=
    \left(
    1 + \frac{\eta^2}{2!} + \frac{\eta^4}{4!} + \cdots
    \right) \mmm{I}
    + \left(
    \eta + \frac{\eta^3}{3!} + \frac{\eta^5}{5!} + \cdots
    \right)\,\mmm{B}\\
  &=
    \cosh\eta\,\mmm{I} + \sinh\eta\,\mmm{B}
    = \cosh\eta
    \begin{pNiceMatrix}
      1 & 0\\
      0 & 1
    \end{pNiceMatrix}
    + \sinh\eta
    \begin{pNiceMatrix}
      0 & 1\\
      1 & 0
    \end{pNiceMatrix}\\
  &=
    \begin{pNiceMatrix}
      \cosh\eta & 0\\
      0 & \cosh\eta
    \end{pNiceMatrix}
    +
    \begin{pNiceMatrix}
      0 & \sinh\eta\\
      \sinh\eta & 0
    \end{pNiceMatrix}
  = \begin{pNiceMatrix}
    \cosh\eta & \sinh\eta\\
    \sinh\eta & \cosh\eta
  \end{pNiceMatrix}
\end{align*}

¿Cuál sería el invariante de esta transformación?
La aplicamos a un vector
\[
  \begin{pNiceMatrix}
    x'\\
    y'
  \end{pNiceMatrix}
  = \begin{pNiceMatrix}
    \cosh\eta & \sinh\eta\\
    \sinh\eta & \cosh\eta
  \end{pNiceMatrix}
  \begin{pNiceMatrix}
    x\\
    y
  \end{pNiceMatrix}
  = \begin{pNiceMatrix}
    x\cosh\eta + y\sinh\eta\\
    x\sinh\eta + y\cosh\eta
  \end{pNiceMatrix}
\]

Calculamos el cuadrado de las componentes del vector transformado
\begin{align*}
  (x')^2
  &=
    (x\cosh\eta + y\sinh\eta)^2
    = x^2\cosh^2\eta + y^2\sinh^2\eta + 2xy\sinh\eta\cosh\eta\\
  (y')^2
  &=
    (x\sinh\eta + y\cosh\eta)^2
    = x^2\sinh^2\eta + y^2\cosh^2\eta + 2xy\sinh\eta\cosh\eta    
\end{align*}

Si sumáramos $(x')^2 + (y')^2$ no nos daría un invariante porque
el resultado dependería del parámetro $\eta$, de manera que
$e^{\eta\mmm{B}}$ no describe una rotación, como la de las secciones
anteriores (rotación en un espacio euclídeo).

Si los restamos sí obtendríamos un invariante
\begin{align*}
  (x')^2 - (y')^2
  &=
    x^2 (\cosh^2\eta - \sinh^2\eta) + y^2 (\sinh\eta - \cosh\eta)\\
  &=
    x^2 \cdot 1 + y^2 \cdot (-1)
  = x^2 - y^2
\end{align*}
porque el resultado solo depende de las componentes del vector original $(x,y)$
y no del parámetro $\eta$.
Este \emph{tipo de rotación} no se da en un espacio euclídeo, sino en otro
tipo de espacio. Representa una transformación de Lorentz que tiene lugar en
relatividad especial.

Como conclusión de este capítulo vemos que:
\begin{itemize}
\item Si queremos una rotación, \emph{nos inventamos} una matriz generadora
  antisimétrica, con determinante unidad, cuyo cuadrado sea $-\mmm{I}$.
\item Si queremos una transformación de Lorentz, la matriz debe ser simétrica,
  su determinante debe valer \emph{menos uno}, y su cuadrado vale $+\mmm{I}$.
\end{itemize}


%%% Local Variables:
%%% coding: utf-8
%%% mode: latex
%%% TeX-engine: luatex
%%% TeX-master: "../cuarentena.tex"
%%% End:

% LaTeX-command: "lualatex --shell-escape"