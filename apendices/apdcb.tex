% apendiceA.tex
%
% Copyright (C) 2022 José A. Navarro Ramón <janr.devel@gmail.com>
% 1) Código LuaLatex:
%    Licencia GPL-2.
% 2) Producto en pdf, postscript, etc.:
%    Licencia Creative Commons Recognition Share alike. (CC-BY-SA)

\chapter{Matriz densidad: Código fuente}
\label{chapt:apcua-matriz-densidad-codigo}

\section{Estado mezcla: lanzamiento de dado}
\label{sect:apcua-estadomezcla-dado}

\subsection{Código de \texttt{maxima}:
  fichero './soft/dado-dos-par.wxmx'.}
\label{subsect:apcua-estadomezcla-dado-maxima}

El siguiente código se encuentra en el fichero './soft/dado-dos-par.wxmx'
% \lstset{maxima}
%\begin{tcblisting}{listing only, listing options={language=python}, title=Titulo}
%  /* Carga de módulos */
%  load("eigen")$
%  load("nchrpl")$
%
%  /* Función que construye la matriz densidad diagonalizada */
%  matdiagonalizada(matriz, EVs) := block(
%     [i, n, mtrzdiag, es, ms],
%     i: 1; n: length(matriz), mtrzdiag: zeromatrix(n,n),
%     es: EVs[1][1], ms: EVs[1][2],
%     for is: 1 thru length(ms) do
%        for c: 1 thru ms[is] do (
%           mtrzdiag[i,i]: es[is], i: i + 1
%        ),
%     mtrzdiag
%  )$
%     
%  /* Vectores de estado |2> y |p> */
%  v2: matrix([0],[1],[0],[0],[0],[0])$
%  vp: matrix([0],[1],[0],[1],[0],[1])$
%  vp: unitvector(vp)$
%   
%  /* Producto escalar no nulo: vectores no ortogonales */
%  v2 . vp;
% 
%  /* Matrices densidad de los estados puros y de la mezcla.
%     Rango de la matriz densidad:
%       - Igual a uno:   estado puro.
%       - Mayor que uno: estado mezcla. */
%  rho1: v2 . conjugate(transpose(v2))$
%  rho2: vp . conjugate(transpose(vp))$
%  rho: 1/6 * rho1 + 1/2 * rho2$ rank(rho);
% 
%/* Matriz diagonalizada */
%  EVs: uniteigenvectors(rho)$
%  D: matdiagonalizada(rho, EVs)$
% 
% /* Entropía (S)
%     es:   lista de autovalores.
%     nums: lista de autovalores distintos de cero. */
%  es: EVS[1][1]$
%  nums: delete(0, es)$
%  S: - apply("+", nums * log(nums))$
%  float(s);
%\end{tcblisting}

\vbox{
}
\begin{tcblisting}{%
   colback=yellow!10,colframe=black!50,left=1.8em,listing only,
   hbox,enhanced, drop fuzzy shadow,
   listing options={%
     language=c,
     numbers=left,
     numberstyle=\scriptsize\color{red!50},
     basicstyle=\footnotesize,
   },
   every listing line={\ttfamily}
 }
 /* Carga de módulos */
 load("eigen")$ load("nchrpl")$

 /* Función que construye la matriz densidad diagonalizada */
 matdiagonalizada(matriz, EVs) := block(
    [i, n, mtrzdiag, es, ms],
    i: 1; n: length(matriz), mtrzdiag: zeromatrix(n,n),
    es: EVs[1][1], ms: EVs[1][2],
    for is: 1 thru length(ms) do
       for c: 1 thru ms[is] do (
          mtrzdiag[i,i]: es[is], i: i + 1
       ),
       mtrzdiag
 )$
 
 /* Vectores de estado |2> y |p> */
 v2: matrix([0],[1],[0],[0],[0],[0])$
 vp: matrix([0],[1],[0],[1],[0],[1])$
 vp: unitvector(vp)$

 /* Producto escalar no nulo: vectores no ortogonales */
 v2 . vp;

 /* Matrices densidad de los estados puros y de la mezcla.
    Rango de la matriz densidad:
      - Igual a uno:   estado puro.
      - Mayor que uno: estado mezcla. */
 rho1: v2 . conjugate(transpose(v2))$
 rho2: vp . conjugate(transpose(vp))$
 rho: 1/6 * rho1 + 1/2 * rho2$ rank(rho);

 /* Matriz diagonalizada */
 EVs: uniteigenvectors(rho)$
 D: matdiagonalizada(rho, EVs)$
 
 /* Entropía (S)
    es:   lista de autovalores.
    nums: lista de autovalores distintos de cero. */
 es: EVS[1][1]$
 nums: delete(0, es)$
 S: - apply("+", nums * log(nums))$
 float(s);
\end{tcblisting}

%\begin{tcblisting}{%
%    colback=yellow!10,colframe=black!50,left=1.8em,listing only,
%    hbox,enhanced, drop fuzzy shadow, breakable,
%    listing options={%
%      language=c,
%      numbers=left,
%      numberstyle=\scriptsize\color{red!50},
%      basicstyle=\footnotesize,
%    },
%    every listing line={\ttfamily}
%  }
%  /* Carga de módulos */
%  load("eigen")$ load("nchrpl")$
%%
%  /* Función que construye la matriz densidad diagonalizada */
%  matdiagonalizada(matriz, EVs) := block(
%     [i, n, mtrzdiag, es, ms],
%     i: 1; n: length(matriz), mtrzdiag: zeromatrix(n,n),
%     es: EVs[1][1], ms: EVs[1][2],
%     for is: 1 thru length(ms) do
%        for c: 1 thru ms[is] do (
%           mtrzdiag[i,i]: es[is], i: i + 1
%        ),
%     mtrzdiag
%  )$
%%     
%  /* Vectores de estado |2> y |p> */
%  v2: matrix([0],[1],[0],[0],[0],[0])$
%  vp: matrix([0],[1],[0],[1],[0],[1])$
%  vp: unitvector(vp)$
%%   
%  /* Producto escalar no nulo: vectores no ortogonales */
%  v2 . vp;
%% 
%  /* Matrices densidad de los estados puros y de la mezcla.
%     Rango de la matriz densidad:
%       - Igual a uno:   estado puro.
%       - Mayor que uno: estado mezcla. */
%  rho1: v2 . conjugate(transpose(v2))$
%  rho2: vp . conjugate(transpose(vp))$
%  rho: 1/6 * rho1 + 1/2 * rho2$ rank(rho);
%% 
%%/* Matriz diagonalizada */
%%  EVs: uniteigenvectors(rho)$
%%  D: matdiagonalizada(rho, EVs)$
%% 
%% /* Entropía (S)
%%     es:   lista de autovalores.
%%     nums: lista de autovalores distintos de cero. */
%%  es: EVS[1][1]$
%%  nums: delete(0, es)$
%%  S: - apply("+", nums * log(nums))$
%%  float(s);
%\end{tcblisting}

%\begin{tcblisting}{%
%   colback=yellow!10,colframe=black!50,left=1.8em,listing only,
%   hbox,enhanced, drop fuzzy shadow, breakable,
%   listing options={%
%     language=c,
%     numbers=left,
%     numberstyle=\scriptsize\color{red!50},
%     basicstyle=\footnotesize,
%   },
%   every listing line={\ttfamily}
% }
% /* Carga de módulos */
% load("eigen")$ load("nchrpl")$
%
% /* Función que construye la matriz densidad diagonalizada */
% matdiagonalizada(matriz, EVs) := block(
%    [i, n, mtrzdiag, es, ms],
%    i: 1; n: length(matriz), mtrzdiag: zeromatrix(n,n),
%    es: EVs[1][1], ms: EVs[1][2],
%    for is: 1 thru length(ms) do
%       for c: 1 thru ms[is] do (
%          mtrzdiag[i,i]: es[is], i: i + 1
%       ),
%    mtrzdiag
% )$
%    
% /* Vectores de estado |2> y |p> */
% v2: matrix([0],[1],[0],[0],[0],[0])$
% vp: matrix([0],[1],[0],[1],[0],[1])$
% vp: unitvector(vp)$
%
% /* Producto escalar no nulo: vectores no ortogonales */
% v2 . vp;
%
% /* Matrices densidad de los estados puros y de la mezcla.
%    Rango de la matriz densidad:
%      - Igual a uno:   estado puro.
%      - Mayor que uno: estado mezcla. */
% rho1: v2 . conjugate(transpose(v2))$
% rho2: vp . conjugate(transpose(vp))$
% rho: 1/6 * rho1 + 1/2 * rho2$ rank(rho);
%
% /* Matriz diagonalizada */
% EVs: uniteigenvectors(rho)$
% D: matdiagonalizada(rho, EVs)$
% 
% /* Entropía (S)
%    es:   lista de autovalores.
%    nums: lista de autovalores distintos de cero. */
% es: EVS[1][1]$
% nums: delete(0, es)$
% S: - apply("+", nums * log(nums))$
% float(s);
%\end{tcblisting}

\subsection{Código de \texttt{sagemath}}
\label{subsect:apcua-estadomezcla-dado-sagemath}
El siguiente código se encuentra en el fichero './soft/dado-dos-par.sage'
%\lstset{maxima}
\begin{tcblisting}{%
   colback=yellow!10,colframe=black!50,left=1.8em,listing only,
   hbox,enhanced, drop fuzzy shadow,
   listing options={%
     language=c,
     numbers=left,
     numberstyle=\scriptsize\color{red!50},
     basicstyle=\footnotesize,
   },
   every listing line={\ttfamily}
 }
 # Vectores
 v1 = vector([0,1,0,0,0,0])
 v2 = vector([0,1,0,1,0,1])
 v1 = (v1/v1.norm()).column()
 v2 = (v2/v2.norm()).column()
 # Matriz densidad
 rho1 = v1 * v1.transpose().conjugate()
 rho2 = v2 * v2.transpose().conjugate()
 rho = 1/6 * rho1 + 1/2 * rho2
 # Matriz diagonalizada: 'diag'
 # Matriz de vectores propios: 'vect'
 D = rho.eigenmatrix_right()
 diag = D[0]; vect = D[1]
 # Obtenemos la lista de valores propios.
 # (Si fuera un proceso computacionalmente costoso,
 # sería preferible extraer los elementos de la
 # matriz diagonal 'D')
 eigenvals = rho.eigenvalues()
 S = -sum([eig*log(eig) for eig in eigenvals if eig != 0])
 print(S.n(digits=20))
\end{tcblisting}


%\section{Problema con el desarrollo original de
%  \mathinhead{ \tilde{\Psi}(\vvv{\tilde{r}})}{ddpsir}
%en los vídeos}








%%% Local Variables:
%%% coding: utf-8
%%% mode: latex
%%% TeX-engine: luatex
%%% TeX-master: "../cuarentena.tex"
%%% End:

