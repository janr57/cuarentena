% apendiceA.tex
%
% Copyright (C) 2022 José A. Navarro Ramón <janr.devel@gmail.com>
% 1) Código LuaLatex:
%    Licencia GPL-2.
% 2) Producto en pdf, postscript, etc.:
%    Licencia Creative Commons Recognition Share alike. (CC-BY-SA)

\chapter{Vectores y valores propios de las matrices de Pauli}
\label{chapt:apcA-eigenvalues-eigenvectors-Pauli}

\section{Procedimiento para obtener vectores y valores propios}
Supongamos la matriz genérica $\mmm{\sigma_i}$, $i=1,2,3$.
La ecuación de autovalores y autovectores es
\[
  \mmm{\sigma_i} \vvv{v} = \lambda \vvv{v}
\]
\[
  \mmm{\sigma_i} \vvv{v} - \lambda \mmm{I}\vvv{v} = \vvv{0}
\]
\begin{equation}\label{eq:apcA-eigen-general}
  [\mmm{\sigma_i} -\lambda\mmm{I}] \vvv{v} = \vvv{0}
\end{equation}

A continuación aplicamos esta ecuación a cada matriz de Pauli.

\subsection{Primera matriz de Pauli}
\label{sect:apcA-Pauli-sigma1}
Desarrollamos la expresión \eqref{eq:apcA-eigen-general} para la primera
matriz de Pauli, $\mmm{\sigma_1}$
\[
  \left[
  \begin{pNiceMatrix}
    0 & 1\\
    1 & 0
  \end{pNiceMatrix}
  -
  \begin{pNiceMatrix}
    \lambda & 0\\
    0 & \lambda
  \end{pNiceMatrix}
\right]
\begin{pNiceMatrix}
  v_1\\
  v_2
\end{pNiceMatrix}
=
\begin{pNiceMatrix}
  0\\
  0
\end{pNiceMatrix}
\]
\[
  \begin{pNiceMatrix}
    -\lambda & 1\\
    1 & -\lambda
  \end{pNiceMatrix}
\begin{pNiceMatrix}
  v_1\\
  v_2
\end{pNiceMatrix}
=
\begin{pNiceMatrix}
  0\\
  0
\end{pNiceMatrix}
\]
\[
  \begin{pNiceMatrix}
    -\lambda v_1 + v_2\\
    v_1 - \lambda v_2
  \end{pNiceMatrix}
  =
\begin{pNiceMatrix}
  0\\
  0
\end{pNiceMatrix}
\]
Tenemos dos ecuaciones con tres incógnitas
\begin{equation}\label{eq:apcA-ecs-sigma1}
  \left\{
    \begin{array}{r@{}r@{}r@{}r@{}}
      -\lambda v_1 & {} + {} & v_2 & {} = 0\\
      v_1 & {} - {} & \lambda v_2 & {} = 0
    \end{array}
    \right.
\end{equation}

De la segunda ecuación despejamos $v_1 = \lambda v_2$ y sustituimos
en la primera, obteniendo
\[
  -\lambda^2 v_2 = v_2
\]
y
\[
  \lambda^2 = 1
\]
Como resultado, tenemos dos valores propios posibles $\lambda = -1$ y
$\lambda = +1$

\begin{itemize}
\item Si $\lambda = -1$

  Sustituimos este valor de $\lambda$ en \eqref{eq:apcA-ecs-sigma1}
  \[
    v_1 + v_2 = 0
  \]

  Si hacemos $v_1 = N$, entonces $v_2 = -N$ y el vector $\vvv{v}$
  \[
    \vvv{v} =
    \begin{pNiceMatrix}
      N \\
      -N
    \end{pNiceMatrix}
    = N
    \begin{pNiceMatrix}
      1\\
      -1
    \end{pNiceMatrix}
  \]

  Normalizando
  \[
    \vvv{v}
    =
    \frac{1}{\sqrt{2}}
    \begin{pNiceMatrix}
      1\\
      -1
    \end{pNiceMatrix}
  \]
  
  \item Si $\lambda = +1$

  Sustituimos este valor de $\lambda$ en \eqref{eq:apcA-ecs-sigma1}
  \[
   -v_1 + v_2 = 0
  \]

  Si hacemos $v_1 = N$, entonces $v_2 = N$ y el vector $\vvv{v}$
  \[
    \vvv{v} =
    \begin{pNiceMatrix}
      N \\
      N
    \end{pNiceMatrix}
    = N
    \begin{pNiceMatrix}
      1\\
      1
    \end{pNiceMatrix}
  \]

  Si normalizamos
  \[
    \vvv{v}
    =
    \frac{1}{\sqrt{2}}
    \begin{pNiceMatrix}
      1\\
      1
    \end{pNiceMatrix}
  \]

\end{itemize}

Las dos soluciones son (sin normalización)
\[
  \begin{pNiceMatrix}
    0 & 1\\
    1 & 0
  \end{pNiceMatrix}
  \begin{pNiceMatrix}
    1\\
    -1
  \end{pNiceMatrix}
  = -1
  \begin{pNiceMatrix}
    1\\
    -1
  \end{pNiceMatrix}
\]
\[
  \begin{pNiceMatrix}
    0 & 1\\
    1 & 0
  \end{pNiceMatrix}
  \begin{pNiceMatrix}
    1\\
    1
  \end{pNiceMatrix}
  = +1
  \begin{pNiceMatrix}
    1\\
    1
  \end{pNiceMatrix}
\]

\subsection{Segunda matriz de Pauli}
\label{sect:apcA-Pauli-sigma2}
Desarrollamos la expresión \eqref{eq:apcA-eigen-general} para la segunda
matriz de Pauli, $\mmm{\sigma_2}$
\[
  \left[
  \begin{pNiceMatrix}
    0 & -i\\
    i & 0
  \end{pNiceMatrix}
  -
  \begin{pNiceMatrix}
    \lambda & 0\\
    0 & \lambda
  \end{pNiceMatrix}
\right]
\begin{pNiceMatrix}
  v_1\\
  v_2
\end{pNiceMatrix}
=
\begin{pNiceMatrix}
  0\\
  0
\end{pNiceMatrix}
\]
\[
  \begin{pNiceMatrix}
    -\lambda & -i\\
    i & -\lambda
  \end{pNiceMatrix}
\begin{pNiceMatrix}
  v_1\\
  v_2
\end{pNiceMatrix}
=
\begin{pNiceMatrix}
  0\\
  0
\end{pNiceMatrix}
\]
\[
  \begin{pNiceMatrix}
    -\lambda v_1 -i v_2\\
    iv_1 - \lambda v_2
  \end{pNiceMatrix}
  =
\begin{pNiceMatrix}
  0\\
  0
\end{pNiceMatrix}
\]
Tenemos dos ecuaciones con tres incógnitas
\begin{equation}\label{eq:apcA-ecs-sigma2}
  \left\{
    \begin{array}{r@{}r@{}r@{}r@{}}
      -\lambda v_1 & {} - {} & iv_2 & {} = 0\\
      iv_1 & {} - {} & \lambda v_2 & {} = 0
    \end{array}
    \right.
\end{equation}

De la segunda ecuación despejamos $v_2 = iv_1/\lambda$ y sustituimos
en la primera, obteniendo
\[
  \lambda v_1 + i(iv_1/\lambda)= 0
\]
\[
  \lambda^2 v_1 - v_1 = 0
\]
y
\[
  \lambda^2 = 1
\]
Como resultado, tenemos dos valores propios posibles $\lambda = -1$ y
$\lambda = +1$

\begin{itemize}
\item Si $\lambda = -1$

  Sustituimos este valor de $\lambda$ en \eqref{eq:apcA-ecs-sigma1}
  \[
    v_1 = iv_2
  \]

  Si hacemos $v_2 = N$, entonces $v_1 = iN$ y el vector $\vvv{v}$
  \[
    \vvv{v} =
    \begin{pNiceMatrix}
      iN \\
      N
    \end{pNiceMatrix}
    = N
    \begin{pNiceMatrix}
      i\\
      1
    \end{pNiceMatrix}
  \]

  
  Normalizando
  \[
    \vvv{v}
    =
    \frac{1}{\sqrt{2}}
    \begin{pNiceMatrix}
      i\\
      1
    \end{pNiceMatrix}
  \]

  \item Si $\lambda = +1$

  Sustituimos este valor de $\lambda$ en \eqref{eq:apcA-ecs-sigma1}
  \[
   v_2 = iv_1
  \]

  Si hacemos $v_1 = N$, entonces $v_2 = iN$ y el vector $\vvv{v}$
  \[
    \vvv{v} =
    \begin{pNiceMatrix}
      N \\
      iN
    \end{pNiceMatrix}
    = N
    \begin{pNiceMatrix}
      1\\
      i
    \end{pNiceMatrix}
  \]

  Si normalizamos
  \[
    \vvv{v}
    =
    \frac{1}{\sqrt{2}}
    \begin{pNiceMatrix}
      1\\
      i
    \end{pNiceMatrix}
  \]

\end{itemize}

Las dos soluciones son (sin normalización)
\[
  \begin{pNiceMatrix}
    0 & 1\\
    1 & 0
  \end{pNiceMatrix}
  \begin{pNiceMatrix}
    i\\
    1
  \end{pNiceMatrix}
  = -1
  \begin{pNiceMatrix}
    i\\
    1
  \end{pNiceMatrix}
\]
\[
  \begin{pNiceMatrix}
    0 & 1\\
    1 & 0
  \end{pNiceMatrix}
  \begin{pNiceMatrix}
    1\\
    i
  \end{pNiceMatrix}
  = +1
  \begin{pNiceMatrix}
    1\\
    i
  \end{pNiceMatrix}
\]

\subsection{Tercera matriz de Pauli}
\label{sect:apcA-Pauli-sigma3}
Desarrollamos la expresión \eqref{eq:apcA-eigen-general} para la primera
matriz de Pauli, $\mmm{\sigma_3}$
\[
  \left[
  \begin{pNiceMatrix}
    1 & 0\\
    0 & -1
  \end{pNiceMatrix}
  -
  \begin{pNiceMatrix}
    \lambda & 0\\
    0 & \lambda
  \end{pNiceMatrix}
\right]
\begin{pNiceMatrix}
  v_1\\
  v_2
\end{pNiceMatrix}
=
\begin{pNiceMatrix}
  0\\
  0
\end{pNiceMatrix}
\]
\[
  \begin{pNiceMatrix}
    1-\lambda & 0\\
    0 & -1-\lambda
  \end{pNiceMatrix}
\begin{pNiceMatrix}
  v_1\\
  v_2
\end{pNiceMatrix}
=
\begin{pNiceMatrix}
  0\\
  0
\end{pNiceMatrix}
\]
\[
  \begin{pNiceMatrix}
    (1-\lambda) v_1 = 0\\
    (1+\lambda) v_2 = 0
  \end{pNiceMatrix}
  =
\begin{pNiceMatrix}
  0\\
  0
\end{pNiceMatrix}
\]
Tenemos dos ecuaciones con tres incógnitas
\begin{equation}\label{eq:apcA-ecs-sigma1}
  \left\{
    \begin{array}{r@{}r@{}r@{}r@{}}
      (1-\lambda) v_1 & {}  {} &  & {} = 0\\
       & {}  {} & (1+\lambda) v_2 & {} = 0
    \end{array}
    \right.
\end{equation}

De la segunda ecuación despejamos $v_1 = \lambda v_2$ y sustituimos
en la primera, obteniendo
\[
  -\lambda^2 v_2 = v_2
\]
y
\[
  \lambda^2 = 1
\]
Como resultado, tenemos dos valores propios posibles $\lambda = -1$ y
$\lambda = +1$

\begin{itemize}
\item Si $\lambda = -1$

  Sustituimos este valor de $\lambda$ en \eqref{eq:apcA-ecs-sigma1}
  \[
    v_1 + v_2 = 0
  \]

  Si hacemos $v_1 = N$, entonces $v_2 = -N$ y el vector $\vvv{v}$
  \[
    \vvv{v} =
    \begin{pNiceMatrix}
      N \\
      -N
    \end{pNiceMatrix}
    = N
    \begin{pNiceMatrix}
      1\\
      -1
    \end{pNiceMatrix}
  \]

  Normalizando
  \[
    \vvv{v}
    =
    \frac{1}{\sqrt{2}}
    \begin{pNiceMatrix}
      1\\
      -1
    \end{pNiceMatrix}
  \]
  
  \item Si $\lambda = +1$

  Sustituimos este valor de $\lambda$ en \eqref{eq:apcA-ecs-sigma1}
  \[
   -v_1 + v_2 = 0
  \]

  Si hacemos $v_1 = N$, entonces $v_2 = N$ y el vector $\vvv{v}$
  \[
    \vvv{v} =
    \begin{pNiceMatrix}
      N \\
      N
    \end{pNiceMatrix}
    = N
    \begin{pNiceMatrix}
      1\\
      1
    \end{pNiceMatrix}
  \]

  Si normalizamos
  \[
    \vvv{v}
    =
    \frac{1}{\sqrt{2}}
    \begin{pNiceMatrix}
      1\\
      1
    \end{pNiceMatrix}
  \]

\end{itemize}

Las dos soluciones son (sin normalización)
\[
  \begin{pNiceMatrix}
    0 & 1\\
    1 & 0
  \end{pNiceMatrix}
  \begin{pNiceMatrix}
    1\\
    -1
  \end{pNiceMatrix}
  = -1
  \begin{pNiceMatrix}
    1\\
    -1
  \end{pNiceMatrix}
\]
\[
  \begin{pNiceMatrix}
    0 & 1\\
    1 & 0
  \end{pNiceMatrix}
  \begin{pNiceMatrix}
    1\\
    1
  \end{pNiceMatrix}
  = +1
  \begin{pNiceMatrix}
    1\\
    1
  \end{pNiceMatrix}
\]


%\section{Problema con el desarrollo original de
%  \mathinhead{ \tilde{\Psi}(\vvv{\tilde{r}})}{ddpsir}
%en los vídeos}








%%% Local Variables:
%%% coding: utf-8
%%% mode: latex
%%% TeX-engine: luatex
%%% TeX-master: "../cuarentena.tex"
%%% End:

