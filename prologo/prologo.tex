% rotaciones.tex
%
% Copyright (C) 2022 José A. Navarro Ramón <janr.devel@gmail.com>
% 1) Código LuaLatex:
%    Licencia GPL-2.
% 2) Producto en pdf, postscript, etc.:
%    Licencia Creative Commons Recognition Share alike. (CC-BY-SA)

\chapter{Prólogo}
Este texto consiste en unos apuntes personales tomados de unas lecciones en
vídeo publicadas en
\href{https://www.youtube.com/playlist?list=PLAnA8FVrBl8BcfT7FX7Q4sKwHlaG-0Ej4}
{Youtube}
por \href{https://www.patreon.com/JavierGarcia/}{Javier García},
durante la cuarentena de 2020, debida al Covid-19, originado por el
virus SARS-CoV-2.

Muchas personas que pretenden aprender física de un cierto nivel, de forma
autodidacta, se dan cuenta de que es muy fácil terminar perdidas y de que
cuesta mucho avanzar.

El autor de las lecciones nos explica conceptos que tienen difícil cabida en
textos de la materia. Además, parte de conocimientos básicos de bachillerato o
de primero de carrera y desde mi punto de vista consigue su objetivo.

El objetivo principal es la confección de unos apuntes que me sirvan para
repasar con detenimiento el contenido de los vídeos. Si al lector le interesa
el contenido de estas notas, recomiendo atender en primer lugar a las
lecciones en vídeo porque, en caso contrario se perdería de algún modo su
objetivo.

Los apuntes se confeccionan utilizando Lua\LaTeX (de TeXLive 2022) y, por
ahora no considero que estén listos, pues están siendo continuamente corregidos
con cada relectura del material. Iré añadiendo más contenido conforme disponga
de tiempo (digamos que este no me sobra).
La licencia del código fuente es GPL-v2.

En cambio, para el resultado final ---el libro, en cualquier formato, como pdf,
postscript, etc.---, utilizo la licencia Creative Commons, con reconocimiento y
compartir igual (CC-BY-SA).
Además, he de hacer constar que el autor de las lecciones en vídeo no promociona
estos apuntes, de hecho, ni siquiera los conoce (por lo menos hasta la fecha).

He intentado mantener el guión de las lecciones, adaptándolo a mis necesidades.
Por eso, a veces he suprimido algunos detalles porque los tenía muy
interiorizados y no necesitaba copiarlos; otras veces he añadido partes de mi
cosecha pues necesitaba hacer hincapié en algún concepto que necesitaba
afianzar y otras veces he cambiado el orden, para adaptarlo a mi forma de
asimilar la materia.

Inevitablemente habré cometido errores. Si el lector es tan amable, podría
señalármelos para que pudiera mejorar el texto.

%Gracias, Javier.

 
%%% Local Variables:
%%% mode: latex
%%% TeX-engine: luatex
%%% TeX-master: "../cuarentena.tex"
%%% End:
